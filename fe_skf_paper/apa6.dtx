% \iffalse meta-comment
% !TEX program  = pdfLaTeX
% %%%%%%%%%%%%%%%%%%%%%%%%%%%%%%%%%%%%%%%%%%%%%%%%%%%%%%%%%%%%%%%%%%%%%%%%%%%
%                                                                           %
%    CHANGE THESE VALUES WITH EACH NEW RELEASE:                             %
%                                                                           %
%<class>\ProvidesClass{apa6}[2020/01/13 v2.34 APA formatting (6th edition)]
%                                                                           %
%<*internal>                                                                %
\def\apaSixVersionDate{2020/01/13}
\def\apaSixVersionNumber{2.34}
%                                                                           %
%                                                                           %
% %%%%%%%%%%%%%%%%%%%%%%%%%%%%%%%%%%%%%%%%%%%%%%%%%%%%%%%%%%%%%%%%%%%%%%%%%%%
\iffalse
%</internal>
%<*readme>
----------------------------------------------------------------------

apa6 - A LaTeX class for formatting documents in compliance with the
American Psychological Association's Publication Manual, 6th edition

Copyright (C) 2011-2020 by Brian D. Beitzel <brian at beitzel.com>

This work may be distributed and/or modified under the
conditions of the LaTeX Project Public License (LPPL), either
version 1.3c of this license or (at your option) any later
version.  The latest version of this license is in the file:

http://www.latex-project.org/lppl.txt

Users may freely modify these files without permission, as long as the
copyright line and this statement are maintained intact.

This work is not endorsed by, affiliated with, or probably even known
by, the American Psychological Association.

----------------------------------------------------------------------

The Publication Manual of the American Psychological Association is
widely used in the social sciences.  The most recent update, in 2009,
altered the formatting guidelines and therefore rendered existing
formatting solutions inadequate.  The apa6 class is an update of
older code from the apa class (available from CTAN), which is no
longer being maintained.  New features have also been added, such as
suppressing references to one's own work to facilitate masked reviews
of manuscripts by independent reviewers.  See the documentation for
details.
%</readme>
%<*internal>
\fi
\def\nameofplainTeX{plain}
\ifx\fmtname\nameofplainTeX\else
  \expandafter\begingroup
\fi
%</internal>
%<*install>
\input docstrip.tex
\keepsilent
\askforoverwritefalse
\preamble
----------------------------------------------------------------------

apa6 - A LaTeX class for formatting documents in compliance with the
American Psychological Association's Publication Manual, 6th edition

Copyright (C) 2011-2020 by Brian D. Beitzel <brian at beitzel.com>

This work may be distributed and/or modified under the
conditions of the LaTeX Project Public License (LPPL), either
version 1.3c of this license or (at your option) any later
version.  The latest version of this license is in the file:

http://www.latex-project.org/lppl.txt

Users may freely modify these files without permission, as long as the
copyright line and this statement are maintained intact.

This work is not endorsed by, affiliated with, or probably even known
by, the American Psychological Association.

----------------------------------------------------------------------

\endpreamble
\postamble

Copyright (C) 2011-2020 by Brian D. Beitzel <brian at beitzel.com>

This work may be distributed and/or modified under the
conditions of the LaTeX Project Public License (LPPL), either
version 1.3c of this license or (at your option) any later
version.  The latest version of this license is in the file:

http://www.latex-project.org/lppl.txt

Users may freely modify these files without permission, as long as the
copyright line and this statement are maintained intact.

This work is not endorsed by, affiliated with, or probably even known
by, the American Psychological Association.


This work is "maintained" (as per LPPL maintenance status) by
Brian D. Beitzel.

This work consists of the file  apa6.dtx
and the derived files           apa6.ins,
                                apa6.cls,
                                apa6.pdf,
                                README,
                                APAamerican.txt,
                                APAbritish.txt,
                                APAdutch.txt,
                                APAenglish.txt,
                                APAgerman.txt,
                                APAngerman.txt,
                                APAgreek.txt,
                                APAczech.txt,
                                APAturkish.txt,
                                APAendfloat.cfg,
                                apa6.ptex,
                                TeX2WordForapa6.bas,
                                Figure1.pdf,
                                shortsample.tex,
                                longsample.tex, and
                                bibliography.bib.

\endpostamble
\usedir{tex/latex/apa6}
\generate{
  \file{\jobname.cls}{\from{\jobname.dtx}{class}}
}
\usedir{tex/latex/apa6/config}
\generate{
  \file{./config/APAamerican.txt}{\from{\jobname.dtx}{american}}
  \file{./config/APAbritish.txt}{\from{\jobname.dtx}{british}}
  \file{./config/APAdutch.txt}{\from{\jobname.dtx}{dutch}}
  \file{./config/APAenglish.txt}{\from{\jobname.dtx}{english}}
  \file{./config/APAgerman.txt}{\from{\jobname.dtx}{german}}
  \file{./config/APAngerman.txt}{\from{\jobname.dtx}{ngerman}}
  \file{./config/APAgreek.txt}{\from{\jobname.dtx}{greek}}
  \file{./config/APAczech.txt}{\from{\jobname.dtx}{czech}}
  \file{./config/APAturkish.txt}{\from{\jobname.dtx}{turkish}}
  \file{./config/APAendfloat.cfg}{\from{\jobname.dtx}{APAendfloat}}
}
\usedir{tex/latex/apa6/samples}
\generate{
  \file{./samples/bibliography.bib}{\from{\jobname.dtx}{bibliography}}
  \file{./samples/shortsample.tex}{\from{\jobname.dtx}{shortsample}}
  \file{./samples/longsample.tex}{\from{\jobname.dtx}{longsample}}
}
\usedir{tex/latex/apa6/pseudoTeX}
\generate{
  \file{./pseudoTeX/apa6.ptex}{\from{\jobname.dtx}{ptex}}
  \file{./pseudoTeX/TeX2WordForapa6.bas}{\from{\jobname.dtx}{bas}}
}
%</install>
%<install>\endbatchfile
%<*internal>
\usedir{source/latex/apa6}
\generate{
  \file{\jobname.ins}{\from{\jobname.dtx}{install}}
}
\nopreamble\nopostamble
\usedir{doc/latex/apa6}
\generate{
  \file{README.txt}{\from{\jobname.dtx}{readme}}
}
\ifx\fmtname\nameofplainTeX
  \expandafter\endbatchfile
\else
  \expandafter\endgroup
\fi
%</internal>
%<class>\NeedsTeXFormat{LaTeX2e}
%<*driver>
\documentclass{ltxdoc}
\usepackage[T1]{fontenc}
\usepackage{lmodern}
\usepackage{hyperref}
\usepackage{graphicx}
\usepackage{booktabs}
\EnableCrossrefs
\CodelineIndex
\OnlyDescription  % don't publish the full code in the PDF documentation file
\RecordChanges
\begin{document}
  \DocInput{\jobname.dtx}
\end{document}
%</driver>
% \fi
%
%\DoNotIndex{\#,\$,\%,\&,\@,\\,\{,\},\^,\_,\~,\ }
%\DoNotIndex{\@ne}
%\DoNotIndex{\advance,\begingroup,\catcode,\closein}
%\DoNotIndex{\closeout,\day,\def,\edef,\else,\empty,\endgroup}
%\DoNotIndex{\newcommand,\renewcommand}
%
% \ifpdf
% \hypersetup{%
%   pdfauthor   = {Brian D. Beitzel <brian@beitzel.com>},
%   pdftitle    = {The apa6 class},
%   pdfsubject  = {Documentation of LaTeX class apa6},
%   pdfkeywords = {LaTeX, APA}
% }%
% \fi
%
% \title{^^A
%     Formatting documents in APA style (6th Edition)\\^^A
%     with the \textsf{apa6} \LaTeX\ class^^A
%     \thanks{This file describes version ^^A
%     \apaSixVersionNumber, last revised \apaSixVersionDate.}^^A
% }
% \author{Brian D.\ Beitzel\thanks{E-mail: brian@beitzel.com}}
% \date{Released \apaSixVersionDate}
%
% \maketitle
%
% \changes{v1.0}{2011/11/24}{Initial release}
%
% \changes{v1.01}{2011/11/28}{Prepended config files with `APA' to
% prevent possible conflicts along the \LaTeX\ path.}
% \changes{v1.01}{2011/11/28}{Minor changes in `apa6.ptex' to
% accommodate the \cs{note} command.}
%
% \changes{v1.02}{2011/12/01}{More reliable font-size selection}
% \changes{v1.02}{2011/12/01}{Corrected default mode to jou}
%
% \changes{v1.11}{2011/12/10}{New option `noextraspace'}
% \changes{v1.11}{2011/12/10}{Added German language support}
% \changes{v1.11}{2011/12/10}{Removed automated rotation of wide figures}
% \changes{v1.11}{2011/12/10}{Load lmodern and fontenc by default}
% \changes{v1.11}{2011/12/10}{Allow whitespace after subsubsection
% (and lower) sectioning commands}
%
% \changes{v1.12}{2011/12/12}{Fixed bug that erroneously displayed the
% first figure on the same page as the references (when no tables were
% present)}
%
% \changes{v1.13}{2012/01/16}{Removed duplicate definitions of
% ifapamodeXXX}
% \changes{v1.13}{2012/01/16}{Removed redundant definition of
% LT@makecaption}
% \changes{v1.13}{2012/01/16}{Removed hyphenation of figure and table
% captions in man mode}
% \changes{v1.13}{2012/01/16}{Increased spacing on top and bottom of
% tables (below caption and above table notes, if present)}
%
% \changes{v1.2}{2012/02/15}{Added `APAbritish.txt' configuration
% file}
% \changes{v1.2}{2012/02/15}{Version 6.00 (2012/02/14) or later of the
% apacite package is now required}
%
% \changes{v1.25}{2012/02/23}{Fixed incompatibility with hyperref package}
% \changes{v1.25}{2012/02/23}{Localized `Keywords' label (in config file)}
%
% \changes{v1.27}{2012/03/24}{Added `APAczech.txt' configuration}
%
% \changes{v1.3}{2012/04/04}{Added `a4paper' option}
%
% \changes{v2.0}{2012/10/20}{EndNote integration via `apa6.ptex' file
% and `FormatTex2WordDocument' macro}
% \changes{v2.0}{2012/10/20}{Fixed `headwidth' for `doc' mode}
% 
% \changes{v2.01}{2012/10/25}{Really fixed `headwidth' for `doc' mode}
% 
% \changes{v2.05}{2012/10/26}{Resolved the EndNote @@author-year
% specification causing no files to be found}
%
% \changes{v2.10}{2012/12/21}{Package `biber' now required when using
% biblatex}
%
% \changes{v2.10}{2012/12/21}{apa6.ptex file now honors `mask' option}
%
% \changes{v2.11}{2012/12/29}{Refined handling of `mask' option in the
% apa6.ptex file}
%
% \changes{v2.12}{2013/01/09}{Fixed `jou' mode to handle a starting
% page other than 1}
%
% \changes{v2.13}{2013/06/26}{Fixed bug for `natbib' bibliography heading}
%
% \changes{v2.13}{2013/06/26}{Disabled overwriting of the `enumerate'
% and `itemize' environments by `APAenumerate' and `APAitemize'; now
% the `enumerate' and `itemize' environments are passed to standard
% \LaTeX\ and will nest properly.}
%
% \changes{v2.14}{2014/11/20}{Fixed compatiblity with `nameref' for
% appendices}
% \changes{v2.14}{2014/11/20}{Fixed bug specifying the headwidth for
% `doc' format}
%
% \changes{v2.20}{2016/05/26}{Added Turkish localization file}
% \changes{v2.20}{2016/05/26}{Corrected `doc' mode to be one-sided}
%
% \changes{v2.21}{2016/07/02}{Added code to be more friendly to
% rotated tables in `man' mode}
%
% \changes{v2.22}{2016/07/02}{Bug fix for `floatsintext' option with
% sideways tables}
%
% \changes{v2.23}{2016/10/31}{Bug fix for rotated tables in `man'
% mode}
%
% \changes{v2.30}{2016/11/15}{Suppress trailing period for section
% headings ending with a question mark}
%
% \changes{v2.31}{2017/05/30}{Tweak to allow the last reference to be
% properly indented on the final line of the reference list}
%
% \changes{v2.32}{2017/06/20}{Minor adjustments to line spacing (e.g.,
% between footnotes) for uniformity across manuscript elements; and
% making footnotes ragged-right justified}
%
% \changes{v2.33}{2018/09/01}{Eliminated extra blank page before
% the References page under some circumstances}
%
% \changes{v2.34}{2020/01/13}{Provided the full title as the short
% title if the full title is 50 or fewer characters}
%
% \changes{v2.34}{2020/01/13}{Changed biblatex loading to freeze to
% APA 6th Edition compatibility}
%
% \begin{abstract}
%   The \textit{Publication Manual} of the American Psychological
%   Association is widely used in the social sciences.  The most
%   recent update, in 2009, altered the formatting guidelines and
%   therefore rendered existing formatting solutions inadequate.  The
%   \textsf{apa6} class is an update of older code from the
%   \textsf{apa} class, which is no longer being maintained.  New
%   features have also been added, such as suppressing references to
%   one's own work to facilitate masked reviews of manuscripts by
%   independent reviewers.
% \end{abstract}
%
% \section{Background}
% Most journals in the social sciences require manuscripts to be
% formatted in compliance with the American Psychological
% Association's \textit{Publication Manual}, which is updated
% periodically.  The 6th Edition, released in 2009, substantially
% changed the guidelines for formatting manuscripts; these
% modifications rendered existing formatting solutions (e.g., the
% \textsf{apa} \LaTeX\ class) inadequate for venues in which 6th
% Edition guidelines are being enforced.  The \textsf{apa6} class
% solves this problem, and provides some new functionality not offered
% by the \textsf{apa} class.
%
% \section{Disclaimer}
% Great care has been taken to ensure the closest possible match
% between APA requirements and the output of this class.  However, it
% is the sole responsibility of the user to ensure compliance with
% specific journal submission requirements!
%
% \section{Usage}
% \label{sec:usage}
%
% \subsection{Class Options}
% \label{sec:classoptions}
% When loading \textsf{apa6} with
% |\documentclass|\oarg{options}|{apa6}|, the following options are
% available.
%
% \vspace{1em}
% \textbf{Document mode:} Three choices are available.
% \begin{itemize}
% \item \DescribeMacro{jou}|jou| (default): Formats the document with an
%   appearance resembling a printed APA journal (e.g., \textit{Journal
%   of Educational Psychology}.  The text is typeset in two-sided,
%   two-column format.
% \item \DescribeMacro{man}|man|: Formats the document in close (if
%   not complete) compliance with the requirements for submission to
%   an APA journal (e.g., title page, double-spacing, etc.).
% \item \DescribeMacro{doc}|doc|: Formats the document as a typical
%   \LaTeX\ document (one-sided, single-column, etc.)
% \end{itemize}
%
% \textbf{Other class options:}
% \begin{itemize}
% \item \DescribeMacro{10pt}|10pt|: Typesets the document in 10-point
%   font.
% \item \DescribeMacro{11pt}|11pt|: Typesets the document in 11-point
%   font.
% \item \DescribeMacro{12pt}|12pt|: Typesets the document in 12-point
%   font.
% \item \DescribeMacro{a4paper}|a4paper|: Specifies A4 paper size
%   (letter is default).
% \item \DescribeMacro{nolmodern}|nolmodern|: Suppresses loading of the
%   \textsf{lmodern} package.
% \item \DescribeMacro{nofontenc}|nofontenc|: Suppresses loading of the
%   \textsf{fontenc} package (which is needed for proper hyphenation
%   of accented characters).
% \item \DescribeMacro{babel}|babel|: In all modes, loads
%   \textsf{babel}; the desired language(s) are listed as options
%   immediately following |babel|; the last language listed is the
%   main one.
% \item \DescribeMacro{noextraspace}|noextraspace|: In |man| mode,
%   removes some of the vertical space between certain elements (e.g.,
%   headers and text) in an attempt to more closely resemble true
%   double-spacing (use at your own risk).
% \item \DescribeMacro{floatsintext}|floatsintext|: In |man| mode,
%   integrates floats (tables and figures) within the body of the text
%   instead of postponing them until after the reference list.
% \item \DescribeMacro{biblatex}|biblatex|: Loads the
%   \textsf{biblatex} package; see Section~\ref{sec:biblatexinfo} for
%   details.
% \item \DescribeMacro{apacite}|apacite|: Loads the \textsf{apacite}
%   package; see Section~\ref{sec:apaciteinfo} for details.
% \item \DescribeMacro{natbib}|natbib|: See
%   Section~\ref{sec:natbibinfo} for details.
% \item \DescribeMacro{mask}|mask|: Masks references that are marked
%   as the author's own (for masked peer review); see
%   Section~\ref{sec:masking} for details.
% \item \DescribeMacro{longtable}|longtable|: If you \textit{must} use
%   long tables (exceeding one page in length), try this option (but
%   it may not work in all contexts).  Do \textbf{\textit{not}} load
%   \textsf{longtable} yourself because of precedence requirements
%   with the \textsf{endfloat} package.  Copy the file
%   |APAendfloat.cfg| from the ``config'' folder of your \textsf{apa}
%   installation to the working folder of your document (not in your
%   |texmf| tree), and rename it to |endfloat.cfg| so that endfloat
%   will recognize it.  The supplied |APAendfloat.cfg| file will also
%   be necessary in conjunction with the \textsf{rotating} package
%   (and its |\sideways| command) to produce rotated tables; this
%   works for |man| mode only.  For rotated tables in |jou| or |doc|
%   mode, the \textsf{rotating} package may be used.  If sideways
%   tables get pushed to the end of your document with the
%   |floatsintext| option, try using the |ph!| placement specifier for
%   your sideways table(s); for example, |\begin{sidewaystable}[ph!]|.
% \item \DescribeMacro{notxfonts}|notxfonts|: In |jou| mode, prevents
%   \textsf{txfonts} from loading, in case \textsf{pslatex} or
%   \textsf{times} is preferable for some reason.
% \item \DescribeMacro{notimes}|notimes|: In |jou| mode, cancels
%   loading \textsf{txfonts} or \textsf{pslatex} or \textsf{times} and
%   uses Computer Modern instead.
% \item \DescribeMacro{notab}|notab|: In |jou| mode, cancels the
%   automatic stretching of tabular environments to the width of their
%   enclosing float.
% \item \DescribeMacro{helv}|helv|: In |man| mode,
%   uses Helvetica font instead of Computer Modern.
% \item \DescribeMacro{nosf}|nosf|: In |man| mode,
%   neutralizes the |\helvetica| command.
% \item \DescribeMacro{tt}|tt|: In |man| mode, uses typewriter-like
%   font.
% \item \DescribeMacro{draftfirst}|draftfirst|: In all modes, places
%   the word ``DRAFT'' as a watermark across the first page.
% \item \DescribeMacro{draftall}|draftall|: In all modes, places the
%   word ``DRAFT'' as a watermark across all pages.
% \end{itemize}
%
% Class options not handled by \textsf{apa6} (e.g., |draft|) will be
% passed on to the \textsf{article} class.
%
% \subsection{Document Preamble}
% \label{sec:preamble}
% The following commands are available within the document preamble
% (i.e., the part of the file preceding |\begin{document}|).
%
% \begin{itemize}
% \item \DescribeMacro{\title}|\title|\marg{document-title}: The title of the document
% \item \DescribeMacro{\shorttitle}|\shorttitle|\marg{short-title}: A
%   shortened version of the title (for page headers)
% \item \DescribeMacro{\author}|\author|\marg{author(s)}: Author name(s)
% \end{itemize}
%
% For authors across multiple affiliations, follow these formats,
% noting that authors must be matched in sequence with their
% affiliations in the |\affiliation| command (hence multiple authors
% inside some braces represent multiple authors from the same
% institution):\par
% |\twoauthors{First Author(s)}{Second Author(s)}|\par
% |\threeauthors{John and Jim}{Mary and Sue}{Nick}|\par
% |\fourauthors{Helen}{Dick}{Tracy and Larry}{James Bond}|\par
% |\fiveauthors{...}{...}{...}{...}{...}|\par
% |\sixauthors{...}{...}{...}{...}{...}{...}|\par
% \begin{itemize}
% \item
%   \DescribeMacro{\leftheader}|\leftheader|\marg{author-last-name(s)}:
%   Author last name(s) (for even-page headers in |jou| mode)
% \item
%   \DescribeMacro{\affiliation}|\affiliation|\marg{affiliation(s)}:
%   Author affiliation(s)
% \end{itemize}
%
% For multiple affiliations, follow these formats:\par
% |\twoaffiliations{Affil. of 1st Author(s)}{Affil. 2nd Author(s)}|\par
% |\threeaffiliations{U of A}{U of B}{U of C}|\par
% |\fouraffiliations{My Company}{Your Department}{Heaven}{Earth}|\par
% |\fiveaffiliations{...}{...}{...}{...}{...}|\par
% |\sixaffiliations{...}{...}{...}{...}{...}{...}|\par
% \begin{itemize}
% \item \DescribeMacro{\abstract}|\abstract|\marg{abstract-text}: The
%   abstract of the article
% \item \DescribeMacro{\keywords}|\keywords|\marg{keywords}: Keywords
%   (typeset after the abstract) See Section~\ref{sec:keywords} for
%   details regarding language localization of the ``Keywords'' label.
% \item \DescribeMacro{\authornote}|\authornote|\marg{author-note}:
%   The Author Note, containing contact information, acknowledgements,
%   etc.
% \end{itemize}
%
% Optional; use if desired:\par
% \begin{itemize}
% \item \DescribeMacro{\note}|\note|\marg{note-text}: Notation of
%   manuscript date or other information desired beneath the
%   affiliation line
% \item \DescribeMacro{\journal}|\journal|\marg{journal-name}: Journal
%   name or other note; typeset in the top left header of page 1
%   (|jou| and |doc| modes only); to change the starting page to a
%   number other than 1, insert the following line immediately after
%   |\maketitle|:\\
%   |\setcounter{page}|\marg{custom-page-number}
% \item \DescribeMacro{\volume}|\volume|\marg{journal-volume}: Volume,
%   number, pages; typeset in the top left header in |jou| and |doc|
%   modes, underneath the content of |\journal|
% \item \DescribeMacro{\ccoppy}|\ccoppy|\marg{copright-notice}:
%   Copyright notice, etc.; typeset in the top right header of page 1
%   (|jou| and |doc| modes only)
% \item \DescribeMacro{\copnum}|\copnum|\marg{more-copyright-info}:
%   Any additional text needed; typeset in the top right header in
%   |jou| and |doc| modes, underneath the content of |\ccoppy|
% \end{itemize}
%
% \subsection{Maketitle}
% The \DescribeMacro{\maketitle}|\maketitle| command formats the
% document title, page headers, author list, author affiliations,
% Author Note (if provided), abstract according to whether |jou|,
% |man|, or |doc| mode has been specified.  This command should be
% on the line after |\begin{document}|, with the first line of text
% immediately following the |\maketitle| line (no blank lines).
%
% \subsection{Heading Levels}
% \label{sec:headinglevels}
% Heading levels are automatically formatted using the following
% standard \LaTeX\ commands:
% \begin{itemize}
% \item \DescribeMacro{\section}|\section|\marg{title}
% \item \DescribeMacro{\subsection}|\subsection|\marg{title}
% \item \DescribeMacro{\subsubsection}|\subsubsection|\marg{title}
% \item \DescribeMacro{\paragraph}|\paragraph|\marg{title}
% \item \DescribeMacro{\subparagraph}|\subparagraph|\marg{title}
% \end{itemize}
%
% Please note that sections cannot be |\ref|'d since APA style does
% not use numbered sections.  So |\label| commands are unnecessary
% unless you wish to use |\refname|.
%
% \subsection{Enumeration}
% \label{sec:enumeration}
% Several forms of enumeration are provided, as follows.
%
% \vspace{0.6em}
% \DescribeMacro{\begin\{seriate\}} \DescribeMacro{\end\{seriate\}}
%   |Blah blah blah|\par
%   |\begin{seriate}|\par
%   |  \item first item,|\par
%   |  \item second item.|\par
%   |\end{seriate}|\par
%   |Blah blah blah|\par\vspace{0.6em}
% \noindent results in:\par\vspace{0.6em}
%   |Blah blah blah (a) first item, (b) second item.  Blah blah blah|
%
% \vspace{3em}
% \DescribeMacro{\begin\{APAenumerate\}} \DescribeMacro{\end\{APAenumerate\}}
%   |Blah blah blah|\par
%   |\begin{APAenumerate}|\par
%   |  \item first item ... ... ... continue continue|\par
%   |  \item second item ... ... ... continue continue|\par
%   |\end{APAenumerate}|\par
%   |Blah blah blah|\par\vspace{0.6em}
% \noindent results in:\par\vspace{0.6em}
%   |Blah blah blah|\par
%   |    1. first item ... ... ...|\par
%   |continue continue|\par
%   |    2. second item ... ... ...|\par
%   |continue continue|\par
%   |Blah blah blah|\par
%
% \vspace{3em}
% \DescribeMacro{\begin\{APAitemize\}} \DescribeMacro{\end\{APAitemize\}}
%   |Blah blah blah|\par
%   |\begin{APAitemize}|\par
%   |  \item first item ... ... ... continue continue|\par
%   |  \item second item ... ... ... continue continue|\par
%   |\end{APAitemize}|\par
%   |Blah blah blah|\par\vspace{0.6em}
% \noindent results in:\par\vspace{0.6em}
%   |Blah blah blah|\par
%   |    o first item ... ... ...|\par
%   |continue continue|\par
%   |    o second item ... ... ...|\par
%   |continue continue|\par
%   |Blah blah blah|\par
%
% \vspace{0.6em}
% In addition to the above, all standard \LaTeX\ enumeration
% environments are available (e.g., |enumerate| and |itemize|).
%
% \subsection{Other Macros}
% \label{sec:othermacros}
% \begin{itemize}
% \item \DescribeMacro{\begin\{figure*\}} \DescribeMacro{\end\{figure*\}}
%   \DescribeMacro{\begin\{table*\}} \DescribeMacro{\end\{table*\}}
%   When a figure is too wide for a single column (in |jou| mode), use
%   |\begin{figure*}| and |\end{figure*}| instead of the non-starred
%   version.  The same applies with |\begin{table*}| and |\end{table*}|.
%   When using double-column tables or figures (|jou| mode), use the
%   |\centering| command; for example:\par
%   |\begin{table*}|\par\vspace{-.5em}
%   |  \centering|\par\vspace{-.5em}
%   |    \begin{threeparttable}|\par
% \item
%   \DescribeMacro{\fitfigure}|\fitfigure|\oarg{height}\marg{eps-filename}:
%   Automatically fit a postscript figure; use instead of
%   |\includegraphics|
% \item
%   \DescribeMacro{\fitbitmap}|\fitbitmap|\oarg{height}\marg{eps-filename}:
%   Same as |\fitfigure| but won't scale figure in |\man| mode for
%   best reproduction of bitmap figures
% \item \DescribeMacro{\tabfnm}|\tabfnm{a}|: Place a superscript
%   \textbf{f}oot\textbf{n}ote \textbf{m}ark inside a table cell.  Any
%   series of unique identifiers can be used in place of |a|.
% \item \DescribeMacro{\tabfnt}|\tabfnt{a}|\marg{footnote-text}:
%   Within table footnotes, specify the \textbf{f}oot\textbf{n}ote
%   \textbf{t}ext for |\tabfnm{a}|
% \item \DescribeMacro{\apavector}|\apavector|\marg{symbol}: Format
%   the \marg{symbol} as a vector by APA rules
% \end{itemize}
%
% \subsection{Appendices}
% \label{sec:appendices}
% \begin{itemize}
% \item \DescribeMacro{\appendix}|\appendix|: Begins the appendices
%   portion of the document
% \item \DescribeMacro{\section}|\section|\marg{appendix-title}:
%   Begins each appendix
% \end{itemize}
%
% Because appendices are numbered (with letters!) you may establish a
% label for each appendix (e.g., |\label{app:xxx}|); when there is
% more than one appendix, use |Appendix~\ref{app:xxx}| within the main
% body of the text to refer to that appendix.  (Of course, if there is
% only one appendix, simply refer to it as |the Appendix|.)
%
% \section{Known Limitations}
% \begin{itemize}
% \item There is a limit of six affiliations for authors (but an
%   unlimited number of authors across those six affiliations).
% \item The |APAenumerate| environment does not nest properly.
% \end{itemize}
%
% \section{Development of \textsf{apa6}}
% The base code for this class is the \textsf{apa} class, which in
% turn was based upon other sources.  In order to comply with 6th
% Edition criteria, certain changes had to be made to update the
% \textsf{apa} code.
%
% \subsection{Section Headings}
% Most prominently, the formatting of section headings had to be
% altered.  The 6th Edition specifies a more straightforward series of
% heading levels than previous editions did.  Briefly, the top-level
% heading is now boldfaced and centered, upper- and lower-case, no
% matter now many levels of heading are in the document; other heading
% levels have similar specifications.  The \textsf{apa6} class
% utilizes code (with permission) from the \textsf{apa6e} class to
% comply with all of these specifications.
%
% \subsection{Float Placement}
% The placement of floats (i.e., tables and figures) within an
% APA-style manuscript has also changed.  The 6th Edition requires
% that tables and figures (in that order) be placed after the
% references but before the appendices.  This creates a bit of a
% conundrum as to what should happen with tables or figures that are
% ultimately typeset within an appendix.  The choices we are left with
% are to place appendix floats (a) along with the floats from the main
% part of the manuscript, which would mean that appendix floats appear
% prior to the point at which they are mentioned; (b) within the
% appendices themselves, which is not consistent with how floats in
% the main part of the manuscript are handled; or (c) in a separate
% float section that follows the appendices, which results in two
% float sections.  Obviously none of these choices is satisfactory, so
% I posed the question to APA's Style Expert.  He responded that at
% least for APA's journals ``it doesn't matter whether appendix tables
% are submitted with text tables or separately, as long as they are
% numbered correctly (e.g., Table A1, Table B1, etc.)''  (J.\
% Hume-Pratuch, personal communication, June 15, 2011).  Therefore,
% \textsf{apa6} takes the most straightforward approach and includes
% all appendix floats within the body of the relevant appendix.  This
% also has the advantage of making appendices more readable.
%
% Because the 6th Edition requires figure captions to be printed on
% the same page as their respective figures, there are no more Figure
% Captions pages.
%
% \subsection{Author Note}
% According to 6th Edition guidelines, Author Notes are placed on the
% title page of manuscripts.
%
% \section{New features}
% In addition to providing compatibility with the 6th Edition of the
% \textit{Manual}, several new features have been implemented beyond
% those available in the \textsf{apa} class.
%
% \subsection{Masked References}
% \label{sec:masking}
% When manuscripts are sent out for review, they customarily must have
% all identifying information stripped so that reviewers do not know
% who the author of the manuscript is.  The new
% \DescribeMacro{mask}|mask| option suppresses the output of the
% author's name and affiliation, the author note, and any references
% that are marked as being the author's own.
%
% \begin{table}[htbp]
%   \caption{Supported masking commands}
%   \label{tab:CitationCommandsTable}
%   \begin{tabular}{@{}llll@{}}                                           \toprule
%     Unmasked Result    &  \multicolumn{3}{c}{Masking Commands}      \\ \cmidrule(r){2-4}
%                        &  \textsf{apacite}   &  \textsf{natbib}     &  \textsf{biblatex}  \\ \midrule
%     (van Dijk, 2001)  &  |\maskcite|        &  |\maskcitep|        &  |\maskparencite|   \\
%     (Van Dijk, 2001)  &                     &  |\maskCitep|        &  |\maskParencite|   \\
%     van Dijk, 2001    &  |\maskciteNP|      &  |\maskcitealp|      &  |\maskcite|        \\
%     Van Dijk, 2001    &                     &  |\maskCitealp|      &  |\maskCite|        \\
%     van Dijk (2001)   &  |\maskciteA|       &  |\maskcitet|        &  |\masktextcite|    \\
%     Van Dijk (2001)   &                     &  |\maskCitet|        &  |\maskTextcite|    \\
%     van Dijk          &  |\maskciteauthor|  &  |\maskciteauthor|   &  |\maskciteauthor|  \\
%     Van Dijk          &                     &  |\maskCiteauthor|   &  |\maskCiteauthor|  \\
%     (2001)             &  |\maskciteyear|    &  |\maskciteyearpar|  &                     \\
%     2001               &  |\maskciteyearNP|  &  |\maskciteyear|     &  |\maskciteyear|    \\ \bottomrule
%   \end{tabular}
% \end{table}
%
% To specify a reference that is to be masked with the |mask| option,
% simply prepend |mask| to the desired citation command (e.g.,
% |\maskcite| instead of |\cite|).  Supported bibliography packages
% are \textsf{biblatex}, \textsf{apacite}, and \textsf{natbib}.  The
% specific masking commands are shown in Table
% \ref{tab:CitationCommandsTable}.  Prenotes and postnotes for
% citations are supported with both \textsf{biblatex} and
% \textsf{natbib}; however, they are not supported with the
% \textsf{apacite} package.
%
% To mask citations that have been prepended with |mask| (and the
% corresponding bibliography entries), the name of the bibliography
% package must also be passed in as an option on the |\documentclass|
% line, in addition to the |mask| option.  For example,
% |\documentclass[jou,biblatex,mask]{apa6}|. (But as explained later,
% the |biblatex| option is the default and is therefore not required,
% even with the |mask| option.)
%
% \textbf{Be warned}, however, that if you have previously generated
% .bbl, etc., files without the |mask| option applied, those files
% must be deleted or re-written before \LaTeX-ing with the |mask|
% option---otherwise, you will see the to-be-masked entries showing up
% in the bibliography.
%
% Masked citations are replaced with the text, \textit{(2 citations
% removed for masked review)} (in the case of two masked citations).
% The corresponding entries in the References section are also
% suppressed.
%
% There is no need to revise the masked citations when removing the
% |mask| option for final production.  The citations that were
% previously masked will not be masked in the absence of the |mask|
% option.
%
% \subsection{Repositioned Floats}
% When revising and proofreading a manuscript, it is most helpful to
% have the tables and figures readily available (rather than turning
% most of the way to the end of the manuscript to access them).  The
% \DescribeMacro{floatsintext}|floatsintext| option (specified in the
% |\documentclass| line) will integrate tables and figures
% approximately where they are mentioned in the text.  This is
% available only with the |man| option, of course, because the other
% formats already have floats integrated with the text.
%
% \subsection{User-defined Font Size}
% \DescribeMacro{10pt} \DescribeMacro{11pt} \DescribeMacro{12pt} Users
% can now select from the font-size options available in standard
% \LaTeX\ (10pt, 11pt, 12pt) by including the appropriate option
% (e.g., |10pt|) in the |\documentclass| line.  The default font size
% is 10pt for |jou| mode, 11pt for |doc| mode, and 12pt for |man|
% mode.
%
% \subsection{Watermark}
% If desired, a ``DRAFT'' watermark can be placed on either the first
% pages or all pages of the document with the
% \DescribeMacro{draftfirst}|draftfirst| and
% \DescribeMacro{draftall}|draftall| options, respectively.  The text,
% font size, angle, and lightness of the text can all be modified
% using commands explained in the \textsf{draftwatermark}
% documentation.  Also loads the \textsf{everypage} package.
%
% \subsection{Flexible Bibliographies}
% The \textsf{apa6} class supports three bibliography packages:
% \textsf{biblatex}, \textsf{apacite}, and \textsf{natbib}.
%
% \subsubsection{\textsf{biblatex}}
% \label{sec:biblatexinfo}
% Biblatex is the most APA-compliant bibiliography package.  The
% \DescribeMacro{biblatex}|biblatex| option directs \textsf{apa6} to
% load the \textsf{biblatex} package with the following options:
% style=apa6,sortcites=true,sorting=nyt; however, the
% |\DeclareLanguageMapping| and |\addbibresource| commands will need
% to be specified by the user within the document preamble.  Please
% note that the \textsf{biblatex-apa6} package must be installed (but
% not explicitly loaded) for 6th Edition compatibility.  In the
% absence of bibliographic options, \textsf{apa6} will mask references
% using \textsf{biblatex} commands if the |mask| option is specified.
% In other words, the only reason to use the |biblatex| option is to
% have \textsf{apa6} automatically load the \textsf{biblatex} package.
%
% \subsubsection{\textsf{apacite}}
% \label{sec:apaciteinfo}
% The \textsf{apacite} package is loaded if the
% \DescribeMacro{apacite}|apacite| option is specified.  This option
% also informs \textsf{apa6} to mask references using \textsf{apacite}
% commands if the |mask| option is specified.  \textbf{The
% \textsf{apacite} package is \textit{not} loaded by default.}
%
% \subsubsection{\textsf{natbib}}
% \label{sec:natbibinfo}
% Specifying the \DescribeMacro{natbib}|natbib| option implicitly
% loads \textsf{apacite} and \textsf{natbib} and directs
% \textsf{apa6} to mask references using \textsf{natbib} commands if
% the |mask| option is specified.
%
% \subsection{Keywords}
% \label{sec:keywords}
% Many journals (including APA journals) request authors to provide
% keywords for their manuscripts to facilitate electronic indexing.
% \textsf{apa6} introduces the \DescribeMacro{\keywords}|\keywords|
% command.  If provided, keywords will be displayed on a line beneath
% the abstract.  For languages other than English, the `Keywords'
% label can be localized by modifying the appropriate configuration
% file (in the `config' folder of the \textsf{apa6} installation).
%
% \subsection{Converting to Microsoft Word\textregistered}
% A common requirement for manuscript submission is that the document
% be in Microsoft Word\textregistered\ format.  \LaTeX\ provides no
% easy way to convert to Word in APA format.  Several conversion
% utilities are available, but after researching and testing several
% of them, it seems that most of them---besides being incredibly
% challenging to implement---are incapable of formatting anything
% close to APA style.  One software package clearly stands out from
% the group, however: Chikrii Softlab's TeX2Word\texttrademark\
% (\url{http://www.chikrii.com/products/tex2word/}; compatible with
% Microsoft Windows\textregistered\ only).  The distinct advantage of
% this software is that it is extensible, allowing customizable
% interpretation of \LaTeX\ commands in a user-specified file---which
% is exactly what I have created for \textsf{apa6}.  The file
% |apa6.ptex| is included in the ``pseudoTeX'' subfolder of the
% \textsf{apa6} installation.  After installing TeX2Word (a 30-day
% trial period is available), find its ``pseudoTeX'' folder (which
% contains all the |.ptex| files automatically generated by TeX2Word),
% and copy |apa6.ptex| into that folder.  To convert your
% \textsf{apa6} \LaTeX\ document, start Microsoft Word (it must be
% 32-bit) and open your \textsf{apa6} document just as you would open
% any other document in Word.  The conversion process will run
% automatically and a mostly-ready document will be produced.
% Formatting of the title page, abstract page, section headings,
% double-spacing, table and figure captions, boldfaced and italicized
% text are all handled by the converter.  Instructions for finishing
% the conversion process will be displayed on the title page of the
% converted document. Mostly this involves moving floats (tables and
% figures) to their places toward the end of the manuscript, some
% table re-formatting, and editing of bibliographic information.  If
% you wish to have this editing more automated, you can open Word's
% Visual Basic Editor and import the ``TeX2WordForapa6.bas'' file
% (from the ``pseudoTeX'' subfolder of the \textsf{apa6} installation)
% as a new module, delete the lines beginning with percent signs at
% the top and bottom of the module, then run its
% |FormatTex2WordDocument| macro.  This will complete all of the above
% steps and configure the bibliographic information as temporary
% citations that EndNote can then interpret (see below for details).
% However, some minor editing is still necessary (e.g., table titles
% and footnotes are not moved).  For more efficient use of this macro,
% create a new Word document, open the Visual Basic Editor (VBE) and
% import the ``TeX2WordForapa6.bas'' file as a new module (be sure to
% delete all lines beginning with two percent signs); close the VBE
% and save the document in your ``Word Startup'' folder.  Then the
% |FormatTex2WordDocument| macro will be available from the Macros
% dialog in Word for all documents.
%
% The major weakness of TeX2Word at the moment is a near-total lack of
% support for bibliographic packages.  Version 2.0 of \textsf{apa6}
% introduced mechanisms in the |apa6.ptex| file and the
% |FormatTex2WordDocument| macro to establish compatibility with
% EndNote for handling bibliographic citations and the reference list
% when using TeX2Word.  The process works as follows: (a) in Word,
% open your |.tex| file and let TeX2Word do the conversion; (b) run
% the |FormatTex2WordDocument| macro as described above to convert
% citations to a format that EndNote recognizes as ``temporary
% citations''; (c) go to the EndNote toolbar within Word and run
% ``Update Citations and Bibliography''; for each temporary citation,
% EndNote will present a dialog containing the matched references from
% the EndNote library; for each matched reference, click ``Insert'';
% (d) after all citations have been resolved and EndNote has generated
% the References list at the end of the document, move the References
% list to the proper position if needed (i.e., when there are
% appendices, tables, or figures).  Unfortunately, this workaround
% does mean that duplicate databases will need to be maintained in
% both a |.bib| file and an EndNote library that contain common
% Bib\TeX\ keys for each reference.  To work properly, one
% modification is necessary to the user preferences in EndNote:
% un-check the ``Omit Author and/or Year from formatted citation if
% removed from temporary citation'' option (in the ``Formatting''
% section of EndNote preferences).
%
% It's well worth experimenting with the 30-day trial of TeX2Word, and
% perhaps even worth finding---or borrowing---a Windows machine if you
% don't have one readily available (the conversion process does not
% require a \TeX\ installation to be present).
%
%
% \section{Dependencies}
% \textsf{apa6} automatically loads the following packages.  If these
% packages are not already installed, producing the first
% \textsf{apa6} document could take a few minutes while these packages
% are downloaded and installed.
%
% \begin{itemize}
% \item \textsf{apacite}: bibliography package; used only if the
%   |apacite| or |natbib| option has been specified
% \item \textsf{biblatex}: bibliographic package; used only if the
%   |biblatex| option has been specified
% \item \textsf{biblatex-apa6}: bibliographic package; this package
%   must must be installed (but not explicitly loaded) for 6th Edition
%   compatibility
% \item \textsf{booktabs}:  formats tables that are much more
%   attractive than the standard \LaTeX\ tables
% \item \textsf{caption}: formats table and figure captions
% \item \textsf{draftwatermark}: includes a ``DRAFT'' watermark; used
%   only if the |draftfirst| or |draftall| options are specified
%   (\textsf{draftwatermark} automatically loads \textsf{everypage}
% \item \textsf{endfloat}: handles placing tables and figures at the
%   end of a manuscript; used only with the |man| option
% \item \textsf{etoolbox}: provides low-level hooks needed to detect
%   user-loaded packages
% \item \textsf{fancyhdr}: formats page headers
% \item \textsf{float}: handles floats placed within text; used only
%   when the |man| and |floatsintext| options are both specified
%   \textsf{biblatex} package (when loaded by the user in the document
%   preamble) in time to set the ``References'' heading to
%   non-boldface; used only if no bibliographic options have been
%   specified
% \item \textsf{geometry}: formats margins
% \item \textsf{graphicx}: allows inclusion of figures
% \item \textsf{longtable}: formats tables that exceed one page in
%   length; loaded only if the |longtable| option is specified.  Also
%   loads the \textsf{array} package.
% \item \textsf{lmodern}: needed for proper default text size for
%   draft watermark, so used on all documents; may be suppressed from
%   loading with option |nolmodern|
% \item \textsf{substr}: counts masked references; used only if the
%   |mask| option is specified
% \item \textsf{threeparttable}: produces nicely formatted table notes
%   that comply with APA style.  See |longsample.tex| (in the
%   ``samples'' subfolder of the \textsf{apa6} installation;
%   specifically, Appendix B in that document) for how to set up a
%   table with notes
% \item \textsf{times}: for math definitions
% \end{itemize}
%
%
% \section{Examples}
%
% Sample documents are included with this class; look in the
% ``samples'' subfolder of your installation. The source and output for
% |shortsample.tex| are reproduced on the following pages for quick
% reference.
%
% \newpage
%
% \subsection{shortsample.tex}
%
% \begin{verbatim}
% \documentclass[jou]{apa6}
%
% \usepackage[american]{babel}
%
% \usepackage{csquotes}
% \usepackage[style=apa6,sortcites=true,sorting=nyt,backend=biber]{biblatex}
% \DeclareLanguageMapping{american}{american-apa}
% \addbibresource{bibliography.bib}
%
% \title{Sample APA-Style Document Using the \textsf{apa6} Package}
%
% \author{Brian D.\ Beitzel}
% \affiliation{SUNY Oneonta}
%
% \leftheader{Beitzel}
%
% \abstract{This demonstration paper uses the \textsf{apa6} \LaTeX\
%   class to format the document in compliance with the 6th Edition of
%   the American Psychological Assocation's \textit{Publication Manual.}
%   The references are managed using \textsf{biblatex}.}
%
% \keywords{APA style, demonstration}
%
% \begin{document}
% \maketitle
% We begin with \textcite{Shotton1989}.  We can also cite this work in
% parenthesis, like this: \parencite{Shotton1989}.
%
% A three-author paper \parencite[e.g.,][]{Lassen2006} lists all
% three authors for the first citation, then only the first author
% on all subsequent citations \parencite{Lassen2006}.
%
% Note the use of five heading levels throughout this demonstration
% Method section.
%
% \section{Method}
% \subsection{Participants}
% We had a lot of people in this study.
%
% \subsection{Materials}
% Several materials were used for this project.  Some of them were
% already created for prior research.
%
% \subsubsection{Paper-and-Pencil Instrument}
% We used an instrument that we found to be highly successful.
%
% \paragraph{Reliability}
% The reliability of this instrument is extraordinary.
%
% \paragraph{Validity}
% We now discuss the validity of our instrument.
%
% \subparagraph{Face validity} The face validity is exceptionally
% strong.  Everyone should be impressed.
%
% \subparagraph{Construct validity} Also very strong.
%
% \subsection{Design}
% This section describes the study's design.
%
% \subsection{Procedure}
% The procedure was fairly straightforward, yet required
% attention to detail.
%
% \section{Results}
% Table \ref{tab:ComplexTable} contains some sample data.  Our
% statistical prowess in analyzing these data is unmatched.
%
% \begin{table}[htbp]
%   \vspace*{2em}
%   \begin{threeparttable}
%     \caption{A Complex Table}
%     \label{tab:ComplexTable}
%     \begin{tabular}{@{}lrrr@{}}         \toprule
%     Distribution type  & \multicolumn{2}{l}{Percentage of} & Total number   \\
%                        & \multicolumn{2}{l}{targets with}  & of trials per  \\
%                        & \multicolumn{2}{l}{segment in}    & participant    \\ \cmidrule(r){2-3}
%                                     &  Onset  &  Coda            &          \\ \midrule
%     Categorical -- onset\tabfnm{a}  &    100  &     0            &  196     \\
%     Probabilistic                   &     80  &    20\tabfnm{*}  &  200     \\
%     Categorical -- coda\tabfnm{b}   &      0  &   100\tabfnm{*}  &  196     \\ \midrule
%     \end{tabular}
%     \begin{tablenotes}[para,flushleft]
%         {\small
%             \textit{Note.} All data are approximate.
% 
%             \tabfnt{a}Categorical may be onset.
%             \tabfnt{b}Categorical may also be coda.
% 
%             \tabfnt{*}\textit{p} < .05.
%             \tabfnt{**}\textit{p} < .01.
%          }
%     \end{tablenotes}
%   \end{threeparttable}
% \end{table}
%
% \section{Discussion}
% This is a lengthy and erudite discussion.  It demonstrates amazing
% skill in interpreting the results for the masses.
%
% \printbibliography
%
% \end{document}
% \end{verbatim}
%
% \subsection{shortsample.pdf}
% \IfFileExists{samples/shortsample.pdf}%
%   {\includegraphics[scale=0.8, trim = 1in 1.5in 1in 0.8in]{samples/shortsample.pdf}}%
%   {PRINTED DOCUMENT WILL BE DISPLAYED HERE \newpage}
%
%
% \StopEventually{^^A
%   \PrintChanges
%   \PrintIndex
% }
%
%    \begin{macrocode}
%<*class>
%    \end{macrocode}
%
%\begin{macro}{apa6.cls}
%    \begin{macrocode}
% % Below is the license statement from the apa class, upon which this
% % apa6 class was built
% %%%%%%%%%%%%%%%%%%%%%%%%%%%%%%%%%%%%%%%%%%%%%%%%%%%%%%%%%%%%%%%%%%%%%%%%%%%%%%
%
% apa.cls  version 1.3.4
%
% NOTE CHANGE IN VERSION NUMBERING : 1.21--1.28 should have been 1.2.1--1.2.8
%
% Athanassios Protopapas, December 2008
% protopap@ilsp.gr
%
% This package may be distributed under the terms of the LaTeX Project Public
% License, as described in lppl.txt in the base LaTeX distribution.  Either
% version 1.0 or, at your option, any later version.
%
% The term "package", in the above license statement, applies to all files
% included with apa.cls (language files, examples, configuration,
% instructions etc.) -- they may be all distributed under LPPL.
%
% Use with LaTeX2e, \documentclass[man|jou|doc]{apa}
% Default mode is doc. Conforms to the APA manual 5th ed.
%
% ** Read apacls.txt and examples.txt carefully before using **
%
% %%%%%%%%%%%%%%%%%%%%%%%%%%%%%%%%%%%%%%%%%%%%%%%%%%%%%%%%%%%%%%%%%%%%%%%%%%%%%%

\DeclareOption{man}{%
  \def\def@man{\@manmode}
}

\DeclareOption{jou}{%
  \def\def@jou{\@joumode}
}

\DeclareOption{doc}{%
  \def\def@doc{\@docmode}
}

\DeclareOption{babel}{%
  \def\def@babel{\@babel}
}

\DeclareOption{notimes}{%
  \@ifundefined{def@jou}{}{\def\def@notimes{\@notimes}}
}

\DeclareOption{notxfonts}{% -- thp 2005/07/23
  \@ifundefined{def@jou}{}{\def\def@notxfonts{\@notxfonts}}
}

\DeclareOption{nosf}{%
  \@ifundefined{def@man}{}{\def\def@nosf{\@nosf}}
}

\DeclareOption{fignum}{%
  \@ifundefined{def@man}{}{\def\fig@num{\relax}}
}

% \DeclareOption{floatmark}{%
%   \@ifundefined{def@man}{}{\def\float@mark{\relax}}
% }

\DeclareOption{longtable}{%
  \def\long@table{\relax}
}

\DeclareOption{tt}{%
  \@ifundefined{def@man}{}{\def\tt@family{\relax}}
}

\DeclareOption{helv}{%
  \@ifundefined{def@man}{}{\def\helv@family{\relax}}
}

\DeclareOption{notab}{\def\no@tab{\relax}}

\DeclareOption{nobf}{\def\no@bf@title{\relax}}

\DeclareOption{nolmodern}{%
  \def\def@nolmodern{\@nolmodernmode}
}

\DeclareOption{nofontenc}{%
  \def\def@nofontenc{\@nofontencmode}
}

% remove extra space around headings, etc., in man mode
\DeclareOption{noextraspace}{%
  \def\def@noextraspace{\@noextraspacemode}
}

% suppress the title at the introduction (in case a different title is desired)
\DeclareOption{donotrepeattitle}{%
  \def\def@donotrepeattitle{\@donotrepeattitlemode}
}

% in order to have tables and figure inside the text, Guillaume Jourjon 19/10/10
\DeclareOption{floatsintext}{%
  \def\def@floatsintext{\@floatsintext}
}

% allow A4 paper size
\DeclareOption{a4paper}{%
  \def\def@aFourPaper{\@aFourPapermode}
}

\DeclareOption{apacite}{% BDB
  \def\def@apacite{\@apacitemode}
}

\DeclareOption{natbib}{% BDB
  \def\def@natbib{\@natbibmode}
}

\DeclareOption{biblatex}{% BDB
  \def\def@biblatex{\@biblatexmode}
}

\DeclareOption{draftfirst}{% BDB
  \def\def@draftfirst{\@draftfirstmode}
}

\DeclareOption{draftall}{% BDB
  \def\def@draftall{\@draftallmode}
}

\DeclareOption{mask}{\def\apaSix@maskauthoridentity{\relax}}  % BDB

% declare the font-size commands
\newcommand\apaSix@ptsize{}
\newcommand\apaSix@noptsize{}
\DeclareOption{10pt}{\renewcommand\apaSix@ptsize{10pt}}
\DeclareOption{11pt}{\renewcommand\apaSix@ptsize{11pt}}
\DeclareOption{12pt}{\renewcommand\apaSix@ptsize{12pt}}


\DeclareOption*{\PassOptionsToClass{\CurrentOption}{article}}

\ProcessOptions\relax

% make jou the default if no other options have been specified
\@ifundefined{def@man}{%
 \@ifundefined{def@doc}{%
  \@ifundefined{def@jou}{%
   \def\def@jou{\@joumode}
   \ClassInfo{apa6}{Using default mode (jou)}
   %\def\def@man{\@manmode}
   %\def\def@doc{\@docmode}
  }{}
 }{}
}{}

% pass the default font-size to the article class
\@ifundefined{def@man}{%
    \@ifundefined{def@jou}{%
        \@ifundefined{def@doc}{%
        }{% doc
          \ifx\apaSix@ptsize\apaSix@noptsize
            \LoadClass[11pt]{article} % default for doc is 11pt
          \else
            \LoadClass[\apaSix@ptsize]{article}
          \fi
        }
    }{% jou
      \ifx\apaSix@ptsize\apaSix@noptsize
        \LoadClass[10pt,twoside]{article} % default for jou is 10pt
      \else
        \LoadClass[\apaSix@ptsize,twoside]{article}
      \fi
    }
}{% man
  \ifx\apaSix@ptsize\apaSix@noptsize
    \LoadClass[12pt,twoside]{article} % default for man is 12pt
  \else
    \LoadClass[\apaSix@ptsize]{article}
  \fi
}


% determine the bibliography package, and load it if selected through the appropriate \documentclass option
\@ifundefined{def@apacite}{%
  \@ifundefined{def@natbib}{%
    \@ifundefined{def@biblatex}{%
      \def\def@biblatex{\@biblatexmode}%  the default bibliography package is biblatex
      \RequirePackage{etoolbox}
      \AtEndPreamble{%
        \@ifpackageloaded{biblatex}{%  the user has loaded biblatex
          \@ifundefined{def@man}{%
            \defbibheading{bibliography}{\section*{\normalfont\refname}}%
          }{%
            \defbibheading{bibliography}{\clearpage\section*{\normalfont\refname}}%
          }
        }{}
      }
      \ClassInfo{apa6}{No bibliography package was specified; defaulting to (but not loading) Biblatex}
    }{%
      \def\def@biblatex{\@biblatexmode}%  the selected bibliography package is Biblatex
      \RequirePackage[style=apa6,sortcites=true,sorting=nyt,backend=biber]{biblatex}
      \@ifundefined{def@man}{%
        \defbibheading{bibliography}{\section*{\normalfont\refname}}%
      }{%
        \defbibheading{bibliography}{\clearpage\section*{\normalfont\refname}}%
      }
      \ClassInfo{apa6}{The selected bibliography package, biblatex, has been loaded}
    }
  }{%
    \def\def@natbib{\@natbibmode}%  the selected bibliography package is natbib (with apacite)
    \@ifundefined{def@man}{%         -- thp 2005/07/23
      \RequirePackage[natbibapa]{apacite}[2012/02/25]}
      {\RequirePackage[natbibapa,bibnewpage]{apacite}[2012/02/25]}
    \ClassInfo{apa6}{The selected bibliography package, apacite and
      natbib, have been loaded}
  }
}{%
  \def\def@apacite{\@apacitemode}%  the selected bibliography package is apacite
  \@ifundefined{def@man}{%         -- thp 2005/07/23
    \RequirePackage{apacite}[2005/06/08]}
    {\RequirePackage[bibnewpage]{apacite}[2005/06/08]}
  \ClassInfo{apa6}{The selected bibliography package, apacite, has been loaded}
}

% load the lmodern package (needed for proper watermark font size)
\@ifundefined{def@nolmodern}{%
  \RequirePackage{lmodern}
}{}

% load the fontenc package (needed for proper hyphenation of accented characters)
\@ifundefined{def@nofontenc}{%
  \RequirePackage[T1]{fontenc}
}{}

% place a watermark on any pages indicated
\@ifundefined{def@draftall}{%
  \@ifundefined{def@draftfirst}{}{%
    \RequirePackage[firstpage]{draftwatermark}
    \SetWatermarkText{DRAFT}
  }
}{%
% a draft watermark has been requested for all pages
  \RequirePackage{draftwatermark}
  \SetWatermarkText{DRAFT}
}

%\@ifundefined{def@noapacite}%       -- Philip Kime 2008/12/03
%  {\ClassInfo{apa6}{Using BibTeX with apacite for citations and references}}%
%  {\ClassInfo{apa6}{apacite not loaded -- User may load BibLaTeX}}

% provide compatibility for authors who wish to use the \and command
% in author lists
\renewcommand\and{\vspace{0.25in}\\}

%==== Conditional compilation macros depending on mode ====

\long\def\ifapamodeman#1#2{\@ifundefined{def@man}{#2}{#1}}
\long\def\ifapamodejou#1#2{\@ifundefined{def@jou}{#2}{#1}}
\long\def\ifapamodedoc#1#2{\@ifundefined{def@doc}{#2}{#1}}
\long\def\ifapamode#1#2#3{%
 \@ifundefined{def@man}{%
  \@ifundefined{def@jou}{%
   \@ifundefined{def@doc}{\ClassError{apa6}{Undefined mode state!}}{#3}%
  }{#2}%
 }{#1}%
}

%=================================================================
\@ifundefined{def@man}{}{%
%
% endnotes and endfloat loaded later
% longtable must be loaded before endfloat
%
\@ifundefined{long@table}{}{%
 \RequirePackage{array}
 \RequirePackage{longtable}
%
% longtable messes with captions too, we need to intervene to match style
% NOTE: the caption package makes the following redundant
%\def\LT@makecaption#1#2#3{%
%  \LT@mcol\LT@cols c{\hbox to\z@{\hss\parbox{\linewidth}{%
%      \vskip 10pt%
%      #1{#2}%
%      \let\BBAB\table@BBAB%   -- thp 2005/07/23
%      \par\emph{#3}%
%      \let\BBAB\normal@BBAB%  -- thp 2005/07/23
%    \endgraf\vskip\baselineskip}%
%  \hss}}}
}% END of loading longtable
%
% Switch font family for manuscript if required
%
\@ifundefined{tt@family}{}{%
% A typewriter font can be used as in the APA manual example. Since tt has
% no bold (or bold-italic), the cmss font metrics are loaded instead
% \DeclareFontShape{OT1}{cmtt}{bx}{n}{ <-> sub * cmss/bx/n }{} % this fails
 \DeclareFontShape{OT1}{cmtt}{bx}{n}{ <-> cmssbx10 }{}  % probably not the
 \DeclareFontShape{OT1}{cmtt}{bx}{it}{ <-> cmssbxo10}{} % right way to do it
 \renewcommand{\familydefault}{cmtt}
 }
\@ifundefined{helv@family}{}{%
 \renewcommand{\familydefault}{phv}}
}


%
% For jou mode, load times (and related math fonts) if available
% First try txfonts because they are supposed to have better math definitions
%
\@ifundefined{def@jou}{}{%
 \@ifundefined{def@notimes}{%
  \newif\iftxfonts          % -- thp 2005/07/23
  \txfontsfalse             % added checks for txfonts because they may be undesirable
                            % for example, there are no Greek txfonts but there are times
  \IfFileExists{txfonts.sty}{\@ifundefined{def@notxfonts}{\txfontstrue}{}}{}
   \iftxfonts%
    \RequirePackage{txfonts}%
    \typeout{apa6.cls: Using txfonts}% Changed from Warning -- thp 2005/12/28
    %%%
    % According to Erik Meijer, txfonts causes problems if amsmath is loaded later
    % (i.e., via \usepackage by the user); instead of providing yet another option
    % to load amsmath by apa.cls, we adopt Erik's suggestion to undefine temporarily
    % the offending macros -- thp 2005/12/28
    \let\tempiint\iint\let\iint\undefined
    \let\tempiiint\iiint\let\iiint\undefined
    \let\tempiiiint\iiiint\let\iiiint\undefined
    \let\tempidotsint\idotsint\let\idotsint\undefined
    \let\tempopenbox\openbox\let\openbox\undefined
    \AtBeginDocument{%
     \let\iint\tempiint\let\tempiint\undefined
     \let\iiint\tempiiint\let\tempiiint\undefined
     \let\iiiint\tempiiiint\let\tempiiiint\undefined
     \let\idotsint\tempidotsint\let\tempidotsint\undefined
     \let\openbox\tempopenbox\let\tempopenbox\undefined
    }
    %%% end of taking care of txfonts problems
   \else%
    % if txfonts are not available/desirable, try pslatextimes/mathptm
    \IfFileExists{pslatex.sty}
     {\RequirePackage{pslatex}}
     % if pslatex is not available, try times/mathptm
     {\RequirePackage{times}
      \IfFileExists{mathptm.sty}{\RequirePackage{mathptm}}{}}%
   \fi% txfonts not available/desirable
 }{}% def@notimes
}% def@jou


% choose a paper size and set the page margins
\@ifundefined{def@aFourPaper}{
  \RequirePackage[top=1in, bottom=1in, left=1in, right=1in]{geometry}  
}{
  \RequirePackage[top=1in, bottom=1in, left=1in, right=1in,a4paper]{geometry}  
}

\RequirePackage{graphicx}  % this is for including graphics

\RequirePackage{booktabs}  % this is for nice-looking tables
\setlength{\abovetopsep}{0pt}  % set the distance between the table title and the table toprule
\setlength{\belowbottomsep}{0pt}  % set the distance between the table bottomrule and any notes

\RequirePackage[para,flushleft]{threeparttable}  % this is for nice-looking table footnotes, etc.
% need to redefine the tablenotes so they hug the bottom line of the table a little closer
% also, the manuscript format needs to have \raggedright
\@ifundefined{def@man}{% BDB
  \def\TPT@doparanotes{\par\vspace{-.5\baselineskip}% BDB
     \prevdepth\z@ \TPT@hsize
     \TPTnoteSettings
     \parindent\z@ \pretolerance 8
     \linepenalty 200
     \renewcommand\item[1][]{\relax\ifhmode \begingroup
         \unskip
         \advance\hsize 10em % \hsize is scratch register, based on real hsize
         \penalty -45 \hskip\z@\@plus\hsize \penalty-19
         \hskip .15\hsize \penalty 9999 \hskip-.15\hsize
         \hskip .01\hsize\@plus-\hsize\@minus.01\hsize
         \hskip 1em\@plus .3em
        \endgroup\fi
        \tnote{##1}\,\ignorespaces}%
     \let\TPToverlap\relax
     \def\endtablenotes{\par}%
  }
}{%
  \def\TPT@doparanotes{\par\vspace{-.4\baselineskip}% BDB
     \prevdepth\z@ \TPT@hsize
     \TPTnoteSettings
     \raggedright
     \parindent\z@ \pretolerance 8
     \linepenalty 200
     \renewcommand\item[1][]{\relax\ifhmode \begingroup
         \unskip
         \advance\hsize 10em % \hsize is scratch register, based on real hsize
         \penalty -45 \hskip\z@\@plus\hsize \penalty-19
         \hskip .15\hsize \penalty 9999 \hskip-.15\hsize
         \hskip .01\hsize\@plus-\hsize\@minus.01\hsize
         \hskip 1em\@plus .3em
        \endgroup\fi
        \tnote{##1}\,\ignorespaces}%
     \let\TPToverlap\relax
     \def\endtablenotes{\par}%
  }
}

%========= Language portability parameters ===========

% Language-specific definition of strings is handled externally -- thp 2005/12/30
% Based on handling of the same issue in apacite.sty

% First define the default American English macros in case no external file is present or needed
\def\rheadname{Running head}
\def\acksname{Author Note}
\def\keywordname{Keywords}
\def\notesname{Footnotes}

\AtBeginDocument{% so that we know what language is active in babel

% \typeout{\string\languagename\space is\space\languagename}
% \@ifundefined{iflanguage}% this is defined by babel, so it means babel is loaded
                           % Unfortunately, because babel is built into the format
                           % in modern distributions, \iflanguage is defined and
                           % \languagename contains whichever language happens to be
                           % last in the definition list, whether or not the babel
                           % package is loaded by the current document

 \@ifundefined{def@babel}% this is defined only if the user requested loading babel
  {\def\@apaSix@langfile{config/APAamerican.txt}}
  {\def\@apaSix@langfile{config/APA\languagename.txt}}
 \InputIfFileExists{\@apaSix@langfile}{}{%
  \ClassInfo{apa6}{Language definition file \@apaSix@langfile\space not found}
 }%
}

%========== Load babel if required ===================

\@ifundefined{def@babel}{}{% -- thp 2005/07/23
 \RequirePackage{babel}    % -- thp 2005/07/23, removed options 2005/12/28
}

\@ifundefined{def@biblatex}{}{% BDB

  % we are using biblatex

    %%%%%%%%%%%% biblatex commands %%%%%%%%%%%%%%%%
    %%
    %%  \cite[e.g.,][p.~11]{vanDijk2001,Ross1987}          =>  e.g., van Dijk, 2001; Ross, 1987, p. 11
    %%  \Cite[e.g.,][p.~11]{vanDijk2001,Ross1987}          =>  e.g., Van Dijk, 2001; Ross, 1987, p. 11
    %%  \parencite[e.g.,][p.~11]{vanDijk2001,Ross1987}     =>  (e.g., van Dijk, 2001; Ross, 1987, p. 11)
    %%  \Parencite[e.g.,][p.~11]{vanDijk2001,Ross1987}     =>  (e.g., Van Dijk, 2001; Ross, 1987, p. 11)
    %%  \textcite[e.g.,][p.~11]{vanDijk2001,Ross1987}      =>  e.g., van Dijk (2001); Ross (1987, p. 11)
    %%  \Textcite[e.g.,][p.~11]{vanDijk2001,Ross1987}      =>  e.g., Van Dijk (2001); Ross (1987, p. 11)
    %%  \citeauthor[e.g.,][p.~11]{vanDijk2001,Ross1987}    =>  e.g., van Dijk (2001); Ross (1987, p. 11)
    %%  \Citeauthor[e.g.,][p.~11]{vanDijk2001,Ross1987}    =>  e.g., Van Dijk (2001); Ross (1987, p. 11)
    %%  \citeyear[e.g.,][p.~11]{vanDijk2001}             =>  e.g., 2001, p. 11)
    %%  \footcite[e.g.,][p.~11]{vanDijk2001,Ross1987}      =>  e.g., van Dijk, 2001; Ross, 1987, p. 11. [as footnote]
    %%  \footcitetext[e.g.,][p.~11]{vanDijk2001,Ross1987}  =>  e.g., van Dijk, 2001; Ross, 1987, p. 11. [as footnotetext]
    %%
    %%%%%%%%%%%%%%%%%%%%%%%%%%%%%%%%%%%%%%%%%%%%%%%

    \@ifundefined{apaSix@maskauthoridentity}%  BDB
      {%  change masked references to unmasked
        \providecommand\maskcite\cite
        \providecommand\maskCite\Cite
        \providecommand\maskparencite\parencite
        \providecommand\maskParencite\Parencite
        \providecommand\masktextcite\textcite
        \providecommand\maskTextcite\Textcite
        \providecommand\maskciteauthor\citeauthor
        \providecommand\maskCiteauthor\Citeauthor
        \providecommand\maskciteyear\citeyear
        \providecommand\maskfootcite\footcite
        \providecommand\maskfootcitetext\footcitetext
      }{%  mask references to author

        \RequirePackage{substr}  % to allow counting of masked references
        \newcounter{maskedRefs}

        % \maskcite
        \newcommand\maskcite{\@ifnextchar[{\maskcite@@also}{\maskcite@@also[]}}
        \newcommand\maskcite@@also{}
        \def\maskcite@@also[#1]{\@ifnextchar[{\maskcite@@@also[#1]}{\maskcite@@@also[][#1]}}

        \def\maskcite@@@also%
            [#1][#2]#3{%
                \setcounter{maskedRefs}{0}%
                \SubStringsToCounter{maskedRefs}{,}{#3}%
                \addtocounter{maskedRefs}{1}%
                \ifnum\value{maskedRefs} = 1%
                \def\apaSix@masked@refs{(\textit{\themaskedRefs\ citation removed for masked review})}%
                \else%
                \def\apaSix@masked@refs{(\textit{\themaskedRefs\ citations removed for masked review})}%
                \fi%
                \apaSix@masked@refs%
        }

        % \maskCite
        \newcommand\maskCite{\@ifnextchar[{\maskCite@@also}{\maskCite@@also[]}}
        \newcommand\maskCite@@also{}
        \def\maskCite@@also[#1]{\@ifnextchar[{\maskCite@@@also[#1]}{\maskCite@@@also[][#1]}}

        \def\maskCite@@@also%
            [#1][#2]#3{%
                \setcounter{maskedRefs}{0}%
                \SubStringsToCounter{maskedRefs}{,}{#3}%
                \addtocounter{maskedRefs}{1}%
                \ifnum\value{maskedRefs} = 1%
                \def\apaSix@masked@refs{(\textit{\themaskedRefs\ citation removed for masked review})}%
                \else%
                \def\apaSix@masked@refs{(\textit{\themaskedRefs\ citations removed for masked review})}%
                \fi%
                \apaSix@masked@refs%
        }

        % \maskparencite
        \newcommand\maskparencite{\@ifnextchar[{\maskparencite@@also}{\maskparencite@@also[]}}
        \newcommand\maskparencite@@also{}
        \def\maskparencite@@also[#1]{\@ifnextchar[{\maskparencite@@@also[#1]}{\maskparencite@@@also[][#1]}}

        \def\maskparencite@@@also%
            [#1][#2]#3{%
                \setcounter{maskedRefs}{0}%
                \SubStringsToCounter{maskedRefs}{,}{#3}%
                \addtocounter{maskedRefs}{1}%
                \ifnum\value{maskedRefs} = 1%
                \def\apaSix@masked@refs{(\textit{\themaskedRefs\ citation removed for masked review})}%
                \else%
                \def\apaSix@masked@refs{(\textit{\themaskedRefs\ citations removed for masked review})}%
                \fi%
                \apaSix@masked@refs%
        }

        % \maskParencite
        \newcommand\maskParencite{\@ifnextchar[{\maskParencite@@also}{\maskParencite@@also[]}}
        \newcommand\maskParencite@@also{}
        \def\maskParencite@@also[#1]{\@ifnextchar[{\maskParencite@@@also[#1]}{\maskParencite@@@also[][#1]}}

        \def\maskParencite@@@also%
            [#1][#2]#3{%
                \setcounter{maskedRefs}{0}%
                \SubStringsToCounter{maskedRefs}{,}{#3}%
                \addtocounter{maskedRefs}{1}%
                \ifnum\value{maskedRefs} = 1%
                \def\apaSix@masked@refs{(\textit{\themaskedRefs\ citation removed for masked review})}%
                \else%
                \def\apaSix@masked@refs{(\textit{\themaskedRefs\ citations removed for masked review})}%
                \fi%
                \apaSix@masked@refs%
        }

        % \maskciteauthor
        \newcommand\maskciteauthor{\@ifnextchar[{\maskciteauthor@@also}{\maskciteauthor@@also[]}}
        \newcommand\maskciteauthor@@also{}
        \def\maskciteauthor@@also[#1]{\@ifnextchar[{\maskciteauthor@@@also[#1]}{\maskciteauthor@@@also[][#1]}}

        \def\maskciteauthor@@@also%
            [#1][#2]#3{%
                \setcounter{maskedRefs}{0}%
                \SubStringsToCounter{maskedRefs}{,}{#3}%
                \addtocounter{maskedRefs}{1}%
                \ifnum\value{maskedRefs} = 1%
                \def\apaSix@masked@refs{(\textit{\themaskedRefs\ citation removed for masked review})}%
                \else%
                \def\apaSix@masked@refs{(\textit{\themaskedRefs\ citations removed for masked review})}%
                \fi%
                \apaSix@masked@refs%
        }

        % \maskCiteauthor
        \newcommand\maskCiteauthor{\@ifnextchar[{\maskCiteauthor@@also}{\maskCiteauthor@@also[]}}
        \newcommand\maskCiteauthor@@also{}
        \def\maskCiteauthor@@also[#1]{\@ifnextchar[{\maskCiteauthor@@@also[#1]}{\maskCiteauthor@@@also[][#1]}}

        \def\maskCiteauthor@@@also%
            [#1][#2]#3{%
                \setcounter{maskedRefs}{0}%
                \SubStringsToCounter{maskedRefs}{,}{#3}%
                \addtocounter{maskedRefs}{1}%
                \ifnum\value{maskedRefs} = 1%
                \def\apaSix@masked@refs{(\textit{\themaskedRefs\ citation removed for masked review})}%
                \else%
                \def\apaSix@masked@refs{(\textit{\themaskedRefs\ citations removed for masked review})}%
                \fi%
                \apaSix@masked@refs%
        }

        % \maskciteyear
        \newcommand\maskciteyear{\@ifnextchar[{\maskciteyear@@also}{\maskciteyear@@also[]}}
        \newcommand\maskciteyear@@also{}
        \def\maskciteyear@@also[#1]{\@ifnextchar[{\maskciteyear@@@also[#1]}{\maskciteyear@@@also[][#1]}}

        \def\maskciteyear@@@also%
            [#1][#2]#3{%
                \setcounter{maskedRefs}{0}%
                \SubStringsToCounter{maskedRefs}{,}{#3}%
                \addtocounter{maskedRefs}{1}%
                \ifnum\value{maskedRefs} = 1%
                \def\apaSix@masked@refs{(\textit{\themaskedRefs\ citation removed for masked review})}%
                \else%
                \def\apaSix@masked@refs{(\textit{\themaskedRefs\ citations removed for masked review})}%
                \fi%
                \apaSix@masked@refs%
        }

        % \maskfootcite
        \newcommand\maskfootcite{\@ifnextchar[{\maskfootcite@@also}{\maskfootcite@@also[]}}
        \newcommand\maskfootcite@@also{}
        \def\maskfootcite@@also[#1]{\@ifnextchar[{\maskfootcite@@@also[#1]}{\maskfootcite@@@also[][#1]}}

        \def\maskfootcite@@@also%
            [#1][#2]#3{%
                \setcounter{maskedRefs}{0}%
                \SubStringsToCounter{maskedRefs}{,}{#3}%
                \addtocounter{maskedRefs}{1}%
                \ifnum\value{maskedRefs} = 1%
                \def\apaSix@masked@refs{(\textit{\themaskedRefs\ citation removed for masked review})}%
                \else%
                \def\apaSix@masked@refs{(\textit{\themaskedRefs\ citations removed for masked review})}%
                \fi%
                \apaSix@masked@refs%
        }

        % \maskfootcitetext
        \newcommand\maskfootcitetext{\@ifnextchar[{\maskfootcitetext@@also}{\maskfootcitetext@@also[]}}
        \newcommand\maskfootcitetext@@also{}
        \def\maskfootcitetext@@also[#1]{\@ifnextchar[{\maskfootcitetext@@@also[#1]}{\maskfootcitetext@@@also[][#1]}}

        \def\maskfootcitetext@@@also%
            [#1][#2]#3{%
                \setcounter{maskedRefs}{0}%
                \SubStringsToCounter{maskedRefs}{,}{#3}%
                \addtocounter{maskedRefs}{1}%
                \ifnum\value{maskedRefs} = 1%
                \def\apaSix@masked@refs{(\textit{\themaskedRefs\ citation removed for masked review})}%
                \else%
                \def\apaSix@masked@refs{(\textit{\themaskedRefs\ citations removed for masked review})}%
                \fi%
                \apaSix@masked@refs%
        }

        % \masktextcite
        \newcommand\masktextcite{\@ifnextchar[{\masktextcite@@also}{\masktextcite@@also[]}}
        \newcommand\masktextcite@@also{}
        \def\masktextcite@@also[#1]{\@ifnextchar[{\masktextcite@@@also[#1]}{\masktextcite@@@also[][#1]}}

        \def\masktextcite@@@also%
            [#1][#2]#3{%
                \setcounter{maskedRefs}{0}%
                \SubStringsToCounter{maskedRefs}{,}{#3}%
                \addtocounter{maskedRefs}{1}%
                \ifnum\value{maskedRefs} = 1%
                \def\apaSix@masked@refs{(\textit{\themaskedRefs\ citation removed for masked review})}%
                \else%
                \def\apaSix@masked@refs{(\textit{\themaskedRefs\ citations removed for masked review})}%
                \fi%
                \apaSix@masked@refs%
        }

        % \maskTextcite
        \newcommand\maskTextcite{\@ifnextchar[{\maskTextcite@@also}{\maskTextcite@@also[]}}
        \newcommand\maskTextcite@@also{}
        \def\maskTextcite@@also[#1]{\@ifnextchar[{\maskTextcite@@@also[#1]}{\maskTextcite@@@also[][#1]}}

        \def\maskTextcite@@@also%
            [#1][#2]#3{%
                \setcounter{maskedRefs}{0}%
                \SubStringsToCounter{maskedRefs}{,}{#3}%
                \addtocounter{maskedRefs}{1}%
                \ifnum\value{maskedRefs} = 1%
                \def\apaSix@masked@refs{(\textit{\themaskedRefs\ citation removed for masked review})}%
                \else%
                \def\apaSix@masked@refs{(\textit{\themaskedRefs\ citations removed for masked review})}%
                \fi%
                \apaSix@masked@refs%
        }

      }

}



%========== APA Section spacing

\newskip\b@level@one@skip   \b@level@one@skip=2.5ex plus 1ex minus .2ex
\newskip\e@level@one@skip   \e@level@one@skip=1.5ex plus .6ex minus .1ex
\newskip\b@level@two@skip   \b@level@two@skip=2.5ex plus 1ex minus .2ex
\newskip\e@level@two@skip   \e@level@two@skip=1.5ex plus .6ex minus .1ex
\newskip\b@level@three@skip \b@level@three@skip=2.0ex plus .8ex minus .2ex
\newskip\e@level@three@skip \e@level@three@skip=1.5ex plus .6ex minus .1ex
\newskip\b@level@four@skip  \b@level@four@skip=1.8ex plus .8ex minus .2ex
\newskip\e@level@four@skip  \e@level@four@skip=1.5ex plus .6ex minus .1ex
\newskip\b@level@five@skip  \b@level@five@skip=1.8ex plus .8ex minus .2ex
\newskip\e@level@five@skip  \e@level@five@skip=0ex

\ifapamodeman{%
  \@ifundefined{def@noextraspace}{}{%
    % redefine the vertical section spacing
    \b@level@one@skip=0.2\baselineskip \@plus 0.2ex \@minus 0.2ex
    \e@level@one@skip=0.2\baselineskip \@plus .2ex
    \b@level@two@skip=0.2\baselineskip \@plus 0.2ex \@minus 0.2ex
    \e@level@two@skip=0.2\baselineskip \@plus 0.2ex
    \b@level@three@skip=0\baselineskip \@plus 0.2ex \@minus 0.2ex
    \e@level@three@skip=-\z@
    \b@level@four@skip=0\baselineskip \@plus 0.2ex \@minus 0.2ex
    \e@level@four@skip=-\z@
    \b@level@five@skip=0\baselineskip \@plus 0.2ex \@minus 0.2ex
    \e@level@five@skip=0ex
  }
}{}



%========== APA Section heading & seriation, adapted from class apa6e

% % Below is the license statement from the apa6e class, from which a
% % small bit of code was used (with permission from Nathaniel Smith)
% % Copyright (C) 2011 by Nathaniel J. Smith <njs@pobox.com>
% %
% % Redistribution and use in source and binary forms, with or without
% % modification, are permitted provided that the following conditions are
% % met:
% %
% %    1. Redistributions of source code must retain the above copyright
% %       notice, this list of conditions and the following disclaimer.
% %
% %    2. Redistributions in binary form must reproduce the above
% %       copyright notice, this list of conditions and the following
% %       disclaimer in the documentation and/or other materials provided
% %       with the distribution.
% %
% % THIS SOFTWARE IS PROVIDED BY THE COPYRIGHT HOLDERS AND CONTRIBUTORS
% % ``AS IS'' AND ANY EXPRESS OR IMPLIED WARRANTIES, INCLUDING, BUT NOT
% % LIMITED TO, THE IMPLIED WARRANTIES OF MERCHANTABILITY AND FITNESS FOR
% % A PARTICULAR PURPOSE ARE DISCLAIMED. IN NO EVENT SHALL THE COPYRIGHT
% % HOLDER OR CONTRIBUTORS BE LIABLE FOR ANY DIRECT, INDIRECT, INCIDENTAL,
% % SPECIAL, EXEMPLARY, OR CONSEQUENTIAL DAMAGES (INCLUDING, BUT NOT
% % LIMITED TO, PROCUREMENT OF SUBSTITUTE GOODS OR SERVICES; LOSS OF USE,
% % DATA, OR PROFITS; OR BUSINESS INTERRUPTION) HOWEVER CAUSED AND ON ANY
% % THEORY OF LIABILITY, WHETHER IN CONTRACT, STRICT LIABILITY, OR TORT
% % (INCLUDING NEGLIGENCE OR OTHERWISE) ARISING IN ANY WAY OUT OF THE USE
% % OF THIS SOFTWARE, EVEN IF ADVISED OF THE POSSIBILITY OF SUCH DAMAGE.
% %

\setcounter{secnumdepth}{0}

\renewcommand{\section}{\@startsection {section}{1}{\z@}%
    {\b@level@one@skip}{\e@level@one@skip}%
    {\centering\normalfont\normalsize\bfseries}}

\renewcommand{\subsection}{\@startsection{subsection}{2}{\z@}%
    {\b@level@two@skip}{\e@level@two@skip}%
    {\normalfont\normalsize\bfseries}}


\newcommand*{\typesectitle}[1]{#1\addperi}

\newcommand*{\addperi}{%
  \relax\ifhmode%
    \ifnum\spacefactor>\@m \else.\fi%
  \fi%
}

\renewcommand{\subsubsection}{\@startsection{subsubsection}{3}{\parindent}%
    {0\baselineskip \@plus 0.2ex \@minus 0.2ex}%
    {-1em}%
    {\normalfont\normalsize\bfseries\typesectitle}}

\renewcommand{\paragraph}{\@startsection{paragraph}{4}{\parindent}%
    {0\baselineskip \@plus 0.2ex \@minus 0.2ex}%
    {-1em}%
    {\normalfont\normalsize\bfseries\itshape\typesectitle}}

\renewcommand{\subparagraph}[1]{\@startsection{subparagraph}{5}{1em}%
    {0\baselineskip \@plus 0.2ex \@minus 0.2ex}%
    {-\z@\relax}%
    {\normalfont\normalsize\itshape\hspace{\parindent}{#1}\textit{\addperi}}{\relax}}

% make the References section non-boldface
\AtBeginDocument{\def\st@rtbibsection{\mspart{\refname}}}%  BDB -- this is for apacite
\AtBeginDocument{\def\bibsection{\mspart{\refname}}}%  BDB -- this is for apacite + natbib
\newcommand{\mspart}{{\ifapamodeman{\clearpage}{}}\@startsection {section}{1}{\z@}%
    {\b@level@one@skip}{\e@level@one@skip}%
    {\centering\normalfont\normalsize}}

% format the table and figure captions
\RequirePackage[singlelinecheck=off]{caption}

\ifapamode{% man
    \DeclareCaptionLabelFormat{tablelabel}{\hspace{-\parindent}\raggedright#1 #2}
    \DeclareCaptionLabelFormat{figurelabel}{\hspace{-\parindent}\raggedright\textit{#1 #2}}
    \DeclareCaptionTextFormat{tabletext}{\hspace{-\parindent}\raggedright\textit{#1}}
}{% jou
    \DeclareCaptionLabelFormat{tablelabel}{\hspace{-\parindent}#1 #2}
    \DeclareCaptionLabelFormat{figurelabel}{\hspace{-\parindent}\textit{#1 #2}}
    \DeclareCaptionTextFormat{tabletext}{\hspace{-\parindent
}\textit{#1}}
}{% doc
    \DeclareCaptionLabelFormat{tablelabel}{\hspace{-\parindent}#1 #2}
    \DeclareCaptionLabelFormat{figurelabel}{\hspace{-\parindent}\textit{#1 #2}}
    \DeclareCaptionTextFormat{tabletext}{\hspace{-\parindent}\textit{#1}}
}
\captionsetup[table]{position=above,skip=0pt,labelformat=tablelabel,labelsep=newline,textformat=tabletext}
\captionsetup[figure]{position=below,skip=0pt,labelformat=figurelabel,labelsep=period}

% space between caption and surrounding document text
\setlength{\belowcaptionskip}{0pt}


%=============================================================
%
% End of code from apa6e
%
%=============================================================


% Code from theapa.sty

% Refer to LaTeX book to modify, if you want, spaces before and after of
%  \begin{...} ... \end{...} or spaces between \item-s.
\newcounter{APAenum}
\newskip\@text@par@indent
\def\APAenumerate{\@text@par@indent\parindent\setbox0\hbox{1. }%
    \list{\arabic{APAenum}.}{\usecounter{APAenum}
    \labelwidth\z@\labelsep\z@\leftmargin\z@\parsep\z@
    \rightmargin\z@\itemsep\z@\topsep\z@\partopsep\z@
    \itemindent\@text@par@indent\advance\itemindent by\wd0
    \def\makelabel##1{\hss\llap{##1 }}}}
\let\endAPAenumerate=\endlist

\def\seriate{\@bsphack\begingroup%
   \setcounter{APAenum}{0}%
   \def\item{\addtocounter{APAenum}{1}(\alph{APAenum})\space}%
   \ignorespaces}
\def\endseriate{\endgroup\@esphack}

\def\APAitemize{\@text@par@indent\parindent\setbox0\hbox{$\bullet$}%
    \list{$\bullet$}{%
    \labelwidth\z@\labelsep.5em\leftmargin\z@\parsep\z@
    \rightmargin\z@\itemsep\z@\topsep\z@\partopsep\z@
    \itemindent\@text@par@indent
    \advance\itemindent by\wd0\advance\itemindent by.5em
    \def\makelabel##1{\hss\llap{##1}}}}
\let\endAPAitemize=\endlist

% \let\enumerate=\APAenumerate
% \let\endenumerate=\endAPAenumerate
% \let\itemize=\APAitemize
% \let\enditemize=\endAPAitemize

%=============================================================
%
% End of code from theapa.sty
%
%=============================================================

%===== apa.cls main declarations for title page contents =====

\long\def\title#1{\long\def\@title{#1}}
\long\def\author#1{\long\def\@author{#1}}
\long\def\shorttitle#1{\long\def\@shorttitle{#1}}
\long\def\twoauthors#1#2{\long\def\@authorOne{#1}\long\def\@authorTwo{#2}%
 \long\def\@author{#1}}
\long\def\onetwoauthors#1#2#3{\long\def\@authorOne{#1}\long\def\@authorTwo{#2}%
 \long\def\@authorThree{#3}\long\def\@author{#1}}
\long\def\twooneauthors#1#2#3{\long\def\@authorOne{#1}\long\def\@authorTwo{#2}%
 \long\def\@authorThree{#3}\long\def\@author{#1}\def\@twofirst{1}}
\let\threeauthors=\onetwoauthors
\long\def\fourauthors#1#2#3#4{\long\def\@authorOne{#1}\long\def\@authorTwo{#2}%
 \long\def\@authorThree{#3}\long\def\@authorFour{#4}\long\def\@author{#1}}
\long\def\fiveauthors#1#2#3#4#5{\long\def\@authorOne{#1}\long\def\@authorTwo{#2}%%%%
 \long\def\@authorThree{#3}\long\def\@authorFour{#4}\long\def\@authorFive{#5}%    %%
 \long\def\@author{#1}} %%     2006/01/05 -- added as contributed by Aaron Geller %%
\long\def\sixauthors#1#2#3#4#5#6{\long\def\@authorOne{#1}%                  %% thp 2006/01/05
 \long\def\@authorTwo{#2}\long\def\@authorThree{#3}\long\def\@authorFour{#4}%% thp 2006/01/05
 \long\def\@authorFive{#5}\long\def\@authorSix{#6}\long\def\@author{#1}}    %% thp 2006/01/05
\long\def\affiliation#1{\long\def\@affil{#1}}
\long\def\twoaffiliations#1#2{\long\def\@affilOne{#1}\long\def\@affilTwo{#2}%
\long\def\@affil{#1}}
\long\def\onetwoaffiliations#1#2#3{\long\def\@affilOne{#1}\long\def\@affilTwo{#2}%
 \long\def\@affilThree{#3}\long\def\@affil{#1}}
\long\def\twooneaffiliations#1#2#3{\long\def\@affilOne{#1}\long\def\@affilTwo{#2}%
 \long\def\@affilThree{#3}\long\def\@affil{#1}}
\let\threeaffiliations=\onetwoaffiliations
\long\def\fouraffiliations#1#2#3#4{\long\def\@affilOne{#1}\long\def\@affilTwo{#2}%
 \long\def\@affilThree{#3}\long\def\@affilFour{#4}\long\def\@affil{#1}}
\long\def\fiveaffiliations#1#2#3#4#5{\long\def\@affilOne{#1}\long\def\@affilTwo{#2}%%
 \long\def\@affilThree{#3}\long\def\@affilFour{#4}\long\def\@affilFive{#5}%        %%
 \long\def\@affil{#1}} %%     2006/01/05 -- added as contributed by Aaron Geller   %%
\long\def\sixaffiliations#1#2#3#4#5#6{\long\def\@affilOne{#1}%           %% thp 2006/01/05
 \long\def\@affilTwo{#2}\long\def\@affilThree{#3}\long\def\@affilFour{#4}%% thp 2006/01/05
 \long\def\@affilFive{#5}\long\def\@affilSix{#6}\long\def\@affil{#1}}    %% thp 2006/01/05
\long\def\note#1{\long\def\@note{#1}}
\long\def\abstract#1{\long\def\@abstract{#1}}
\long\def\keywords#1{\long\def\@keywords{#1}}
%\long\def\acknowledgements#1{\long\def\@acks{#1}}
\long\def\authornote#1{\long\def\@acks{#1}}
\def\journal#1{\RequirePackage{fancyhdr}\def\@journal{#1}}
\def\volume#1{\def\@vvolume{#1}}
\def\ccoppy#1{\def\@ccoppy{#1}}
\def\copnum#1{\def\@copnum{#1}}
\def\@error@toomanyauthors{\ClassWarningNoLine{apa6}{More authors than affiliations defined}}
\def\@error@toomanyaffils{\ClassWarningNoLine{apa6}{More affiliations than authors defined}}
% \RequirePackage{xstring}  %% need this if inserting default shorttitle (three lines commented out below)
\def\check@author{%
 \@ifundefined{@author}{%
  \ClassWarningNoLine{apa6}{Author not defined}\def\@author{Author}}{}
 \@ifundefined{@title}{%
  \ClassWarningNoLine{apa6}{Title not defined}\def\@title{Title}}{}
 \@ifundefined{@shorttitle}{%
   \ClassWarningNoLine{apa6}{Short title not defined}\def\@shorttitle{INSERT SHORTTITLE COMMAND IN PREAMBLE}}{}
%    \ClassWarningNoLine{apa6}{Short title not defined}%
%      \StrLen{\@title}[\titleLen]\def\@shorttitle{%
%      \ifnum\titleLen 50INSERT SHORTTITLE COMMAND IN PREAMBLE\else\@title\fi}}{}
 \@ifundefined{@affil}{%
  \ClassWarningNoLine{apa6}{Affiliation not defined}\def\@affil{Affiliation}}{}
 \@ifundefined{@abstract}{%
  \ClassWarningNoLine{apa6}{Abstract not defined}}{}
 \@ifundefined{@keywords}{%
  \ClassInfo{apa6}{Keywords not defined}}{}
 \@ifundefined{@authorSix}{%                                   % -- thp 2006/01/05
  \@ifundefined{@authorFive}{%                                 % -- thp 2006/01/05
   \@ifundefined{@authorFour}{%
    \@ifundefined{@authorThree}{%
     \@ifundefined{@authorTwo}{%
     }{\@ifundefined{@affilTwo}{\@error@toomanyauthors}{}}
    }{\@ifundefined{@affilThree}{\@error@toomanyauthors}{}}
   }{\@ifundefined{@affilFour}{\@error@toomanyauthors}{}}
  }{\@ifundefined{@affilFive}{\@error@toomanyauthors}{}}       % -- thp 2006/01/05
 }{\@ifundefined{@affilSix}{\@error@toomanyauthors}{}}         % -- thp 2006/01/05
 \@ifundefined{@affilSix}{%                                    % -- thp 2006/01/05
  \@ifundefined{@affilFive}{%                                  % -- thp 2006/01/05
   \@ifundefined{@affilFour}{%
    \@ifundefined{@affilThree}{%
     \@ifundefined{@affilTwo}{%
     }{\@ifundefined{@authorTwo}{\@error@toomanyaffils}{}}
    }{\@ifundefined{@authorThree}{\@error@toomanyaffils}{}}
   }{\@ifundefined{@authorFour}{\@error@toomanyaffils}{}}
  }{\@ifundefined{@authorFive}{\@error@toomanyaffils}{}}       % -- thp 2006/01/05
 }{\@ifundefined{@authorSix}{\@error@toomanyaffils}{}}         % -- thp 2006/01/05
}

%==== Automatic figure size and orientation determination ====

\newsavebox\gr@box
\newlength\gr@boxwidth
\newlength\gr@boxheight

\newcommand{\fitfigure}[2][0.5]{%
%\ifapamodeman
%{% man
%\sbox\gr@box{\includegraphics[width=\linewidth]{#2}}
%\settowidth{\gr@boxwidth}{\usebox\gr@box}
%\settoheight{\gr@boxheight}{\usebox\gr@box}
%\ifdim\gr@boxheight>\gr@boxwidth{% display upright (h)
% \ifdim\gr@boxheight>\textheight%
%  \centerline{\vspace{\fill}\includegraphics[height=\textheight]{#2}\vspace{\fill}}%
% \else%
%  \usebox\gr@box%
% \fi}
%\else{% display rotated
% \sbox\gr@box{\includegraphics[angle=90,width=\linewidth]{#2}}
% \settoheight{\gr@boxheight}{\usebox\gr@box}
% \ifdim\gr@boxheight>\textheight%
%  \centerline{\vspace{\fill}\includegraphics[angle=90,height=\textheight]{#2}\vspace{\fill}}%
% \else%
%  \usebox\gr@box%
% \fi}%
%\fi
%}{% jou, doc
\sbox\gr@box{\includegraphics[width=\linewidth]{#2}}
\settoheight{\gr@boxheight}{\usebox\gr@box}
\ifdim\gr@boxheight>\textheight%
 \centerline{\includegraphics[height=#1\textheight]{#2}}% need to leave space for caption
\else%
 \usebox\gr@box%
\fi
%}
}

\newcommand{\fitbitmap}[2][0.5]{ % like fitfigure but no scaling in man mode for best quality
%\ifapamodeman
%{% man
%\sbox\gr@box{\includegraphics{#2}}
%\settowidth{\gr@boxwidth}{\usebox\gr@box}
%\settoheight{\gr@boxheight}{\usebox\gr@box}
%\ifdim\gr@boxwidth>\textwidth% display rotated
%  \centerline{\vspace{\fill}\includegraphics[angle=90]{#2}\vspace{\fill}}%
%  \ifdim\gr@boxheight>\textwidth\ClassWarning{apa6}{Figure #2 too high!}\fi% display rotated
%  \ifdim\gr@boxwidth>\textheight\ClassWarning{apa6}{Figure #2 too wide!}\fi% display rotated
%\else% display upright
%  \centerline{\vspace{\fill}\includegraphics{#2}\vspace{\fill}}%
%  \ifdim\gr@boxheight>\textheight\ClassWarning{apa6}{Figure #2 too high!}\fi% display rotated
%\fi}
%{% jou, doc
\sbox\gr@box{\includegraphics[width=\linewidth]{#2}}
\settoheight{\gr@boxheight}{\usebox\gr@box}
\ifdim\gr@boxheight>\textheight%
 \centerline{\includegraphics[height=#1\textheight]{#2}}% need to leave space for caption
\else%
 \centerline{\usebox\gr@box}%
\fi
%}
}

%==== Miscellaneous definitions

\let\normal@BBAB\BBAB                        % -- thp 2005/07/23
\let\table@BBAB\BBAA                         % -- thp 2005/07/23

\setlength{\doublerulesep}{\arrayrulewidth}
\newcommand\thickline{\hline\hline}
\renewcommand\footnoterule{%
  \kern-3\p@
%  \hrule.5\@width.4\columnwidth
  \hrule height0.125pt width.5in
  \kern2.6\p@}

\let\apaSixtabular\tabular
\let\apaSix@doc@tabular\tabular
%\let\old@tabular\tabular
%\def\apaSixtabular#1{\def\@halignto{to\linewidth}%
% \let\BBAB\table@BBAB%                         -- thp 2005/07/23
% \@tabular{@{\extracolsep{\fill}}#1}}
%\def\apaSix@doc@tabular{%% thp 2006/01/02
% \let\BBAB\table@BBAB%% Erik Meijer noticed that in doc mode tables weren't centered
% \hfill\old@tabular} %% and that the in-table citation form was incorrect
%%
\let\apaSix@doc@endtabular\endtabular
%\let\orig@endtabular\endtabular              % -- thp 2005/07/23
%\def\endtabular{%
% \let\BBAB\normal@BBAB%                        -- thp 2005/07/23
% \orig@endtabular\ifapamodedoc{\hfill}{}}% 2006/01/02 \hfill for centering in doc mode
%\def\apaSix@doc@endtabular{%% thp 2006/01/03
% \let\BBAB\normal@BBAB% %% separate redefinition of tabular/endtabular for doc mode
% \orig@endtabular\hfill}%% to ensure centering, especially in case array.sty is loaded

% Define footnote mark commands for tables that won't mess up vertical alignment
\def\@tab@fn#1{\ensuremath{^{\mbox{{\scriptsize #1}}}}}
\def\tabfnm#1{\rlap{\@tab@fn{#1}}}
%\def\tabfnt#1#2{\par\raggedright\@tab@fn{#1}#2}%BDB
\def\tabfnt#1#2{\raggedright\@tab@fn{#1}#2}

% Provide a command to display vector symbols appropriately in each mode, thp 10/00
% Use improved definition of apaSixvector provided by Erik Meijer -- 2005/12/28
\def\apaSixvector#1{{\ensuremath
  \uprightlowercasegreek
  \ifapamodeman
  {\apaSixsmash{\mathop{\kern\z@\mathrm{#1}}\limits_{\scriptscriptstyle\sim}}}%
  {\if@bm@loaded\bm{\mathrm{#1}}\else\mathbf{#1}\fi% in case bm is not available
  }%
}}
% and here are the necessary definitions & packages, as instructed by Erik Meijer
\newcommand{\apaSixsmash}{%
  \def\finsm@sh{\dp\z@\z@ \box\z@}%
  \expandafter\mathpalette\expandafter\mathsm@sh
}%
\newif\if@bm@loaded\@bm@loadedfalse
\IfFileExists{bm.sty}{\RequirePackage{bm}\@bm@loadedtrue}{}% if not, apaSixvector will fail
\newcommand{\uprightlowercasegreek}{%
  \@ifundefined{alphaup}{}{%
    \def\alpha     {\alphaup     }%
    \def\beta      {\betaup      }%
    \def\gamma     {\gammaup     }%
    \def\delta     {\deltaup     }%
    \def\epsilon   {\epsilonup   }%
    \def\varepsilon{\varepsilonup}%
    \def\zeta      {\zetaup      }%
    \def\eta       {\etaup       }%
    \def\theta     {\thetaup     }%
    \def\vartheta  {\varthetaup  }%
    \def\iota      {\iotaup      }%
    \def\kappa     {\kappaup     }%
    \def\lambda    {\lambdaup    }%
    \def\mu        {\muup        }%
    \def\nu        {\nuup        }%
    \def\xi        {\xiup        }%
    \def\pi        {\piup        }%
    \def\varpi     {\varpiup     }%
    \def\rho       {\rhoup       }%
    \def\varrho    {\varrhoup    }%
    \def\sigma     {\sigmaup     }%
    \def\varsigma  {\varsigmaup  }%
    \def\tau       {\tauup       }%
    \def\upsilon   {\upsilonup   }%
    \def\phi       {\phiup       }%
    \def\varphi    {\varphiup    }%
    \def\chi       {\chiup       }%
    \def\psi       {\psiup       }%
    \def\omega     {\omegaup     }%
  }%
}
\let\apaSixmatrix\apaSixvector

%==== Appendix macros added by Michael Erickson 2000/07/11, revised by thp 2000/07/18

\newcounter{appendix}\setcounter{appendix}{0}
\renewcommand{\theappendix}{\@Alph\c@appendix}
\def\apaSixappfig{%
 \renewcommand\thefigure{\theappendix\@arabic\c@figure}%
 \ifapamodeman{\renewcommand\thepostfig{\theappendix\arabic{postfig}}}{}}
\def\apaSixapptab{%
 \renewcommand\thetable{\theappendix\@arabic\c@table}%
 \ifapamodeman{\renewcommand\theposttbl{\theappendix\arabic{posttbl}}}{}}
\newif\ifoneappendix
\oneappendixtrue % one appendix by default
\newif\ifappendix
\appendixfalse

\def\appendix{%
%
  \ifapamodeman{\processdelayedfloats}{}%  BDB -- output all tables and figures prior to the appendix
%
  \appendixtrue
  \apaSixappfig
  \apaSixapptab
%  \ifapamodejou{}{\clearpage} %% commented out -- thp 2005/07/23
  \let\old@apaSix@section=\section % This will not work right with five levels in appendix.
                                 % Should go into \section, not \leveltwo but would also require
                                 % changes to section* and section[ (see \def\section above)
                                 % Who uses five level heading appendices anyway?
%  \newlength{\app@t@width}
%  \setlength{\app@t@width}{\columnwidth}
%  \addtolength{\app@t@width}{-8em}%% leveltwo is defined with leftskip=rightskip=4em plus 1fill
  \long\def\section##1{%
                   \makeatletter%
                     \def\@currentlabelname{##1}%
                   \makeatother%
                   \ifapamodeman{%
                    \clearpage
                    \setcounter{postfig}{0}
                    \setcounter{posttbl}{0}
%                    \efloat@condopen{fff}
%                    \efloat@iwrite{fff}{\string\addtocounter{appendix}{1}}
%                    \efloat@iwrite{fff}{\string\setcounter{figure}{0}}
%                    \efloat@condopen{ttt}
%                    \efloat@iwrite{ttt}{\string\addtocounter{appendix}{1}}
%                    \efloat@iwrite{ttt}{\string\setcounter{table}{0}}%
                   }{%
%                    \setcounter{figure}{0}%
%                    \setcounter{table}{0}%
%                    \vskip2.5ex%
%                    %\small% the appendix should be set in smaller type in jou, including headers!
                   }%
                    \setcounter{figure}{0}%
                    \setcounter{table}{0}%
                    \vskip2.5ex%
%                  \addtocounter{appendix}{1}%
                   \refstepcounter{appendix}% 2002/07/20 this takes care of references too
                   \ifnum\c@appendix>1\immediate\write\@auxout{\global\string\oneappendixfalse}\fi%
%                   \old@apaSix@section{%
%                      \old@tabular[t]{@{}p{\app@t@width}@{}}% tabular may have been redefined
%                         \centering{%
%                          \appendixname\ifoneappendix\else~\theappendix\fi\\
%                          ##1%
%                      }\endtabular
%                   }%
%
                      \centerline{\normalfont\normalsize\appendixname\ifoneappendix\else~\theappendix\fi}%
                      \centerline{##1}\par%
                      \setlength{\parindent}{0.4in}
                      \makeatletter%
                        \@afterindentfalse%
                        \@afterheading%
                      \makeatother%
%
                  }%
%
} % end of appendix definition

%==== End of general apa.cls macro definitions ====


%%%%%%%%%%%%%%%%%%%%%%%%%
%%                     %%
%%  MANUSCRIPT FORMAT  %%
%%                     %%
%%%%%%%%%%%%%%%%%%%%%%%%%

\@ifundefined{def@man}{}{%

% \def\@@spacing{1.75}
\def\@@spacing{1.655}

\newcommand{\@doublespacing}{\linespread{1.655}}
\@doublespacing

% allow a little more space between captions and floats in man mode
\captionsetup[table]{skip=10pt}
\captionsetup[figure]{skip=10pt}

\def\rightheader#1{\def\r@headr{\protect\MakeUppercase{#1}}}
\def\leftheader#1{\def\r@headl{#1}}
%\def\shorttitle#1{\def\s@title{#1}}%\markright{\hfill #1 \thepage} %% no rm - thp 020227

%%%%%%%%%%%%%%%%%%%%%%%%%%%%%%%\newcommand{\shorttitle}[1]{\def\@shorttitle{#1}}

\raggedright

\RequirePackage{fancyhdr}
\setlength{\headheight}{15.2pt}
\fancyhf{}
\renewcommand{\headrulewidth}{0pt}

\fancypagestyle{titlepage}{%
    \lhead{\rheadname: \MakeUppercase{\@shorttitle}}%
    \rhead{\thepage}%
}
\fancypagestyle{otherpage}{%
    \lhead{\MakeUppercase{\@shorttitle}}%
    \rhead{\thepage}%
}
\pagestyle{otherpage}

%========== Alterations to endnotes ===================

%\RequirePackage{endnotes}
%
%% added by BDB to make section heading non-bolface
%\def\enoteheading{\section*{\normalfont\normalsize\notesname
%  \@mkboth{\MakeUppercase{\notesname}}{\MakeUppercase{\notesname}}}%
%  \mbox{}\par\vskip-\baselineskip}
%
%
%\def\@makeenmark{\hbox{$^{\mbox{\small\@ifundefined{hrm}{}{\sf}\@theenmark}}$}}
%\def\enoteformat{\rightskip\z@ \leftskip\z@
%     \raggedright\fussy\hyphenpenalty 10000 \parindent=0.4in
%     \leavevmode\llap{\hbox{$^{\mbox{\small\@ifundefined{hrm}{}{\sf}\@theenmark}}$}}}
%\AtEndDocument{%
% \ifappendix% set section back to no numbering after appendix
%  \appendixfalse\let\section=\old@apaSix@section%
% \fi
%% \@ifundefined{@acks}{}{\newpage\section{\acksname}\@acks}
% \@ifundefined{hrm}{\def\enotesize{\normalsize}}
%                   {\def\enotesize{\normalsize\@ifundefined{hrm}{}{\sf}}}
% \@ifundefined{NoEndnotes}{\if@enotesopen\newpage\theendnotes\fi}{}
% \let\footnote=\oldfootnote
%}

%========= Alterations to endfloat ====================

% prepare to bypass endfloat for any tables or figures in appendices
\let\@noendfloattab\table% BDB
\let\@noendfloatfig\figure% BDB

% \@ifundefined{float@mark}{% Figure and table in-text markers are no longer required
%  \RequirePackage[notablist,figlist,notabhead,nofighead,tablesfirst,nomarkers]{endfloat}[1995/10/11]}{%
%  \RequirePackage[notablist,figlist,notabhead,nofighead,tablesfirst]{endfloat}[1995/10/11]}

\RequirePackage[notablist,figlist,notabhead,nofighead,tablesfirst,nomarkers]{endfloat}[1995/10/11]

%\renewcommand{\figureplace}{%
%   \par\begin{center}
%   \parbox{\textwidth}
%   {\begin{center}
%   \rule{2.1in}{0.005in}\\
%   Insert \figurename~\thepostfig\ about here\\
%   \rule{2.1in}{0.005in}\\
%   \end{center}}
%   \end{center}\par}
%\renewcommand{\tableplace}{%
%   \par\begin{center}
%   \parbox{\textwidth}
%   {\begin{center}
%   \rule{2.1in}{0.005in}\\
%   Insert \tablename~\theposttbl\ about here\\
%   \rule{2.1in}{0.005in}\\
%   \end{center}}
%   \end{center}\par}

\def\@gobbleuntilnext[#1]{}
\let\eatarg\@gobbleuntilnext
\let\ifnextchar\@ifnextchar

\@ifundefined{def@floatsintext}{%  Guillaume Jourjon 19/10/10
  \def\figure{%
      \ifappendix
          \vspace*{\intextsep}
          \def\fps@figure{!hbt}%
          \@noendfloatfig
      \else
           \efloat@condopen{fff}
           \efloat@iwrite{fff}{\string\begin{figure*}[hbt]}%
           \global\def\@figure@written{\relax}% Set a flag that there is at least one figure -- thp 20010705
      %% should be a global declaration to be "visible" at end document
           \ifnextchar[{\gobbleuntilnext[}{}
           \efloat@iwrite{fff}{\string\ifnextchar[{\string\eatarg}{}}
%          \if@domarkers%
%             \addtocounter{postfig}{1}% % bj
%             \figureplace%              % bj
%          \fi%
          \def\@currenvir{efloat@float}%
          \begingroup%
          \let\do\ef@makeinnocent \dospecials%
          \ef@makeinnocent\^^L% and whatever other special cases
          \endlinechar`\^^M \catcode`\^^M=12 \ef@xfigure%
      \fi%
  }%

  \def\table{%
      \ifappendix
          \vspace*{\intextsep}
          \def\fps@table{!hbt}
          \@noendfloattab
      \else
          \efloat@condopen{ttt}
          \efloat@iwrite{ttt}{\string\begin{table*}[hbt]}%
          \ifnextchar[{\gobbleuntilnext[}{}
          \@ifundefined{hrm}{}{%
          \efloat@iwrite{ttt}{\string\sf}}%
          \efloat@iwrite{ttt}{\string\ifnextchar[{\string\eatarg}{}} % bj
%          \if@domarkers
%             \addtocounter{posttbl}{1} % bj
%             \tableplace               % bj
%          \fi
          \def\@currenvir{efloat@float}%
          \begingroup
          \let\do\ef@makeinnocent \dospecials
          \ef@makeinnocent\^^L% and whatever other special cases
          \endlinechar`\^^M \catcode`\^^M=12 \ef@xtable%
      \fi
  }%

% for rotated tables, per Ulrike Fischer:  http://tex.stackexchange.com/a/79843
  \RequirePackage{etoolbox}
  \AtEndPreamble{%
    \@ifpackageloaded{rotating}{%
      \DeclareDelayedFloatFlavor{sidewaystable}{table}
      \DeclareDelayedFloatFlavor{sidewaysfigure}{figure}
      }{}%
  }%

}{%
  \RequirePackage{float}
  \floatplacement{figure}{htb}
  \def\figure{\@float{figure}}
  \def\endfigure{\end@float}

  \floatplacement{table}{htb}
  \def\table{\@float{table}}
  \def\endtable{\end@float}
}

%%%%%%%%%%%%%%\long\def\@@contentsline#1#2#3{ #2 }
%%%%%%%%%%%%%%\long\def\numberline#1#2{\noindent{\em\figurename\ #1.\/} #2\vspace{0.5\baselineskip}\par}
%%%%%%%%%%%%%%\long\def\@@caption{\refstepcounter\@captype \@dblarg{\@@@caption\@captype}}
%%%%%%%%%%%%%%\long\def\@@@caption#1[#2]#3{\addcontentsline{\csname
%%%%%%%%%%%%%%  ext@#1\endcsname}{#1}{\protect\numberline{\csname
%%%%%%%%%%%%%%  the#1\endcsname}{\ignorespaces #2}}}
%\def\listoffigures{\section*{\figurecaptionsname}\@starttoc{lof}}% changed from 'Figure Captions'

%\long\def\figurecaptionsformat{
%  \let\contentsline=\@@contentsline
%  \let\numberline=\@@numberline
%  \def\baselinestretch{\@@spacing}
%}

\def\processfigures{%
 \expandafter\ifnum \csname @ef@fffopen\endcsname>0
  \immediate\closeout\efloat@postfff \ef@setct{fff}{0}
  \clearpage
  \if@figlist
   \@ifundefined{@figure@written}{}{%
%    \figurecaptionsformat
    {\normalsize\@ifundefined{hrm}{}{\sf}%
%     \let\BBAB\table@BBAB% -- thp 2005/07/23
%     \listoffigures%
%     \let\BBAB\normal@BBAB% -- thp 2005/07/23
    }
%    \clearpage
   }
  \fi
%  \if@fighead
%     \section*{\figuresection}
%     \suppressfloats[t]
%  \fi
  \@ifundefined{fig@num}{%
%  \markboth{}{}
   \pagestyle{otherpage}
%  \pagestyle{empty}%
  }{%
  \setcounter{page}{1}
%  \markright{\rm\hfill \s@title, Figure}
   \def\@oddhead{\rightmark}                                    % changed by Michael Erickson
   \markright{\hfill \s@title, \figurename\ \protect\thefigure} % to include appendix numbering
                                                                % remove rm - thp 020227
  }
%  \let\caption=\@@caption
  \def\baselinestretch{\@@spacing}\normalsize\@ifundefined{hrm}{}{\sf}
  \processfigures@hook \@input{\jobname.fff}
 \fi}

\def\processtables{%
  \expandafter\ifnum \csname @ef@tttopen\endcsname>0
  \immediate\closeout\efloat@postttt \ef@setct{ttt}{0}
  \clearpage
  \if@tabhead
      \section*{\tablesection}
      \suppressfloats[t]
  \fi
  \def\baselinestretch{\@@spacing}
  \processtables@hook \@ifundefined{hrm}{}{\sf}%
  \tiny\normalsize%
% We need to deal with citation alterations here because otherwise we either
% lose the definition through the .ttt diversion or we miss the table*
  \let\BBAB\table@BBAB%  -- thp 2005/07/23
  \@input{\jobname.ttt}%
  \let\BBAB\normal@BBAB% -- thp 2005/07/23
 \fi}



%========== Alterations to captions (BDB) ===================

\captionsetup{justification=raggedright}


%========= Maketitle, margins, lengths, etc. ==========

\def\maketitle{
\@ifundefined{hrm}{}{\hrm}
 \check@author
% \@ifundefined{s@title}{\ClassInfo{apa6}{Using title for short title}
%                        \def\s@title{\@title}
%                        \markright{\hfill\@title}}%\protect\thepage}}  %% removed rm
%                       {\markright{\hfill\s@title}}%\protect\thepage}} %% thp 020227
% \@ifundefined{r@headr}{\ClassInfo{apa6}{Using short title for running head}
%                        \def\r@headr{\protect\MakeUppercase{\s@title}}}{}


  \begin{center}
  \vspace*{0.5in}
%  \makebox[\linewidth][l]{\rheadname\hspace{0.1in}\MakeUppercase{\r@headr}}\\ %thp090298

  \vspace*{1in}
  \@title%
  \ifapamodeman{%
    \@ifundefined{def@noextraspace}{%
      \vspace{0.25in}\\
    }{}
  }{%
    \vspace{0.25in}\\
  }

  \@ifundefined{apaSix@maskauthoridentity}{%  BDB

      \@ifundefined{@authorTwo}{
      \@author \\

      \@affil \vspace{0.25in} \\ }{
      \@ifundefined{@authorThree}{
      \@authorOne \\

      \@affilOne \vspace{0.2in} \\
      \@authorTwo \\

      \@affilTwo \vspace{0.25in} \\ }{
      \@ifundefined{@authorFour}{
      \@authorOne \\

      \@affilOne \vspace{0.2in} \\
      \@authorTwo \\

      \@affilTwo \vspace{0.2in} \\
      \@authorThree \\

      \@affilThree \vspace{0.25in} \\ }{
      \@ifundefined{@authorFive}{       %% 2006/01/05 added as contributed by Aaron Geller
      \@authorOne \\

      \@affilOne \vspace{0.2in} \\
      \@authorTwo \\

      \@affilTwo \vspace{0.2in} \\
      \@authorThree \\

      \@affilThree \vspace{0.2in} \\
      \@authorFour \\

      \@affilFour \vspace{0.25in} \\ }{ %% 2006/01/05 beginning of Aaron Geller contribution
      \@ifundefined{@authorSix}{ %% -- thp 2006/01/05
      \@authorOne \\

      \@affilOne \vspace{0.2in} \\
      \@authorTwo \\

      \@affilTwo \vspace{0.2in} \\
      \@authorThree \\

      \@affilThree \vspace{0.2in} \\
      \@authorFour \\

      \@affilFour \vspace{0.2in} \\ %% thp corrected distance to non-final value of 0.2in
      \@authorFive \\

      \@affilFive \vspace{0.25in} \\ }{%% 2006/01/05 end of Aaaron Geller contribution
    %% --- thp 2006/01/05 beginning of six-author display
      \@authorOne \\

      \@affilOne \vspace{0.2in} \\
      \@authorTwo \\

      \@affilTwo \vspace{0.2in} \\
      \@authorThree \\

      \@affilThree \vspace{0.2in} \\
      \@authorFour \\

      \@affilFour \vspace{0.2in} \\
      \@authorFive \\

      \@affilFive \vspace{0.2in} \\
      \@authorSix \\

      \@affilSix \vspace{0.25in} \\ }}}}}
    %% --- thp 2006/01/05 end of six-author display
      \@ifundefined{@note}
       {\vspace*{\baselineskip} }
       {\@note}

  }{%  mask author identity -- show nothing in the author or author note space
  }

  \end{center}

  \@ifundefined{apaSix@maskauthoridentity}{
      \@ifundefined{@acks}
       {}
       {%
         \vfill%
         \begin{center}%
            \acksname%
         \end{center}%
         \protect\raggedright
         \setlength{\parindent}{0.4in}
         \indent\par\@acks%
       }
  }{%  mask author identity -- show nothing in the author or author note space
  }

  \newpage
  %BDB\hyphenpenalty 10000
  \fussy
  \@ifundefined{@abstract}{}{%
    \section{\normalfont\normalsize\abstractname}% BDB
    \noindent\@abstract\par% BDB
    \@ifundefined{@keywords}{}{%
      \setlength{\parindent}{0.4in}% BDB
      \indent\textit{\keywordname:} \@keywords%
    }%
    \newpage
  }

  \@ifundefined{def@donotrepeattitle}{
    \section{\protect\normalfont{\@title}}
  }{}%
  \raggedright%
  \setlength{\parindent}{0.4in}%
%  \indent%
}

\thispagestyle{titlepage}

%\ThreeLevelHeading
% \def\baselinestretch{\@@spacing}

% footnote issues: separate multiple footnotes on the same page:
\setlength{\footnotesep}{16pt}

% make footnotes ragged-right (credit to Mako Hill):
\renewcommand\@makefntext[1]{\raggedright\textsuperscript{\@thefnmark}~#1}

%\setlength{\topmargin}{0in}
%\setlength{\oddsidemargin}{0.25in}
%\setlength{\evensidemargin}{0.25in}
%\setlength{\textwidth}{6in}
%\setlength{\textheight}{8.5in}
%\setcounter{secnumdepth}{0}
%
%\def\ps@myheadings{
% \let\@mkboth\@gobbletwo
% \def\@oddhead{\hbox{}\rightmark\hfil\hspace{0.1in}\thepage} %% removed rm - thp 020227
% \def\@oddfoot{}
% \def\@evenhead{}
% \def\@evenfoot{}
% \def\sectionmark##1{}
% \def\subsectionmark##1{}}

%\pagestyle{myheadings}
%\raggedright
%BDB\hyphenpenalty 10000
\newcommand{\footmark}[1]{${}^{\mbox{\normalsize #1}}$}

%\long\def\@makecaption#1#2{%
% \let\BBAB\table@BBAB % thp 2005/07/23
% \vskip10pt%
% \setbox\@tempboxa\hbox{\parbox{\columnwidth}{#1\cap@line{\cap@style #2}}}%
% \ifdim\wd\@tempboxa>\hsize\parbox{\columnwidth}{#1\cap@line{\cap@style #2}}\par%
% \else\hbox to\hsize{\box\@tempboxa\hfil}\par%
% \fi%
% \let\BBAB\normal@BBAB % thp 2005/07/23
% \vskip\baselineskip}
%
%\long\def\@caption#1[#2]#3{\par\addcontentsline{\csname
%  ext@#1\endcsname}{#1}{\protect\numberline{\csname
%  the#1\endcsname}{\ignorespaces #2}}\begingroup
%    \@parboxrestore
%    \normalsize\@ifundefined{hrm}{}{\hrm}
%    \@makecaption{\csname fnum@#1\endcsname}{\ignorespaces #3 } \par
%  \endgroup}
%
%\def\caption{\refstepcounter\@captype \@dblarg{\@caption\@captype}}
%
%\def\centeredcaption#1{\caption}
%
%\def\fnum@figure{\def\cap@style{}\def\cap@line{}{\em\figurename\ {\thefigure}}. }
%% added second set of braces around \em  to get citations in man mode -- tp 17/7/2000
%% then removed them again because they were cancelling application of em to the caption
%\long\def\fnum@table{\def\cap@style{\em}\def\cap@line{\\ }\tablename\ \thetable }

\setcounter{topnumber}{1}
\def\topfraction{.7}
\setcounter{bottomnumber}{1}
\def\bottomfraction{.6}
\setcounter{totalnumber}{1}
\def\textfraction{0}
\def\floatpagefraction{.7}
\setcounter{dbltopnumber}{1}
\def\dbltopfraction{.7}
\def\dblfloatpagefraction{.7}
\def\dbltextfloatsep{\textfloatsep}

\fussy
%\setlength{\parindent}{0.4in}
%\let\oldfootnote=\footnote % 3/10/00
%\let\footnote=\endnote
%\def\footnotesize{\normalsize\rm}

\@ifundefined{def@nosf}{%
\def\helvetica{%
\ClassWarning{apa6}{ignored \string\helvetica\space (use helv option)}
}}{\def\helvetica{\relax}}

%==== Appendix definitions added by Michael Erickson 2000/07/11

%\efloat@condopen{fff}
%\efloat@iwrite{fff}{\string\apaSixappfig}
%\efloat@iwrite{fff}{\string\setcounter{appendix}{0}}
%\efloat@condopen{ttt}
%\efloat@iwrite{ttt}{\string\apaSixapptab}
%\efloat@iwrite{ttt}{\string\setcounter{appendix}{0}}

}% end of man mode (manuscript format)


%%%%%%%%%%%%%%%%%%%%
%%                %%
%% JOURNAL FORMAT %%
%%                %%
%%%%%%%%%%%%%%%%%%%%

\@ifundefined{def@jou}{}{%


%========== balance last-page columns =================

% need the `keeplastbox' option to allow final reference to be
% properly indented on the final line
% https://tex.stackexchange.com/questions/198287/reference-and-biography-indentation-issue
\IfFileExists{flushend.sty}{\RequirePackage[keeplastbox]{flushend}}{}

%========== ftnright and alterations ==================

\IfFileExists{ftnright.sty}{
 \let\savefootnoterule\footnoterule
 \let\save@makefntext\@makefntext
 \RequirePackage{ftnright}
 \let\footnoterule\savefootnoterule
 \let\@makefntext\save@makefntext
}{}

%======================================================

\def\rightheader#1{\def\r@headr{\protect\MakeUppercase{\protect\scriptsize #1}}}
\def\leftheader#1{\def\r@headl{\protect\MakeUppercase{\protect\scriptsize #1}}}
\def\r@headr{\protect\MakeUppercase{\protect\scriptsize\@shorttitle}}% BDB
%%%%%%%%%%%%%%%%%%%%%%%\def\shorttitle#1{\def\r@headr{\protect\MakeUppercase{\protect\scriptsize #1}}}% BDB

\def\put@one@authaffil#1#2{%
  \parbox[t]{\textwidth}{\begin{center}{\large #1\vspace{0in}}%
                        {\\ #2\vspace{0.05in}\\}\end{center}}}

\newsavebox\auone@box
\newsavebox\autwo@box
\newsavebox\autot@box
\newlength\auone@boxwidth
\newlength\autwo@boxwidth
\newlength\autot@boxwidth

\def\default@d@authaffil#1#2#3#4{%
        \parbox[t]{\columnwidth}{\begin{center}{\large #1\vspace{0in}}%
                                {\\ #2\vspace{0.05in}\\}\end{center}}%
        \parbox[t]{\columnwidth}{\begin{center}{\large #3\vspace{0in}}%
                                {\\ #4\vspace{0.05in}\\}\end{center}}}

\def\uneven@d@authaffil#1#2#3#4{%
     \hfill\parbox[t]{\auone@boxwidth}{\begin{center}{\large #1\vspace{0in}}%
                                      {\\ #2\vspace{0.05in}\\}\end{center}}\hfill\hfill%
           \parbox[t]{\autwo@boxwidth}{\begin{center}{\large #3\vspace{0in}}%
                                      {\\ #4\vspace{0.05in}\\}\end{center}}\hfill}

\def\put@two@authaffil#1#2#3#4{%
     \let\disp@authaffil\default@d@authaffil
     \sbox\auone@box{\begin{tabular}{c}\large #1\\ #2\end{tabular}}
     \settowidth{\auone@boxwidth}{\usebox\auone@box}
     \sbox\autwo@box{\begin{tabular}{c}\large #3\\ #4\end{tabular}}
     \settowidth{\autwo@boxwidth}{\usebox\autwo@box}
     \ifdim\auone@boxwidth<1.25\columnwidth
      \ifdim\autwo@boxwidth<1.25\columnwidth
       \sbox\autot@box{\usebox\auone@box\hspace{0.4in}\usebox\autwo@box}
       \settowidth{\autot@boxwidth}{\usebox\autot@box}
       \ifdim\autot@boxwidth<\textwidth
        \let\disp@authaffil\uneven@d@authaffil
       \fi
      \fi
     \fi
     \ifdim\auone@boxwidth<\columnwidth
      \ifdim\autwo@boxwidth<\columnwidth
       \let\disp@authaffil\default@d@authaffil
      \fi
     \fi
     \disp@authaffil{#1}{#2}{#3}{#4}
}

\def\maketitle{
 \check@author
 \@ifundefined{r@headr}{\def\r@headr{\protect\MakeUppercase{\protect\scriptsize\@title}}}{}
 \@ifundefined{r@headl}{\def\r@headl{\protect\MakeUppercase{\protect\scriptsize\@author}}}{}

\twocolumn[  % anything appearing within the brackets is set in one-column mode
  \vspace{0.03in}
  \begin{center}
% title
  {\LARGE \@title}\\
  \vspace{-0.05in}

  \@ifundefined{apaSix@maskauthoridentity}{%  BDB

      \@ifundefined{@authorTwo}{
    % one author-affiliation
      \put@one@authaffil{\@author}{\@affil}}{
      \@ifundefined{@authorThree}{
    % two authors-affiliations
      \put@two@authaffil{\@authorOne}{\@affilOne}{\@authorTwo}{\@affilTwo}}{
      \@ifundefined{@authorFour}{
    % three authors-affiliations
      \@ifundefined{@twofirst}{
    % first one, then two
      \put@one@authaffil{\@authorOne}{\@affilOne}\vspace{-0.15in}\\
      \put@two@authaffil{\@authorTwo}{\@affilTwo}{\@authorThree}{\@affilThree}
      }{
    % first two, then one
      \put@two@authaffil{\@authorOne}{\@affilOne}{\@authorTwo}{\@affilTwo}\vspace{-0.15in}\\
      \put@one@authaffil{\@authorThree}{\@affilThree}
      }}{
      \@ifundefined{@authorFive}{ % 2006/01/05 as contributed by Aaron Geller
    % four authors-affiliations
      \put@two@authaffil{\@authorOne}{\@affilOne}{\@authorTwo}{\@affilTwo}\vspace{-0.15in}\\
      \put@two@authaffil{\@authorThree}{\@affilThree}{\@authorFour}{\@affilFour}
      }{                          % 2006/01/05 beginning of Aaron Geller contribution
      \@ifundefined{@authorSix}{ % -- thp 2006/01/05
    % five authors-affiliations
      \put@two@authaffil{\@authorOne}{\@affilOne}{\@authorTwo}{\@affilTwo}\vspace{-0.15in}\\
      \put@two@authaffil{\@authorThree}{\@affilThree}{\@authorFour}{\@affilFour}%
      \vspace{-0.15in}\\ % thp added negative vertical space
      \put@one@authaffil{\@authorFive}{\@affilFive}
      }{                          % 2006/01/05 end of Aaron Geller contribution
    % six authors-affiliations
    %% --- thp 2006/01/05 beginning of six-author display
      \put@two@authaffil{\@authorOne}{\@affilOne}{\@authorTwo}{\@affilTwo}\vspace{-0.15in}\\
      \put@two@authaffil{\@authorThree}{\@affilThree}{\@authorFour}{\@affilFour}\vspace{-0.15in}\\
      \put@two@authaffil{\@authorFive}{\@affilFive}{\@authorSix}{\@affilSix}
    %% --- thp 2006/01/05 end of six-author display
      }}}}}
      \@ifundefined{@note}
       {\vspace{0.07in}}
       {\vspace{0.07in}\\ {\large\@note\vspace{0.07in}}}

  }{%  mask author identity -- show nothing in the author or author space
    \vspace{0.32in}
  }


  \@ifundefined{@abstract}
  {\par }
  {\par \parbox{4.6875in}
   {\small \noindent \@abstract
     \@ifundefined{@keywords}{}{%
      \par\vspace{0.12in}\raggedright\textit{\keywordname:} \@keywords%
     }%
   }
   \vspace{0.24in}%
  }
  \end{center}
 ] % end of \twocolumn[]

 \pagenumbering{arabic}
 \@ifundefined{@journal}{\thispagestyle{empty}}{%
  \@ifundefined{@vvolume}{\def\@vvolume{\strut}}{}%
  \@ifundefined{@copnum}{\def\@copnum{\strut}}{}%
  \@ifundefined{@ccoppy}{\def\@ccoppy{\strut}}{}%
  \fancyhead{}
  \fancyhead[LO]{\stiny{\@journal}\vspace{-0.15\baselineskip}\\
                 \stiny{\@vvolume}}
  \fancyhead[RO]{\stiny{\@ccoppy}\vspace{-0.15\baselineskip}\\
                 \stiny{\@copnum}}
  \fancyfoot[CO]{\small\rm\thepage}
  % the following are needed if the starting page number is changed to
  % an even number:
  \fancyhead[LE]{\stiny{\@journal}\vspace{-0.15\baselineskip}\\
                 \stiny{\@vvolume}}
  \fancyhead[RE]{\stiny{\@ccoppy}\vspace{-0.15\baselineskip}\\
                 \stiny{\@copnum}}
  \fancyfoot[CE]{\small\rm\thepage}
  \renewcommand{\headrulewidth}{0pt}
  \renewcommand{\footrulewidth}{0pt}
  \thispagestyle{fancy}
 }

  \@ifundefined{apaSix@maskauthoridentity}{%  BDB
     \@ifundefined{@acks}
      {}
      {\begin{figure}[b]
       \parbox{\columnwidth}{\setlength{\parindent}{0.18in}
       \noindent\makebox[\columnwidth]{\vrule height0.125pt width\columnwidth}\vspace*{0.05in}\par
       {\footnotesize\hspace{-0.04in}\@acks\par}}
       \end{figure}}
  }{%  mask author identity -- show nothing in the author note space
  }

  \@ifundefined{apaSix@maskauthoridentity}{%  BDB
     \markboth{\hfill\r@headl\hfill}{\hfill\r@headr\hfill}
  }{%  mask author identity -- show the short title for both the left and right headers
     \markboth{\hfill\r@headr\hfill}{\hfill\r@headr\hfill}
  }
 \@ifundefined{no@tab}{\let\tabular\apaSixtabular}{}
 %\noindent
}

\newcommand\stiny{\@setfontsize\stiny\@vipt\@viipt}

%\ThreeLevelHeading
%\def\baselinestretch{0.923}
%\def\refname{References}
\setlength{\footnotesep}{0.2813in}
\setlength{\topmargin}{-0.275in}
\addtolength{\headheight}{0.02in}
\addtolength{\headsep}{-0.156in}
\setlength{\oddsidemargin}{-0.25in}
\setlength{\evensidemargin}{-0.25in}
\setlength{\textwidth}{6.94in}
\setlength{\textheight}{8.9in}
\setlength{\columnwidth}{8.5cm}
\setlength{\columnsep}{0.25in}
\setlength{\parindent}{0.15625in}
%%\setlength{\parskip}{0in}
\setlength{\textfloatsep}{0.35in}
%\flushbottom

\setcounter{secnumdepth}{0}

\def\ps@myheadings{%
  \let\@mkboth\@gobbletwo
  \def\@oddhead{\hbox{}\rightmark \hfil\rm\thepage}
  \def\@oddfoot{}
  \def\@evenhead{\rm\thepage\hfil\leftmark\hbox{}}
  \def\@evenfoot{}
  \def\sectionmark##1{}
  \def\subsectionmark##1{}
}
\pagestyle{myheadings}

%\long\def\@makecaption#1#2{%
%  \let\BBAB\table@BBAB % thp 2005/07/23
%  \vskip\abovecaptionskip
%  \sbox\@tempboxa{#1 #2}%
%  \ifdim \wd\@tempboxa >\hsize
%    #1 #2\par
%  \else
%    \global \@minipagefalse
%    \hb@xt@\hsize{\hfil\box\@tempboxa\hfil}%
%  \fi
%  \let\BBAB\normal@BBAB % thp 2005/07/23
%  \vskip\belowcaptionskip}
%
%\long\def\@makeccaption#1#2#3{
% \let\BBAB\table@BBAB % thp 2005/07/23
% \vskip\abovecaptionskip
% \setbox\@tempboxa\hbox{#1 #2}
% \ifdim \wd\@tempboxa > \hsize
%    \hbox to\hsize{\hfil\parbox[t]{#3}{#1 #2}\hfil}
%  \else
%    \ifdim \wd\@tempboxa > #3
%      \hbox to\hsize{\hfil\parbox[t]{#3}{#1 #2}\hfil}
%    \else
%      \hbox to\hsize{\hfil\mbox{#1 #2}\hfil}
%    \fi
% \fi
% \let\BBAB\normal@BBAB % thp 2005/07/23
% \vskip\belowcaptionskip}
%
%\long\def\@caption#1[#2]#3{\par\addcontentsline{\csname
%  ext@#1\endcsname}{#1}{\protect\numberline{\csname
%  the#1\endcsname}{\ignorespaces #2}}\begingroup
%    \@parboxrestore
%    \normalsize
%    \csname setcskip@#1\endcsname
%    \@makecaption{\csname fnum@#1\endcsname}{\ignorespaces #3}\par
%  \endgroup}
%
%\long\def\@ccaption#1[#2]#3#4{\def\@ccwidth{#3} \par\addcontentsline{\csname
%  ext@#1\endcsname}{#1}{\protect\numberline{\csname
%  the#1\endcsname}{\ignorespaces #2}}\begingroup
%    \@parboxrestore
%    \normalsize
%    \@makeccaption{\csname fnum@#1\endcsname}{\ignorespaces #4}{\@ccwidth}\par
%  \endgroup}
%
%\def\caption{\refstepcounter\@captype \@dblarg{\@caption\@captype}}
%
%\long\def\centeredcaption{\refstepcounter\@captype \@dblarg{\@ccaption\@captype}}
%
%\def\setcskip@table{\setlength{\abovecaptionskip}{3.2pt}\setlength{\belowcaptionskip}{2pt}}
%\def\setcskip@figure{\setlength{\abovecaptionskip}{-0.04in}\setlength{\belowcaptionskip}{-1.9pt}}
%
%\def\fnum@figure{\small{\em\figurename\ {\thefigure}}.\hspace{0.07in}}
%\long\def\fnum@table{\makebox[\linewidth][l]{\tablename\ \thetable} \\ \em }
%
%\def\@makefnmark{\hbox{\@textsuperscript{\normalfont{\scriptsize\@thefnmark}}}}%
%\long\def\@makefntext#1{\parindent 1em\noindent
%           \hb@xt@1.8em{%
%           \hss\@textsuperscript{\normalfont{\tiny\@thefnmark}\hspace{1.5pt}}}#1}%


\setcounter{topnumber}{2}
\def\topfraction{.85}
\setcounter{bottomnumber}{2}
\def\bottomfraction{.75}
\setcounter{totalnumber}{3}
\def\textfraction{.10}
\def\floatpagefraction{.85}
\setcounter{dbltopnumber}{2}
\def\dbltopfraction{.85}
\def\dblfloatpagefraction{.85}
\def\dbltextfloatsep{0.8\textfloatsep}
\let\footnotesize=\small

\def\helvetica{\relax}

\doublehyphendemerits5000
\hfuzz0pt
\tolerance=9999
\pretolerance=-1
\emergencystretch=25pt
\hbadness=30000
\hyphenpenalty=100


\@ifundefined{def@apacite}{}{% -- Philip Kime 2008/12/03
  % Removed some bibliography redefinitions as per the instructions of Erik Meijer
  \bibleftmargin=1.2em        % left these in because the default is too big
  \bibindent=-\bibleftmargin  % and this is apparently not refefined each time
  \renewcommand{\bibliographytypesize}{\footnotesize}}
\@ifundefined{def@natbib}{}{% -- Philip Kime 2008/12/03
  % Removed some bibliography redefinitions as per the instructions of Erik Meijer
  \bibleftmargin=1.2em        % left these in because the default is too big
  \bibindent=-\bibleftmargin  % and this is apparently not refefined each time
  \renewcommand{\bibliographytypesize}{\footnotesize}}



%\@ifundefined{def@noapacite}{% -- Philip Kime 2008/12/03
%  % Removed some bibliography redefinitions as per the instructions of Erik Meijer
%  \bibleftmargin=1.2em        % left these in because the default is too big
%  \bibindent=-\bibleftmargin  % and this is apparently not refefined each time
%  \renewcommand{\bibliographytypesize}{\footnotesize}}{}

}% end of jou mode (journal format)

%%%%%%%%%%%%%%%%%%%%
%%                %%
%% REGULAR FORMAT %%
%%                %%
%%%%%%%%%%%%%%%%%%%%

\@ifundefined{def@doc}{}{%

%\pagestyle{myheadings}

%%%%%%%%%%%%%%%%%%%%%%%%%%%%%%%%%%%%%%%%%%%%\renewcommand{\shorttitle}[1]{\def\@shorttitle{#1}}

\def\leftheader#1{\def\r@headl{#1}}

\RequirePackage{fancyhdr}
\setlength{\headheight}{15.2pt}
\fancyhf{}
\renewcommand{\headrulewidth}{0pt}

\fancypagestyle{otherpage}{%
% %%%%%%%%%%%%%%%%%%%%%%%%%%%%%%%%%%%%%%%%%%%%%%%%%%%%%%%%%%%%%%%%%%%%%%%%%%%%%%%%%%%%%%%%%%%%%%%%%%%%%%%%%%%%%%%%%%%%%%%
    \lhead{\MakeUppercase{\@shorttitle}}%
    \rhead{\thepage}%
}
\pagestyle{otherpage}

\def\maketitle{
 \global\@topnum\z@  % to prevent tables before the title   %% Erik Meijer, 2006/01/03
 \@ifundefined{@acks}{% if there acknowledgements they make up a "float" on the 1st page
 \global\@botnum\z@}{% to prevent tables on the first page  %% Erik Meijer, 2006/01/03
 \global\@botnum\@ne}% to prevent tables below the footnote -- thp 2006/01/10
 \check@author
% %%%%%%%%%%%%%%%%%%%%%%%%%%%%%%%%%%%%%%%%%%%%%%%%%%%%%%%%%%%%%%%%%%%%%%%%%%%%%%%%%%%%%%%%%%%%%%%%%%%%%%%%%%%%%%%%%%%%%%%
 \@ifundefined{r@headr}{\typeout{Using title for running head}
                        \def\r@headr{\protect\MakeUppercase{\@title}}
                        \markright{\rm \@title \protect\\ \thepage}}{}
 \@ifundefined{r@headl}{\let\r@headl\r@headr}{}
 \@ifundefined{s@title}{\let\s@title\r@headr}{}
  \sloppy
  \setlength{\parindent}{0.4in}
  \begin{center}
   \@ifundefined{@journal}{}{%
    \@ifundefined{@vvolume}{\def\@vvolume{}}{}%
    \@ifundefined{@copnum}{\def\@copnum{}}{}%
    \@ifundefined{@ccoppy}{\def\@ccoppy{}}{}%
    {\scriptsize{\@journal}}\hspace{\fill}{\scriptsize{\@ccoppy}}\vspace{-0.3\baselineskip}\\
    {\scriptsize{\@vvolume}}\hspace{\fill}{\scriptsize{\@copnum}}\vspace{0.1in}\\
   }
  \vspace*{0.3in}

  {\LARGE \@title}\\

  \vspace{0.3in}
  \@ifundefined{apaSix@maskauthoridentity}{%  BDB

      \@ifundefined{@authorTwo}{
    % one author-affiliation
      {\Large \@author} \\

      \@affil \vspace{0.1in} \\ }{
      \@ifundefined{@authorThree}{
    % two authors-affiliations
      {\Large \@authorOne} \\

      \@affilOne \vspace{0.1in} \\
      {\Large \@authorTwo}\\

      \@affilTwo \vspace{0.1in} \\ }{
      \@ifundefined{@authorFour}{
    % three authors-affiliations
      {\Large \@authorOne} \\

      \@affilOne \vspace{0.1in} \\
      {\Large \@authorTwo}\\

      \@affilTwo \vspace{0.1in} \\
      {\Large \@authorThree}\\

      \@affilThree \vspace{0.1in} \\ }{
      \@ifundefined{@authorFive}{ %% 2006/01/05 added as contributed by Aaron Geller
    % four authors-affiliations
      {\Large \@authorOne} \\

      \@affilOne \vspace{0.1in} \\
      {\Large \@authorTwo}\\

      \@affilTwo \vspace{0.1in} \\
      {\Large \@authorThree}\\

      \@affilThree \vspace{0.1in} \\
      {\Large \@authorFour}\\

      \@affilFour \vspace{0.1in} \\ }{  %%% 2006/01/05 beginning of Aaron Geller contribution
      \@ifundefined{@authorSix}{ %% -- thp 2006/01/05
    % five authors-affiliations
      {\Large \@authorOne} \\

      \@affilOne \vspace{0.1in} \\
      {\Large \@authorTwo}\\

      \@affilTwo \vspace{0.1in} \\
      {\Large \@authorThree}\\

      \@affilThree \vspace{0.1in} \\
      {\Large \@authorFour}\\

      \@affilFour \vspace{0.1in} \\
      {\Large \@authorFive}\\

      \@affilFive \vspace{0.1in} \\ }{  %%% 2006/01/05 end of Aaron Geller contribution
    % six authors-affiliations
    %% --- thp 2006/01/05 beginning of six-author display
      {\Large \@authorOne} \\

      \@affilOne \vspace{0.1in} \\
      {\Large \@authorTwo}\\

      \@affilTwo \vspace{0.1in} \\
      {\Large \@authorThree}\\

      \@affilThree \vspace{0.1in} \\
      {\Large \@authorFour}\\

      \@affilFour \vspace{0.1in} \\
      {\Large \@authorFive}\\

      \@affilFive \vspace{0.1in} \\
      {\Large \@authorSix}\\

      \@affilSix \vspace{0.1in} \\ }
    %% --- thp 2006/01/05 end of six-author display
    }}}}
    %
      \@ifundefined{@note}
       {\vspace*{\baselineskip} }
       {\@note\vspace{0.2in}}

  }{%  mask author identity -- show nothing in the author note space
  }

  \@ifundefined{@abstract}{}{%
   {\abstractname}\vspace{0.1in}% BDB

   \parbox{5in}{\@abstract%
     \@ifundefined{@keywords}{}{%
       \par\vspace{0.12in}\textit{\keywordname:} \@keywords%
     }%
   }\vspace{0.25in}%
  }
  \end{center}
 \pagenumbering{arabic}
 \thispagestyle{empty}

  \@ifundefined{apaSix@maskauthoridentity}{%  BDB

     \@ifundefined{@acks}
      {}
      {\begin{figure}[b]
       \parbox{\textwidth}{ \setlength{\parindent}{0.2in}
       \noindent \makebox[\linewidth]{\vrule height0.125pt width\linewidth}

       \vspace*{0.05in}
       {\footnotesize
       \indent \@acks

       }}
       \end{figure}}

  }{%  mask author identity -- show nothing in the author note space
  }

%%%%%%%%%%%%%%%%%%%%%%%%%%%%%%%%%%%%%%%%%%%%%%%%%%%%%%%%%%%%%%%%%%%%%%%%%%%%%%%%%%%%%%%%%%%%%%%%%%%%%%%%%%%%%%%%%%%%%%%%
 \markboth{\hfill \r@headl \hfill}{\hfill \r@headr \hfill}
 \let\tabular\apaSix@doc@tabular% -- thp 2006/01/02
 \let\endtabular\apaSix@doc@endtabular% -- thp 2006/01/03
% \noindent
}

%\ThreeLevelHeading
%\def\refname{References}
\setlength{\topmargin}{-0.25in}
\setlength{\oddsidemargin}{0.25in}
\setlength{\evensidemargin}{0.25in}
\setlength{\textwidth}{6in}
\setlength{\textheight}{8.5in}
\setcounter{secnumdepth}{0}
\flushbottom

% fix headwidth for even-numbered pages
% suggestion from http://tex.stackexchange.com/questions/42798/apa6-package-heading-line-exceeds-textwidth
\setlength{\headwidth}{\textwidth}

\setcounter{topnumber}{2}
\def\topfraction{.7}
\setcounter{bottomnumber}{2}
\def\bottomfraction{.6}
\setcounter{totalnumber}{3}
\def\textfraction{.2}
\def\floatpagefraction{.7}
\setcounter{dbltopnumber}{2}
\def\dbltopfraction{.8}
\def\dblfloatpagefraction{.8}
\def\dbltextfloatsep{0.8\textfloatsep}

\def\helvetica{\relax}
%%\def\timesroman{\relax}%    Commented out all \timesroman -- thp 2005/07/23
%\def\centeredcaption#1{\caption}
%
%\long\def\@makecaption#1#2{
% \let\BBAB\table@BBAB% thp 2005/07/23
% \vskip 10pt
% \setbox\@tempboxa\hbox{#1 #2}
% \ifdim \wd\@tempboxa >\hsize #1 #2\par \else \hbox
%to\hsize{\hfil\box\@tempboxa\hfil}
% \fi
% \let\BBAB\normal@BBAB% thp 2005/07/23
% \vskip 4pt}
%
%\long\def\@caption#1[#2]#3{\par\addcontentsline{\csname
%  ext@#1\endcsname}{#1}{\protect\numberline{\csname
%  the#1\endcsname}{\ignorespaces #2}}\begingroup
%    \@parboxrestore
%    \normalsize
%    \@makecaption{\csname fnum@#1\endcsname}{\ignorespaces #3}\par
%  \endgroup}
%
%\def\fnum@figure{\small {\em\figurename\ {\thefigure}}.\hspace{0.07in}}
%\long\def\fnum@table{\small\tablename\ {\thetable}:}


\@ifundefined{def@apacite}{}{% -- Philip Kime 2008/12/03
  % Removed defs as per the instructions of Erik Meijer -- thp 2005/07/23
  \renewcommand{\bibliographytypesize}{\small}}
\@ifundefined{def@natbib}{}{% -- Philip Kime 2008/12/03
  % Removed defs as per the instructions of Erik Meijer -- thp 2005/07/23
  \renewcommand{\bibliographytypesize}{\small}}

%\@ifundefined{def@noapacite}{% -- Philip Kime 2008/12/03
%  % Removed defs as per the instructions of Erik Meijer -- thp 2005/07/23
%  \renewcommand{\bibliographytypesize}{\small}}{}


}% end of doc mode (regular LaTeX format)

%---------- end of mode specific definitions -----------

% =======================================================
%       Some final definitions for all modes
% =======================================================

\let\ignore\@gobble


\@ifundefined{def@apacite}{}{% -- Philip Kime 2008/12/03
  \bibliographystyle{apacite}
  %
  % Thanks to Donald Arsenau for the right way to ignore \bibliographystyle
  %
  \def\bibliographystyle#1{\ClassWarning{apa6}{\string\bibliographystyle\space
      command ignored}}}

\@ifundefined{def@natbib}{}{% -- Philip Kime 2008/12/03
  \bibliographystyle{apacite}
  %
  % Thanks to Donald Arsenau for the right way to ignore \bibliographystyle
  %
  \def\bibliographystyle#1{\ClassWarning{apa6}{\string\bibliographystyle\space
      command ignored}}}

%\@ifundefined{def@noapacite}{% -- Philip Kime 2008/12/03
%  \bibliographystyle{apacite}
%  %
%  % Thanks to Donald Arsenau for the right way to ignore \bibliographystyle
%  %
%  \def\bibliographystyle#1{\ClassWarning{apa6}{\string\bibliographystyle\space
%      command ignored}}}{}




%    \end{macrocode}
%\end{macro}
%
%    \begin{macrocode}
%</class>
%    \end{macrocode}
%
%
% %%%%%%%%%%%  SAMPLE FILES  %%%%%%%%%%%%
%
%\begin{macro}{bibliography.bib}
%    \begin{macrocode}
%<*bibliography>

@ARTICLE{Borst2011a,
  author = {Borst, Gr\'{e}goire and Kosslyn, Stephen M. and Kievit, Rogier A. and  Thompson, William L.},
  title = {A fictitious entry to demonstrate complicated bibliographies},
  journal = {Journal of Nothingness},
  year = {2011},
  volume = {1},
  pages = {2--3}
}

@ARTICLE{Borst2011b,
  author = {Borst, Gr\'{e}goire and Kievit, Rogier A. and Thompson, William L. and
    Kosslyn, Stephen M.},
  title = {Mental rotation is not easily cognitively penetrable},
  journal = {Journal of Cognitive Psychology},
  year = {2011},
  volume = {23},
  pages = {60--75}
}

@ARTICLE{Borst2011c,
  author = {Borst, Gr\'{e}goire and Thompson, William L. and Kosslyn, Stephen M.},
  title = {Understanding the dorsal and ventral systems of the human cerebral
	cortex: {B}eyond dichotomies},
  journal = {American Psychologist},
  year = {2011},
  volume = {66},
  pages = {624--632}
}

@ARTICLE{deWaal2009,
  author = {de Waal, Elda and Grosser, M. M.},
  title = {Safety and security at school: {A} pedagogical perspective},
  journal = {Teaching and Teacher Education},
  year = {2009},
  volume = {25},
  pages = {697--706}
}

@ARTICLE{Franklin2010,
  author = {Franklin, Jr., Robert G. and Adams, Jr., Reginald B.},
  title = {The two sides of beauty: {L}aterality and the duality of facial attractiveness},
  journal = {Brain and Cognition},
  year = {2010},
  volume = {72},
  pages = {300--305}
}

@ARTICLE{Gilbert2004,
  author = {Gilbert, David and McClernon, Joseph and Rabinovich, Norka and Sugai, Chihiro
    and Plath, Louisette and Asgaard, Greg and Zuo, Yantao and	Huggenvik, Jodi and Botros, Nazeih},
  title = {Effects of quitting smoking on {EEG} activation and attention last for more than 31
    days and are more severe with stress, dependence,	{DRD2 A1} allele, and depressive traits},
  journal = {Nicotine \& Tobacco Research},
  year = {2004},
  volume = {6},
  pages = {249--267},
  number = {2},
  doi = {10.1080/14622200410001676305}
}

@INBOOK{Haybron2008,
  author = {Haybron, Daniel M.},
  title = {Philosophy and the science of subjective well-being},
  editor = {Eid, Michael and Larsen, Randy J.},
  booktitle = {The science of subjective well-being},
  year = {2008},
  pages = {17--43},
  publisher = {Guilford Press},
  address = {New York, NY}
}

@ARTICLE{Herbst-Damm2005,
  author = {Herbst-Damm, Kathryn L. and Kulik, James A.},
  title = {Volunteer support, marital status, and the survival times of terminally ill patients},
  journal = {Health Psychology},
  year = {2005},
  volume = {24},
  pages = {225--229},
  number = {2}
}

@ARTICLE{Lassen2006,
  author = {Lassen, Stephen R. and Steele, Michael M. and Sailor, Wayne},
  title = {The relationship of school-wide positive behavior support to academic achievement in
    an urban middle school},
  journal = {Psychology in the Schools},
  year = {2006},
  volume = {43},
  pages = {701--712},
  number = {6}
}

@ARTICLE{JRLevin2000,
  author = {Levin, Joel R. and O'Donnell, Angela M.},
  title = {What to do about educational research's credibility gaps?},
  journal = {Journal of Cognitive Education and Psychology},
  year = {2000},
  volume = {1},
  pages = {201--256}
}

@ARTICLE{MLevin1990,
  author = {Levin, Mary E. and Levin, Joel R.},
  title = {Scientific mnemonomies: {M}ethods for maximizing more than memory},
  journal = {American Educational Research Journal},
  year = {1990},
  volume = {27},
  pages = {301--321}
}

@ARTICLE{MayerInPressA,
  author = {Mayer, Richard E.},
  title = {Fake article to illustrate in-press citations},
  journal = {Journal of Cognitive Psychology},
  year = {in press}
}

@ARTICLE{MayerInPressB,
  author = {Mayer, Richard E.},
  title = {Second fake article to illustrate in-press citations},
  journal = {Journal of Educational Psychology},
  year = {in press}
}

@ARTICLE{Mayer2008a,
  author = {Mayer, Richard E.},
  title = {Applying the science of learning: {E}vidence-based principles for
	the design of multimedia instruction},
  journal = {American Psychologist},
  year = {2008},
  volume = {63},
  pages = {760--769}
}

@ARTICLE{Mayer2008b,
  author = {Mayer, Richard E.},
  title = {Old advice for new researchers},
  journal = {Educational Psychology Review},
  year = {2008},
  volume = {20},
  pages = {19--28}
}

@BOOK{Shotton1989,
  author = {Shotton, Margaret A.},
  title = {Computer addiction? {A} study of computer dependency},
  publisher = {Taylor \& Francis},
  year = {1989},
  location = {London, England}
}

@ARTICLE{vonDavier2011,
  author = {von Davier, Matthias and Xu, Xueli and Carstensen, Claus H.},
  title = {Measuring growth in a longitudinal large-scale assessment with a
    general latent variable model},
  journal = {Psychometrika},
  year = {2011},
  volume = {76},
  pages = {318--336}
}

%</bibliography>
%    \end{macrocode}
%\end{macro}
%
%\begin{macro}{shortsample.tex}
%    \begin{macrocode}
%<*shortsample>
\documentclass[jou]{apa6}

\usepackage[american]{babel}

\usepackage{csquotes}
\usepackage[style=apa6,sortcites=true,sorting=nyt,backend=biber]{biblatex}
\DeclareLanguageMapping{american}{american-apa}
\addbibresource{bibliography.bib}

\title{Sample APA-Style Document Using the \textsf{apa6} Package}

\author{Brian D.\ Beitzel}
\affiliation{SUNY Oneonta}

\leftheader{Beitzel}

\abstract{This demonstration paper uses the \textsf{apa6} \LaTeX\
  class to format the document in compliance with the 6th Edition of
  the American Psychological Assocation's \textit{Publication Manual.}
  The references are managed using \textsf{biblatex}.}

\keywords{APA style, demonstration}

\begin{document}
\maketitle
We begin with \textcite{Shotton1989}.  We can also cite this work in
parenthesis, like this: \parencite{Shotton1989}.

A three-author paper \parencite[e.g.,][]{Lassen2006} lists all
three authors for the first citation, then only the first author
on all subsequent citations \parencite{Lassen2006}.

Note the use of five heading levels throughout this demonstration
Method section.

\section{Method}
\subsection{Participants}
We had a lot of people in this study.

\subsection{Materials}
Several materials were used for this project.  Some of themwere
already created for prior research.

\subsubsection{Paper-and-Pencil Instrument}
We used an instrument that we found to be highly successful.

\paragraph{Reliability}
The reliability of this instrument is extraordinary.

\paragraph{Validity}
We now discuss the validity of our instrument.

\subparagraph{Face validity} The face validity is exceptionally
strong.  Everyone should be impressed.

\subparagraph{Construct validity} Also very strong.

\subsection{Design}
This section describes the study's design.

\subsection{Procedure}
The procedure was fairly straightforward, yet required
attention to detail.

\section{Results}
Table \ref{tab:ComplexTable} contains some sample data.  Our
statistical prowess in analyzing these data is unmatched.

\begin{table}[htbp]
  \vspace*{2em}
  \begin{threeparttable}
    \caption{A Complex Table}
    \label{tab:ComplexTable}
    \begin{tabular}{@{}lrrr@{}}         \toprule
    Distribution type  & \multicolumn{2}{l}{Percentage of} & Total number   \\
                       & \multicolumn{2}{l}{targets with}  & of trials per  \\
                       & \multicolumn{2}{l}{segment in}    & participant    \\ \cmidrule(r){2-3}
                                    &  Onset  &  Coda            &          \\ \midrule
    Categorical -- onset\tabfnm{a}  &    100  &     0            &  196     \\
    Probabilistic                   &     80  &    20\tabfnm{*}  &  200     \\
    Categorical -- coda\tabfnm{b}   &      0  &   100\tabfnm{*}  &  196     \\ \midrule
    \end{tabular}
    \begin{tablenotes}[para,flushleft]
        {\small
            \textit{Note.} All data are approximate.

            \tabfnt{a}Categorical may be onset.
            \tabfnt{b}Categorical may also be coda.

            \tabfnt{*}\textit{p} < .05.
            \tabfnt{**}\textit{p} < .01.
         }
    \end{tablenotes}
  \end{threeparttable}
\end{table}

\section{Discussion}
This is a lengthy and erudite discussion.  It demonstrates amazing
skill in interpreting the results for the masses.

\printbibliography

\end{document}

%</shortsample>
%    \end{macrocode}
%\end{macro}
%
%\begin{macro}{longsample.tex}
%    \begin{macrocode}
%<*longsample>
\documentclass[man]{apa6}

\usepackage{lipsum}

\usepackage[american]{babel}

\usepackage{csquotes}
\usepackage[style=apa6,sortcites=true,sorting=nyt,backend=biber]{biblatex}
\DeclareLanguageMapping{american}{american-apa}
\addbibresource{bibliography.bib}

\title{Sample APA-Style Document Using the \textsf{apa6} Package}
\shorttitle{Sample Document}

\author{Brian D.\ Beitzel}
\affiliation{SUNY Oneonta}

\leftheader{Beitzel}

\abstract{\lipsum[1]}

\keywords{APA style, demonstration}

\authornote{Brian D.\ Beitzel, Department of Educational Psychology,
  Counseling and Special Education, SUNY Oneonta.

  Correspondence concerning this article should be addressed to Brian
  D.\ Beitzel, Department of Educational Psychology, Counseling and
  Special Education, SUNY Oneonta, 366 Fitzelle Hall, Oneonta, NY
  13820.  E-mail: beitzebd@oneonta.edu}


\begin{document}
\maketitle
\lipsum[2]

\Textcite{vonDavier2011} said this,
too \parencite{vonDavier2011,Lassen2006}.  Further evidence comes from
other sources \parencite{Shotton1989,Lassen2006}.  \lipsum[3]

\section{Method}
\subsection{Participants}
\lipsum[4]

\subsection{Materials}
\lipsum[5]

\subsection{Design}
\lipsum[6]

\subsection{Procedure}
\lipsum[7]

\subsubsection{Instrument \#1}
\lipsum[8]

\paragraph{Reliability}
\lipsum[9]

\subparagraph{Inter-rater reliability}
\lipsum[10]

\subparagraph{Test-retest reliability}
\lipsum[11]

\paragraph{Validity}
\lipsum[12]

\subparagraph{Face validity}
\lipsum[13]

\subparagraph{Construct validity}
\lipsum[14]


\section{Results}
Table~\ref{tab:BasicTable} summarizes the data. \lipsum[15]

\begin{table}
  \caption{Sample Basic Table}
  \label{tab:BasicTable}
  \begin{tabular}{@{}llr@{}}         \toprule
  \multicolumn{2}{c}{Item}        \\ \cmidrule(r){1-2}
  Animal    & Description & Price \\ \midrule
  Gnat      & per gram    & 13.65 \\
            & each        &  0.01 \\
  Gnu       & stuffed     & 92.50 \\
  Emu       & stuffed     & 33.33 \\
  Armadillo & frozen      &  8.99 \\ \bottomrule
  \end{tabular}
\end{table}

\begin{figure}
    \includegraphics[bb=0in 0in 2.5in 2.5in, height=2.5in, width=2.5in]{Figure1.pdf}
    \caption{This is my first figure caption.}
    \label{fig:Figure1}
\end{figure}

Figure~\ref{fig:Figure1} shows this trend. \lipsum[16]

\section{Discussion}
\lipsum[17]

\lipsum[18]

\lipsum[19]


\printbibliography

\appendix

\section{Instrument}
\label{app:instrument}

As shown in Figure~\ref{fig:Figure2}, these results are impressive. \lipsum[20]

\begin{figure}
    \includegraphics[bb=0in 0in 2.5in 2.5in, height=2.5in, width=2.5in]{Figure1.pdf}
    \caption{This is my second figure caption.}
    \label{fig:Figure2}
\end{figure}

\lipsum[21]
\section{Pilot Data}
\label{app:surveydata}

The detailed results are shown in Table~\ref{tab:DeckedTable}. \lipsum[22]

\begin{table}
  \begin{threeparttable}
    \caption{A More Complex Decked Table}
    \label{tab:DeckedTable}
    \begin{tabular}{@{}lrrr@{}}         \toprule
    Distribution type  & \multicolumn{2}{l}{Percentage of} & Total number   \\
                       & \multicolumn{2}{l}{targets with}  & of trials per  \\
                       & \multicolumn{2}{l}{segment in}    & participant    \\ \cmidrule(r){2-3}
                                    &  Onset  &  Coda            &          \\ \midrule
    Categorical -- onset\tabfnm{a}  &    100  &     0            &  196     \\
    Probabilistic                   &     80  &    20\tabfnm{*}  &  200     \\
    Categorical -- coda\tabfnm{b}   &      0  &   100\tabfnm{*}  &  196     \\ \midrule
    \end{tabular}
    \begin{tablenotes}[para,flushleft]
        {\small
            \textit{Note.} All data are approximate.

            \tabfnt{a}Categorical may be onset.
            \tabfnt{b}Categorical may also be coda.

            \tabfnt{*}\textit{p} < .05.
            \tabfnt{**}\textit{p} < .01.
         }
    \end{tablenotes}
  \end{threeparttable}
\end{table}

\lipsum[23]

\end{document}

%</longsample>
%    \end{macrocode}
%\end{macro}
%
%
% %%%%%%%%%%%  PTEX FILE (for Tex2Word)  %%%%%%%%%%%%
%
%\begin{macro}{apa6.ptex}
%    \begin{macrocode}
%<*ptex>

%% pseudo {apa6}
%% Copyright (C) 2001-2011 Chikrii Softlab.
%% All rights reserved.
%% http://www.chikrii.com
%% mailto: support@chikrii.com
%% License: You are allowed to create your own translators based
%% on the contents of this file solely for use with TeX2Word.
%% Chikrii Softlab is not responsible for any damages caused by the
%% use of this file or derived works.
%%
%% Modified by Brian Beitzel for the 'apa6' class
%% Distributed with permission from Chikrii Softlab

\DeclareOption{mask}{\def\apaSix@maskauthoridentity{\relax}}

\ProcessOptions\relax

\newenvironment{abstract}{\section*{\abstractname}}{\relax}

\newenvironment{titlepage}{\relax}{\relax}
\newcommand\appendix{\par
  \setcounter{section}{0}%
  \setcounter{subsection}{0}%
  \gdef\thesection{\@Alph\c@section}}
\renewcommand\theequation{\@arabic\c@equation}
\renewcommand\thefigure{\@arabic\c@figure}
\renewcommand\thetable{\@arabic\c@table}

\newcommand\tableofcontents{\section*{\contentsname}\entity@toc@placeholder}
\newcommand\listoffigures{\section*{\listfigurename}}
\newcommand\listoftables{\section*{\listtablename}}

\newenvironment{theindex}{\relax}{\relax}
\newcommand\contentsname{Contents}
\newcommand\listfigurename{List of Figures}
\newcommand\listtablename{List of Tables}
\newcommand\refname{References}
\newcommand\indexname{Index}
\newcommand\figurename{Figure}
\newcommand\tablename{Table}
\newcommand\partname{Part}
\newcommand\appendixname{Appendix}
\newcommand\abstractname{Abstract}
\def\today{\ifcase\month\or
  January\or February\or March\or April\or May\or June\or
  July\or August\or September\or October\or November\or December\fi
  \space\number\day, \number\year}


\def\shorttitle#1{\gdef\@shorttitle{#1}\relax}
\def\leftheader#1{\gdef\@leftheader{#1}\relax}
\long\def\abstract#1{\gdef\@abstract{#1}\relax}
\long\def\keywords#1{\gdef\@keywords{#1}\relax}
\long\def\authornote#1{\gdef\@authornote{#1}\relax}
\long\def\note#1{\gdef\@note{#1}\relax}

% for the booktabs package
\long\def\cmidrule#1{\gdef\@cmidrule{#1}}

% for superscripts
\newcommand{\tabfnm}[1]{\textsuperscript{#1}}
\newcommand{\tabfnt}[1]{\textsuperscript{#1}}
\newcommand{\tnote}[1]{\textsuperscript{#1}}

% table definition
\newcommand\table[1][tbp]{%
	\begingroup%
    \entity@paragraph@first=0%
	\let\caption=\tab@caption}
\def\endtable{\par\endgroup}

% table and figure captions
\@namedef{caption*}#1{\entity@paragraph@new\entity@paragraph@just=1\relax%
{\normalsize\rm{#1}\par}}
\newcommand{\tab@caption}[2][?]{\refstepcounter{table}\entity@paragraph@new\relax%
		{\entity@paragraph@first=0\normalsize\rm{\tablename~\thetable\par{\it #2}\par}}%
}
\newcommand{\fig@caption}[2][?]{\refstepcounter{figure}\entity@paragraph@new\relax%
		{\entity@paragraph@first=0\normalsize\rm{{\it \figurename~\thefigure.}\space{#2}\par}}%
}

% redefine sections to be non-numbered
\newcommand\@secdef@nonum[7]{%
        \begingroup%
        \entity@paragraph@style=#1%
        \entity@paragraph@span@font=0%
        \entity@paragraph@just=#2%
        \entity@paragraph@spacebefore=#3%
        \entity@paragraph@spaceafter=#4%
        \entity@paragraph@keepwithnext=1%
        \entity@paragraph@linespacingmultiple=1%
        \entity@paragraph@spacebetween=480%
        \entity@paragraph@sbasedon=0\entity@paragraph@snext=0%
        \relax #5\relax%
        \stylesheet@style#1=#7; aka Heading #1;%
        \endgroup%
        \@namedef{#6}{%
        \def\@sec@hook@setstyle{\entity@paragraph@style=#1\relax}%
        \def\@sec@hook@refstepcounter{\relax}%
        \def\@sec@hook@prefix{\relax}%
        \@sec@hook}}


% \@sec@hook that does not start a new paragraph
\def\@sec@hook@paragraphlevel{%
        \entity@paragraph@new%
        \entity@paragraph@linespacingmultiple=1%
        \entity@paragraph@spacebetween=480%
        \@@sec@hook@paragraphlevel}
\def\@@sec@hook@paragraphlevel{%
        \@ifnextchar*{\@@@sec@hook@paragraphlevel}{\@@@@sec@hook@paragraphlevel}}
\def\@@@sec@hook@paragraphlevel*#1{%
        {\@sec@hook@setstyle{#1}\entity@paragraph@new}}
\def\@@@@sec@hook@paragraphlevel{%
        \@ifnextchar[{\@@@@@sec@hook@paragraphlevel}{\@@@@@@sec@hook@paragraphlevel}}
\def\@@@@@sec@hook@paragraphlevel[#1]#2{%
        {\@sec@hook@setstyle\@sec@hook@refstepcounter\@sec@hook@prefix{#2}.}}
\def\@@@@@@sec@hook@paragraphlevel#1{%
        {\@sec@hook@setstyle\@sec@hook@refstepcounter\@sec@hook@prefix{#1}.}}


% paragraph-level sections
\newcommand\@secdef@nonum@paragraphlevel[7]{%
        \begingroup%
        \entity@paragraph@style=#1%
        \entity@paragraph@just=1%
        \entity@paragraph@span@font=0%
        \entity@paragraph@sbasedon=0\entity@paragraph@snext=0%
        \relax #5\relax%
        \stylesheet@style#1=#7; aka Normal;%
        \endgroup%
        \@namedef{#6}{%
        \def\@sec@hook@setstyle{\entity@paragraph@style=#1\relax}%
        \def\@sec@hook@refstepcounter{\relax}%
        \def\@sec@hook@prefix{\relax}%
        \@sec@hook@paragraphlevel}}

% {styleno}{justification}{space before}{space after}{attribs}{counter}{name}
% Consider first parameter as section level, second one is alignment (left, right,
% centered, justified... -- possible numbered values are listed in TeX2Word User
% Manual), then spacing before and after paragraph, then formatting (\bf goes for
% bold, \it for italics). Then goes name for counter and then name of that
% paragraph style as it is seen in MS Word.
\@secdef@nonum{1}{3}{0}{0}{\normalsize\rm\bf}{section}{Section}
\@secdef@nonum{2}{1}{0}{0}{\normalsize\rm\bf}{subsection}{Subsection}
\@addtoreset{subsection}{section}
\@secdef@nonum@paragraphlevel{3}{1}{0}{0}{\normalsize\rm\bf}{subsubsection}{Subsubsection}
\@addtoreset{subsubsection}{subsection}
\@secdef@nonum@paragraphlevel{4}{1}{0}{0}{\normalsize\rm\bf\it}{paragraph}{Paragraph}
\@addtoreset{paragraph}{subsubsection}
\@secdef@nonum@paragraphlevel{5}{1}{0}{0}{\normalsize\rm\it}{subparagraph}{Subparagraph}
\@addtoreset{subparagraph}{paragraph}

\newcommand\DeclareLanguageMapping[2]{}
\newcommand\addbibresource[1]{}
\newcommand\printbibliography{}

% convert citations to temporary citations for EndNote
\def\specialComma{,}
\def\specialSemicolon{;}

\catcode`\^^G=12
\newcommand*{\doachar}[1]{%
	\if#1\specialComma\specialSemicolon\else#1\fi%%
}
\newcommand*{\makeCommaIntoSemicolon}[1]{%
	\def\stuff{#1}\ifx\stuff\@empty\else\@llchars#1^^G\fi}
\def\@llchars#1#2^^G{%
	\def\letter{#1}\def\others{#2}%
	\ifx\letter\@empty\let\next\@gobble%
	\else%
		\doachar{#1}%
	\ifx\others\@empty \let\next\@gobble%
	\else \let\next\@llchars \fi%
	\fi%
\next#2^^G}
\catcode`\^^G=15

% handle bibliographic citations
\def\parencite{%
  \@ifnextchar[%
  {\@parencite}%
  {\@parencite[]}%
}
\def\@parencite[#1]{%
  \@ifnextchar[%
  {\@@parencite[#1]}%
  {\@@parencite[#1][]}%
}
\def\@@parencite[#1][#2]#3{%
  \makeatletter%
  {\{{{#1}\textbackslash}{\makeCommaIntoSemicolon{#3}}@{#2}\}}%
  \makeatother%
}

\let\Parencite=\parencite
\let\citep=\parencite
\let\Citep=\parencite
\let\citeyear=\parencite
\let\citeyearpar=\parencite


\def\textcite{%
  \@ifnextchar[%
  {\@textcite}%
  {\@textcite[]}%
}
\def\@textcite[#1]{%
  \@ifnextchar[%
  {\@@textcite[#1]}%
  {\@@textcite[#1][]}%
}
\def\@@textcite[#1][#2]#3{%
  \makeatletter%
  {\{ , , \makeCommaIntoSemicolon{#3}@{#2}@author-year\}}%
  \makeatother%
}
\let\Textcite=\textcite

\let\citet=\textcite
\let\Citet=\textcite

\let\citeA=\textcite

\let\citeauthor=\textcite
\let\Citeauthor=\textcite

\let\citeyearNP=\textcite
\let\citeNP=\textcite

\@ifundefined{apaSix@maskauthoridentity}{%  change masked references to unmasked

    % \maskcite
    \let\maskcite=\parencite

    \let\maskparencite=\maskcite
    \let\maskParencite=\maskparencite
    \let\maskcitep=\maskparencite
    \let\maskCitep=\maskparencite
    \let\maskciteyear=\maskparencite
    \let\maskciteyearpar=\maskparencite
    \let\masktextcite=\maskparencite
    \let\maskTextcite=\maskparencite
    \let\maskcitet=\maskparencite
    \let\maskCitet=\maskparencite
    \let\maskciteA=\maskparencite
    \let\maskciteauthor=\maskparencite
    \let\maskCiteauthor=\maskparencite
    \let\maskciteyearNP=\maskparencite
    \let\maskciteNP=\maskparencite

}{%  mask references to author

    % \maskcite
    \newcommand\maskcite{\@ifnextchar[{\maskcite@@also}{\maskcite@@also[]}}
    \newcommand\maskcite@@also{}
    \def\maskcite@@also[#1]{\@ifnextchar[{\maskcite@@@also[#1]}{\maskcite@@@also[][#1]}}

    \def\maskcite@@@also%
        [#1][#2]#3{%
            \def\apaSix@masked@refs{\it (citation(s) removed for masked review)}%
            {\apaSix@masked@refs}%
    }

    \let\maskparencite=\maskcite
    \let\maskParencite=\maskparencite
    \let\maskcitep=\maskparencite
    \let\maskCitep=\maskparencite
    \let\maskciteyear=\maskparencite
    \let\maskciteyearpar=\maskparencite

    \let\masktextcite=\maskparencite
    \let\maskTextcite=\maskparencite
    \let\maskcitet=\maskparencite
    \let\maskCitet=\maskparencite
    \let\maskciteA=\maskparencite
    \let\maskciteauthor=\maskparencite
    \let\maskCiteauthor=\maskparencite
    \let\maskciteyearNP=\maskparencite
    \let\maskciteNP=\maskparencite

}


% re-define the maketitle
\long\def\maketitle{\entity@paragraph@new\begingroup\entity@paragraph@just=1\relax%
{\raggedright\entity@paragraph@first=0\entity@paragraph@linespacingmultiple=0\entity@paragraph@spacebetween=240%
{\bf INSTRUCTIONS:}\par\entity@paragraph@first=0%
\begin{enumerate}%
\item Insert page numbers (non-italic) in top right corner\par%
\item Insert the following text on the left-hand side of the page header:\par%
  \begin{itemize}%
    \item Running head: \MakeUppercase{\@shorttitle}\par%
  \end{itemize}%
\item Adjust the vertical spacing on this title page to look appealing%
\item Verify formatting accuracy of bibliographic entries and References list%
\item Move each table and figure to its appropriate place at the end of the document%
  \begin{itemize}%
    \item {\it Note:} Captions are already formatted properly (above tables, below figures)
  \end{itemize}%
\item Format tables
  \begin{itemize}%
    \item {\it Hint:} Start by showing all borders, left-align, and make 100\% wide
    \item See the .pdf version of the \LaTeX\ file to see how the table should look
  \end{itemize}%
\item The final pages must appear in this order:%
  \begin{enumerate}%
    \item References
    \item Tables
    \item Figures
    \item Appendices
  \end{enumerate}%
\item Delete these instructions!%
\end{enumerate}}%
\@ifundefined{apaSix@maskauthoridentity}{%
	{\par\centering\entity@paragraph@first=0{\newline}\par}%
	{\centering\entity@paragraph@first=0%
		{\@title}\newline%
		{\@author}\newline%
		{\@affiliation}%
	        \newline{\@note}%
		\par%
		{%
			\@ifundefined{\@authornote}{}{%
				\par\entity@paragraph@first=0%
				\newline\newline\newline%
				{Author Note\par\par}%
			}%
		}%
	}%
	\@ifundefined{\@authornote}{}{%
	    {\entity@paragraph@first=720\@authornote}%
	}
}{%
	{\par\centering\entity@paragraph@first=0{\newline}\par}%
	{\centering\entity@paragraph@first=0%
		{\@title}\newline%
	        \newline{\@note}%
		\par%
	}%
    	{\entity@paragraph@first=720\par}
}%
\newpage%
{\centering\entity@paragraph@first=0{Abstract\par\par}}%
{\entity@paragraph@first=0\@abstract\par}%
{\entity@paragraph@first=720{\it Keywords:} \@keywords}%
\newpage%
{\entity@paragraph@new\begingroup\entity@paragraph@just=3\entity@paragraph@first=0\relax%
  {\@title}%
  \par\endgroup%
}%
\entity@paragraph@first=720%
\endgroup}

\entity@paragraph@just=1
\entity@paragraph@first=720%
\entity@paragraph@linespacingmultiple=1%
\entity@paragraph@spacebetween=480%
\stylesheet@style8=Normal; aka Normal;%


\ProcessOptions

\api@lockfile apa6

\@@endinput

% End of file

%</ptex>
%    \end{macrocode}
%\end{macro}
%
%
%
% %%%%%%%%%%%  TeX2WordForapa6.bas FILE (macro for Word)  %%%%%%%%%%%%
%
%\begin{macro}{TeX2WordForapa6.bas}
%    \begin{macrocode}
%<*bas>
Attribute VB_Name = "TeX2WordForapa6"
Sub FormatTex2WordDocument()

    Dim strRunningHead, r, E
    Dim myrange As Range
    
    strRunningHead = InputBox("Please type the running head:", "Running Head")
    If strRunningHead = "" Then Exit Sub
    
    Selection.EndKey Unit:=wdStory
    Selection.InsertBreak Type:=wdPageBreak
    Selection.TypeText Text:="References" & vbCrLf
    Selection.MoveUp Unit:=wdLine, Count:=1
    Selection.ParagraphFormat.LeftIndent = InchesToPoints(0)
    Selection.ParagraphFormat.Alignment = wdAlignParagraphCenter

    Selection.HomeKey Unit:=wdStory

    Call FormatTex2WordPageHeader(strRunningHead)
    Call FormatAndMoveTex2WordTables
    Call FormatAndMoveTex2WordFigures

    Selection.EndKey Unit:=wdStory
    Selection.InsertBreak Type:=wdPageBreak
    Selection.TypeText Text:="Appendix" & vbCrLf
    Selection.MoveUp Unit:=wdLine, Count:=1
    Selection.ParagraphFormat.LeftIndent = InchesToPoints(0)
    Selection.ParagraphFormat.Alignment = wdAlignParagraphCenter

    ' touch up temporary citations
    Selection.HomeKey Unit:=wdStory

    Selection.Find.ClearFormatting
    Selection.Find.Replacement.ClearFormatting
    With Selection.Find
        .Text = "{\"
        .Replacement.Text = "{"
        .Forward = True
        .Wrap = wdFindContinue
        .Format = False
        .MatchCase = False
        .MatchWholeWord = False
        .MatchWildcards = False
        .MatchSoundsLike = False
        .MatchAllWordForms = False
    End With
    Selection.Find.Execute Replace:=wdReplaceAll
    
    With Selection.Find
        .Text = "@}"
        .Replacement.Text = "}"
        .Forward = True
        .Wrap = wdFindContinue
        .Format = False
        .MatchCase = False
        .MatchWholeWord = False
        .MatchWildcards = False
        .MatchSoundsLike = False
        .MatchAllWordForms = False
    End With
    Selection.Find.Execute Replace:=wdReplaceAll

    With Selection.Find
        .Text = "{e.g.,\"
        .Replacement.Text = "{e.g.`, \"
        .Forward = True
        .Wrap = wdFindContinue
        .Format = False
        .MatchCase = False
        .MatchWholeWord = False
        .MatchWildcards = False
        .MatchSoundsLike = False
        .MatchAllWordForms = False
    End With
    Selection.Find.Execute Replace:=wdReplaceAll

    With Selection.Find
        .Text = "{i.e.,\"
        .Replacement.Text = "{i.e.`, \"
        .Forward = True
        .Wrap = wdFindContinue
        .Format = False
        .MatchCase = False
        .MatchWholeWord = False
        .MatchWildcards = False
        .MatchSoundsLike = False
        .MatchAllWordForms = False
    End With
    Selection.Find.Execute Replace:=wdReplaceAll

    With Selection.Find
        .Text = "{cf.\"
        .Replacement.Text = "{cf. \"
        .Forward = True
        .Wrap = wdFindContinue
        .Format = False
        .MatchCase = False
        .MatchWholeWord = False
        .MatchWildcards = False
        .MatchSoundsLike = False
        .MatchAllWordForms = False
    End With
    Selection.Find.Execute Replace:=wdReplaceAll

    With Selection.Find
        .Text = "@p. "
        .Replacement.Text = "@"
        .Forward = True
        .Wrap = wdFindContinue
        .Format = False
        .MatchCase = False
        .MatchWholeWord = False
        .MatchWildcards = False
        .MatchSoundsLike = False
        .MatchAllWordForms = False
    End With
    Selection.Find.Execute Replace:=wdReplaceAll

    With Selection.Find
        .Text = "@pp. "
        .Replacement.Text = "@"
        .Forward = True
        .Wrap = wdFindContinue
        .Format = False
        .MatchCase = False
        .MatchWholeWord = False
        .MatchWildcards = False
        .MatchSoundsLike = False
        .MatchAllWordForms = False
    End With
    Selection.Find.Execute Replace:=wdReplaceAll

    ' for table footnotes
    With Selection.Find
        .Text = "[para,flushleft] "
        .Replacement.Text = ""
        .Forward = True
        .Wrap = wdFindContinue
        .Format = False
        .MatchCase = False
        .MatchWholeWord = False
        .MatchWildcards = False
        .MatchSoundsLike = False
        .MatchAllWordForms = False
    End With
    Selection.Find.Execute Replace:=wdReplaceAll

    ' remove extra spaces at the end of paragraphs
    With Selection.Find
        .Text = " ^p"
        .Replacement.Text = "^p"
        .Forward = True
        .Wrap = wdFindContinue
        .Format = False
        .MatchCase = False
        .MatchWholeWord = False
        .MatchWildcards = False
        .MatchSoundsLike = False
        .MatchAllWordForms = False
    End With
    Selection.Find.Execute Replace:=wdReplaceAll

   ' delete the instructions
    Selection.HomeKey Unit:=wdStory
    
    Set myrange = Selection.Range
    myrange.Start = Selection.Start
     
    Selection.HomeKey Unit:=wdStory
    Selection.Find.ClearFormatting
    With Selection.Find
        .Execute findText:="Delete these instructions!", Forward:=True, Wrap:=wdFindStop
        myrange.End = Selection.End + 1
        myrange.Select
        myrange.Delete
    End With
    
    Selection.HomeKey Unit:=wdStory

End Sub

Sub FormatTex2WordPageHeader(strRunningHead)
    
    Dim r, E
    
    ' set the first page to be a different header
    ActiveDocument.PageSetup.DifferentFirstPageHeaderFooter = True
    
    With ActiveDocument.Sections(1).Headers(wdHeaderFooterFirstPage)
        .Range.ParagraphFormat.LineSpacingRule = wdLineSpaceDouble
        .Range.ParagraphFormat.FirstLineIndent = InchesToPoints(0)
        .Range.ParagraphFormat.TabStops.ClearAll
        .Range.ParagraphFormat.TabStops.Add Position:=InchesToPoints(6.5), _
            Alignment:=wdAlignTabRight, Leader:=wdTabLeaderSpaces
        .Range.Text = "Running head: " & UCase(strRunningHead) & vbTab
        With .Range.Font
            .Name = "Times New Roman"
            .Size = 12
            .Bold = False
            .Italic = False
        End With
        
        Set r = .Range
        E = .Range.End
        r.Start = E
        .Range.Fields.Add r, wdFieldPage
    
    End With
    
    With ActiveDocument.Sections(1).Headers(wdHeaderFooterPrimary)
        .Range.ParagraphFormat.LineSpacingRule = wdLineSpaceDouble
        .Range.ParagraphFormat.FirstLineIndent = InchesToPoints(0)
        .Range.Text = UCase(strRunningHead)
        .PageNumbers.Add PageNumberAlignment:=wdAlignPageNumberRight
        With .Range.Font
            .Name = "Times New Roman"
            .Size = 12
            .Bold = False
            .Italic = False
        End With
    End With
    
End Sub

Sub FormatAndMoveTex2WordTables()
    
    Dim i, rngParagraphs As Range
    
    If ActiveDocument.Tables.Count > 0 Then
    
        For i = 1 To ActiveDocument.Tables.Count
        
            Selection.HomeKey Unit:=wdStory
            
            Set rngParagraphs = ActiveDocument.Range( _
                    Start:=ActiveDocument.Tables(1).Range.Start, _
                    End:=ActiveDocument.Tables(1).Range.End)
            rngParagraphs.Select
            
            With Selection.Tables(1)
                .Borders(wdBorderLeft).LineStyle = wdLineStyleNone
                .Borders(wdBorderRight).LineStyle = wdLineStyleNone
                With .Borders(wdBorderTop)
                    .LineStyle = wdLineStyleSingle
                    .LineWidth = wdLineWidth050pt
                    .Color = wdColorAutomatic
                End With
                With .Borders(wdBorderBottom)
                    .LineStyle = wdLineStyleSingle
                    .LineWidth = wdLineWidth050pt
                    .Color = wdColorAutomatic
                End With
                .Borders(wdBorderHorizontal).LineStyle = wdLineStyleNone
                .Borders(wdBorderVertical).LineStyle = wdLineStyleNone
                .Borders(wdBorderDiagonalDown).LineStyle = wdLineStyleNone
                .Borders(wdBorderDiagonalUp).LineStyle = wdLineStyleNone
                .Borders.Shadow = False
                '.Rows.Alignment = wdAlignRowLeft
                .PreferredWidthType = wdPreferredWidthPercent
                .PreferredWidth = 100
                .TopPadding = InchesToPoints(0.08)
                .BottomPadding = InchesToPoints(0.08)
                .LeftPadding = InchesToPoints(0.08)
                .RightPadding = InchesToPoints(0.08)
                .Spacing = 0
                .AllowPageBreaks = True
                .AllowAutoFit = False
            End With
            
            With Selection.ParagraphFormat
                .LineSpacingRule = wdLineSpaceSingle
                .LeftIndent = InchesToPoints(0)
                .RightIndent = InchesToPoints(0)
                .SpaceBefore = 0
                .SpaceBeforeAuto = False
                .SpaceAfter = 0
                .SpaceAfterAuto = False
                .WidowControl = False
                .KeepWithNext = False
                .KeepTogether = False
                .PageBreakBefore = False
                .NoLineNumber = False
                .Hyphenation = True
                .FirstLineIndent = InchesToPoints(0)
                .CharacterUnitLeftIndent = 0
                .CharacterUnitRightIndent = 0
                .CharacterUnitFirstLineIndent = 0
                .LineUnitBefore = 0
                .LineUnitAfter = 0
                .MirrorIndents = False
                .TextboxTightWrap = wdTightNone
            End With
            
            rngParagraphs.Cut
        
            Selection.EndKey Unit:=wdStory
            Selection.InsertBreak Type:=wdPageBreak
            Selection.TypeText Text:="Table " & i & vbCrLf
            Selection.Paste
        
        Next
    
    End If
            
End Sub

Sub FormatAndMoveTex2WordFigures()
    
    Dim i, rngParagraphs As Range
    
    If ActiveDocument.InlineShapes.Count > 0 Then
    
        For i = 1 To ActiveDocument.InlineShapes.Count
        
            Selection.HomeKey Unit:=wdStory
            
            Set rngParagraphs = ActiveDocument.Range( _
                    Start:=ActiveDocument.InlineShapes(1).Range.Start, _
                    End:=ActiveDocument.InlineShapes(1).Range.End)
            rngParagraphs.Select
            rngParagraphs.Cut
        
            Selection.EndKey Unit:=wdStory
            Selection.InsertBreak Type:=wdPageBreak
            Selection.Paste
            Selection.TypeText Text:=vbCrLf & "Figure " & i & vbCrLf
        
        Next
    
    End If
            
End Sub
%
%</bas>
%    \end{macrocode}
%\end{macro}
%
%
% %%%%%%%%%%%  CONFIGURATION FILES  %%%%%%%%%%%%
%
%\begin{macro}{APAamerican.txt}
%    \begin{macrocode}
%<*american>
\ProvidesFile{APAamerican.txt}[2012/02/23 v1.25 apa6 configuration for American English]

\renewcommand{\rheadname}{Running head}
\renewcommand{\acksname}{Author Note}
\renewcommand{\keywordname}{Keywords}
\renewcommand{\notesname}{Footnotes}


%</american>
%    \end{macrocode}
%\end{macro}
%
%
%\begin{macro}{APAbritish.txt}
%    \begin{macrocode}
%<*british>
\ProvidesFile{APAbritish.txt}[2012/02/23 v1.25 apa6 configuration for British English]

\renewcommand{\rheadname}{Running head}
\renewcommand{\acksname}{Author Note}
\renewcommand{\keywordname}{Keywords}
\renewcommand{\notesname}{Footnotes}


%</british>
%    \end{macrocode}
%\end{macro}
%
%
%\begin{macro}{APAenglish.txt}
%    \begin{macrocode}
%<*english>
\ProvidesFile{APAenglish.txt}[2012/02/23 v1.25 apa6 configuration for non-American English]

\renewcommand{\rheadname}{Running head}
\renewcommand{\acksname}{Author Note}
\renewcommand{\keywordname}{Keywords}
\renewcommand{\notesname}{Footnotes}

%</english>
%    \end{macrocode}
%\end{macro}
%
%\begin{macro}{APAdutch.txt}
%    \begin{macrocode}
%<*dutch>
\ProvidesFile{APAdutch.txt}[2012/02/23 v1.25 apa6 configuration for Dutch]

\renewcommand{\rheadname}{Kopregel}% Running head
\renewcommand{\acksname}{Auteursnoot}% Author Note
\renewcommand{\keywordname}{Trefwoorden}% Keywords
\renewcommand{\notesname}{Voetnoten}% Footnotes

%</dutch>
%    \end{macrocode}
%\end{macro}
%
%\begin{macro}{APAgerman.txt}
%    \begin{macrocode}
%<*german>
\ProvidesFile{APAgerman.txt}[2012/02/23 v1.25 apa6 configuration for German]

\renewcommand{\rheadname}{Kolumnentitel}% Running head
\renewcommand{\acksname}{Autorenhinweis}% Author Note
\renewcommand{\keywordname}{Schl\"usselw\"orter}% Keywords
\renewcommand{\notesname}{Fußnoten}% Footnotes

%</german>
%    \end{macrocode}
%\end{macro}
%
%\begin{macro}{APAngerman.txt}
%    \begin{macrocode}
%<*ngerman>
\ProvidesFile{APAgerman.txt}[2012/02/23 v1.25 apa6 configuration for "new" German]

\renewcommand{\rheadname}{Kolumnentitel}% Running head
\renewcommand{\acksname}{Autorenhinweis}% Author Note
\renewcommand{\keywordname}{Schl\"usselw\"orter}% Keywords
\renewcommand{\notesname}{Fußnoten}% Footnotes

%</ngerman>
%    \end{macrocode}
%\end{macro}
%
%\begin{macro}{APAgreek.txt}
%    \begin{macrocode}
%<*greek>
\ProvidesFile{APAgreek.txt}[2012/02/23 v1.25 apa6 configuration for Greek]

\renewcommand{\rheadname}{Up'ertitlos}% Running head
\renewcommand{\acksname}{Shmei'wseis tou suggraf'ea}% Author Note
\renewcommand{\keywordname}{L'exeis kleidi'a}% Keywords
\renewcommand{\notesname}{Uposhmei'wseis}% Footnotes

%</greek>
%    \end{macrocode}
%\end{macro}
%
%\begin{macro}{APAczech.txt}
%    \begin{macrocode}
%<*czech>
\ProvidesFile{APAczech.txt}[2012/02/23 v1.25 apa6 configuration for Czech]

\renewcommand{\rheadname}{Záhlaví}% Running head
\renewcommand{\acksname}{Poznámka~autora}% Author Note
\renewcommand{\keywordname}{Klícová~1slova}% Keywords
\renewcommand{\notesname}{Poznámky}% Footnotes

%</czech>
%    \end{macrocode}
%\end{macro}
%
%\begin{macro}{APAturkish.txt}
%    \begin{macrocode}
%<*turkish>
\ProvidesFile{APAturkish.txt}[2016/05/26 v2.2 apa6 configuration for Turkish]

\renewcommand{\rheadname}{Kısa Başlık}% Running head
\renewcommand{\acksname}{Yazar Notu}% Author Note
\renewcommand{\keywordname}{Anahtar Kelimeler}% Keywords
\renewcommand{\notesname}{Dipnotlar}% Footnotes

%</turkish>
%    \end{macrocode}
%\end{macro}
%
%\begin{macro}{APAendfloat.cfg}
%    \begin{macrocode}
%<*APAendfloat>
%%
%% This is file `endfloat.cfg',
%% modifed from the original supplied with the endfloat package
%% to handle both sideways floats and longtable
%%
%% Athanassios Protopapas <protopap@ilsp.gr>
%% July 2005
%%
%% Original authors: James Darrell McCauley <jdm5548@diamond.tamu.edu>,
%% Jeff Goldberg <j.goldberg@cranfield.ac.uk>
%% Original version: Version 2.4i <October 1995>
%%
\RequirePackage{rotating}
\let\efsaved@sidewaysfigure\sidewaysfigure
\let\efsaved@sidewaystable\sidewaystable
\let\efsaved@longtable\longtable
\AtBeginTables{\let\sidewaystable=\efsaved@sidewaystable\relax}
\AtBeginTables{\let\longtable=\efsaved@longtable\relax}
\AtBeginFigures{\let\sidewaysfigure=\efsaved@sidewaysfigure\relax}
\def\sidewaystable{\efloat@condopen{ttt}
    \efloat@iwrite{ttt}{\string\begin{sidewaystable}}%
%    \if@domarkers
%       \addtocounter{posttbl}{1}
%       \tableplace
%    \fi
    \def\@currenvir{efloat@float}%
    \begingroup
    \let\do\ef@makeinnocent \dospecials
    \ef@makeinnocent\^^L% and whatever other special cases
    \endlinechar`\^^M \catcode`\^^M=12 \ef@xsidetable}
{\catcode`\^^M=12 \endlinechar=-1 %
 \gdef\ef@xsidetable#1^^M{\def\test{#1}
      \ifx\test\ef@endsidetabletest
          \efloat@foundend{ttt}{sidewaystable}
      \else
          \efloat@iwrite{ttt}{#1}%
          \let\next\ef@xsidetable
      \fi \next}
}
\def\sidewaysfigure{\efloat@condopen{fff}
    \efloat@iwrite{fff}{\string\begin{sidewaysfigure}}%
%    \if@domarkers
%       \addtocounter{postfig}{1}
%       \figureplace
%    \fi
    \def\@currenvir{efloat@float}%
    \begingroup
    \let\do\ef@makeinnocent \dospecials
    \ef@makeinnocent\^^L% and whatever other special cases
    \endlinechar`\^^M \catcode`\^^M=12 \ef@xsidefigure}
{\catcode`\^^M=12 \endlinechar=-1 %
 \gdef\ef@xsidefigure#1^^M{\def\test{#1}
      \ifx\test\ef@endsidefiguretest
          \efloat@foundend{fff}{sidewaysfigure}
      \else
          \efloat@iwrite{fff}{#1}%
          \let\next\ef@xsidefigure
      \fi \next}
}
\def\longtable{\efloat@condopen{ttt}
    \efloat@iwrite{ttt}{\string\begin{longtable}}%
%    \if@domarkers
%       \addtocounter{posttbl}{1}
%       \tableplace
%    \fi
    \def\@currenvir{efloat@float}%
    \begingroup
    \let\do\ef@makeinnocent \dospecials
    \ef@makeinnocent\^^L% and whatever other special cases
    \endlinechar`\^^M \catcode`\^^M=12 \ef@xlongtable}
{\catcode`\^^M=12 \endlinechar=-1 %
 \gdef\ef@xlongtable#1^^M{\def\test{#1}
      \ifx\test\ef@endlongtabletest
          \efloat@foundend{ttt}{longtable}
      \else
          \efloat@iwrite{ttt}{#1}%
          \let\next\ef@xlongtable
      \fi \next}
}
{\escapechar=-1%
 \xdef\ef@endsidefiguretest{\string\\end\string\{sidewaysfigure\string\}}%
 \xdef\ef@endsidetabletest{\string\\end\string\{sidewaystable\string\}}
 \xdef\ef@endlongtabletest{\string\\end\string\{longtable\string\}}}%
\endinput

%</APAendfloat>
%    \end{macrocode}
%\end{macro}
%
%
%\Finale
