% Created 2020-01-13 Mon 15:16
% Intended LaTeX compiler: pdflatex
\documentclass[11pt]{article}
\usepackage[utf8]{inputenc}
\usepackage[T1]{fontenc}
\usepackage{graphicx}
\usepackage{grffile}
\usepackage{longtable}
\usepackage{wrapfig}
\usepackage{rotating}
\usepackage[normalem]{ulem}
\usepackage{amsmath}
\usepackage{textcomp}
\usepackage{amssymb}
\usepackage{capt-of}
\usepackage{hyperref}
\usepackage{minted}
\author{Tigany Zarrouk}
\date{\today}
\title{Final Ti Model Results}
\hypersetup{
 pdfauthor={Tigany Zarrouk},
 pdftitle={Final Ti Model Results},
 pdfkeywords={},
 pdfsubject={},
 pdfcreator={Emacs 26.3 (Org mode 9.1.9)}, 
 pdflang={English}}
\begin{document}

\maketitle
\tableofcontents



\section{Objective Function}
\label{sec:orgd6bab5b}


PARAMETERS
  fdd=0.1958363809 qdds=0.5591275855 qddp=0.5690351902 qddd=0.7745947522 b0=58.0906936439 p0=1.2185323579 b1=-3.2299188646 p1=0.6862915307 b2=593519.1134129359 m2=-11.5000000000 p2=0.0000000000 ndt=2.0000000000 cr1=-6.0000000000 cr2=3.0474400934 cr3=-1.2317472193 r1dd=6.5000000000 rcdd=10.0000000000 rmaxhm=10.1000000000 npar=18 
VARGS
    -vfdd=0.1958363809 -vqdds=0.5591275855 -vqddp=0.5690351902 -vqddd=0.7745947522 -vb0=58.0906936439 -vp0=1.2185323579 -vb1=-3.2299188646 -vp1=0.6862915307 -vb2=593519.1134129359 -vm2=-11.5000000000 -vp2=0.0000000000 -vndt=2.0000000000 -vcr1=-6.0000000000 -vcr2=3.0474400934 -vcr3=-1.2317472193 -vr1dd=6.5000000000 -vrcdd=10.0000000000 -vrmaxhm=10.1000000000 



\begin{center}
\begin{tabular}{lrr}
Quantity & From Model & Target\\
\hline
a\(_{\text{hcp}}\) & 5.58523112 & 5.57678969\\
c/a & 1.58371266 & 1.58731122\\
a\(_{\text{omega}}\) & 8.93475285 & 8.73254342\\
c\(_{\text{omega}}\) & 5.38726911 & 5.32343103\\
a\(_{\text{4h}}\) & 5.57584691 & 5.56325146\\
c\(_{\text{4h}}\) & 18.09810672 & 17.75908031\\
a\(_{\text{6h}}\) & 5.57365569 & 5.54639384\\
c\(_{\text{6h}}\) & 27.18378460 & 26.77136353\\
a\(_{\text{bcc}}\) & 6.20079768 & 6.17948863\\
a\(_{\text{fcc}}\) & 7.87290654 & 7.88677000\\
DE(o,h) & 0.58764167 & -0.63343333\\
DE(4h,h) & 1.58019500 & 3.17160000\\
DE(6h,h) & 2.48264833 & 3.72005000\\
DE(b,h) & 5.35128500 & 7.63520000\\
DE(f,h) & 3.78088500 & 4.51880000\\
c\(_{\text{11}}\) & 171.60928873 & 176.10000000\\
c\(_{\text{33}}\) & 198.90063708 & 190.50000000\\
c\(_{\text{44}}\) & 47.42549704 & 50.80000000\\
c\(_{\text{12}}\) & 94.65941969 & 86.90000000\\
c\(_{\text{13}}\) & 61.22624060 & 68.30000000\\
M\(_{\text{freq}}\)\(_{\text{0}}\) & 2.59341377 & 2.85858719\\
M\(_{\text{freq}}\)\(_{\text{1}}\) & 2.59341378 & 2.85858719\\
M\(_{\text{freq}}\)\(_{\text{2}}\) & 2.59341378 & 2.85858719\\
M\(_{\text{freq}}\)\(_{\text{3}}\) & 2.59341379 & 2.85858719\\
M\(_{\text{freq}}\)\(_{\text{4}}\) & 5.85272461 & 5.66706047\\
M\(_{\text{freq}}\)\(_{\text{5}}\) & 5.85272461 & 5.66706047\\
H\(_{\text{freq}}\)\(_{\text{0}}\) & 3.82320403 & 4.80643423\\
H\(_{\text{freq}}\)\(_{\text{1}}\) & 3.82320403 & 5.58010025\\
H\(_{\text{freq}}\)\(_{\text{2}}\) & 6.40288977 & 5.65316738\\
H\(_{\text{freq}}\)\(_{\text{3}}\) & 6.40288977 & 6.36651842\\
H\(_{\text{freq}}\)\(_{\text{4}}\) & 7.92857431 & 6.40050186\\
H\(_{\text{freq}}\)\(_{\text{5}}\) & 7.92857431 & 7.64082373\\
bandw.  G & 3.69394702 & 5.87085872\\
bandw.  K & 4.65178817 & 4.97424321\\
bandw.  M & 5.19329495 & 7.78109872\\
bandw.  L & 4.21232412 & 6.34433701\\
bandw.  H & 3.54700549 & 9.70902614\\
DOSerr\(_{\text{h}}\) & 0.00000000 & 0.00000000\\
DOSerr\(_{\text{o}}\) & 0.00000000 & 0.00000000\\
E\(_{\text{pris}}\)\(_{\text{f}}\) & 98.95340236 & 220.00000000\\
\end{tabular}
\end{center}



----------     E\(_{\text{prismatic}}\)\(_{\text{fault}}\)     -----------

\begin{center}
\begin{tabular}{lrll}
tbe: & 98.953 & mJ/m\(^{\text{2}}\) & \\
DFT: & 250.000 & mJ/m\(^{\text{2}}\) & [Benoit  2012]\\
DFT: & 233.000 & mJ/m\(^{\text{2}}\) & [Ackland 1999]\\
\end{tabular}
\end{center}


----------     E\(_{\text{Basal}}\)\(_{\text{fault}}\) I2     -----------

\begin{center}
\begin{tabular}{lrll}
tbe: & 211.658 & mJ/m\(^{\text{2}}\) & \\
DFT: & 260.000 & mJ/m\(^{\text{2}}\) & [Benoit  2012]\\
\end{tabular}
\end{center}

\section{Defect Clusters}
\label{sec:org192b232}

----------     E\(_{\text{vacancy}}\)\(_{\text{formation}}\)     ----------

\begin{center}
\begin{tabular}{lll}
tbe: & 2.347  eV & \\
DFT: & 1.950  eV & GGA-PAW:   Angsten  (2013)\\
exp: & 1.270  eV & Hashimoto  (1984)\\
\end{tabular}
\end{center}

\subsection{Octahedral O interstitial relaxation}
\label{sec:orgf61526e}

Initial:
\begin{center}
\includegraphics[width=.9\linewidth]{Images/initial_octahedral_ox_ovito.png}
\end{center}

Final:
\begin{center}
\includegraphics[width=.9\linewidth]{Images/final_octahedral_ox_ovito.png}
\end{center}

\subsection{Tetrahedral O interstitial relaxation}
\label{sec:org7a0fc23}

Initial:
\begin{center}
\includegraphics[width=.9\linewidth]{Images/final_model_final_tetra_ox.png}
\end{center}

Final:
\begin{center}
\includegraphics[width=.9\linewidth]{Images/final_model_initial_tetra_ox_ovito.png}
\end{center}

\subsection{Energies for defects}
\label{sec:orge0c6d5c}

Relative differences are 

>> (E\(_{\text{tetrahedral}}\) - E\(_{\text{octahedral}}\)) 
\begin{center}
\begin{tabular}{lll}
tbe: & 1.65 eV & \\
GGA-DFT: & 1.23 eV & Kwasniak (2013)\\
\end{tabular}
\end{center}

>> (E\(_{\text{hexahedral}}\) - E\(_{\text{octahedral}}\))
\begin{center}
\begin{tabular}{ll}
tbe: & 0.90 eV\\
\end{tabular}
\end{center}

> Note: Preference for tetrahedral oxygen to go into hexahedral site as seen by images above

All formation energies below use the chemical potential of Akysonov
(2013) of value \(\mu_{\text{oxygen}} = \frac{5.6}{ 2} eV\).

\subsection{All formation energies}
\label{sec:orgcfbb118}

\begin{center}
\begin{tabular}{ll}
Quantity & Energy (eV)\\
\hline
Ef\(_{\text{Vf}}\) & 2.347\\
 & \\
Ef\(_{\text{T}}\)\(_{\text{sol}}\) & -  21.783\\
Ef\(_{\text{T}}\)\(_{\text{dil}}\)\(_{\text{imp}}\) & -  28.991\\
Ef\(_{\text{T}}\)\(_{\text{formation}}\) & -  21.783\\
Ef\(_{\text{T}}\)\(_{\text{V}}\)\(_{\text{formation}}\) & -  18.905\\
Ef\(_{\text{T}}\)\(_{\text{vac}}\)\(_{\text{sol}}\)\(_{\text{bind}}\) & -   0.530\\
 & \\
Ef\(_{\text{O}}\)\(_{\text{sol}}\) & -  23.436\\
Ef\(_{\text{O}}\)\(_{\text{dil}}\)\(_{\text{imp}}\) & -  30.645\\
Ef\(_{\text{O}}\)\(_{\text{formation}}\) & -  23.436\\
Ef\(_{\text{O}}\)\(_{\text{V}}\)\(_{\text{formation}}\) & -  18.905\\
Ef\(_{\text{O}}\)\(_{\text{vac}}\)\(_{\text{sol}}\)\(_{\text{bind}}\) & -   2.183\\
 & \\
Ef\(_{\text{OO}}\)\(_{\text{sol}}\) & -  49.606\\
Ef\(_{\text{OO}}\)\(_{\text{dil}}\)\(_{\text{imp}}\) & -  56.814\\
Ef\(_{\text{OO}}\)\(_{\text{formation}}\) & -  46.806\\
Ef\(_{\text{OO}}\)\(_{\text{V}}\)\(_{\text{formation}}\) & -  41.910\\
Ef\(_{\text{OO}}\)\(_{\text{vac}}\)\(_{\text{sol}}\)\(_{\text{bind}}\) & -   2.547\\
 & \\
Ef\(_{\text{OOO}}\)\(_{\text{sol}}\) & -  76.037\\
Ef\(_{\text{OOO}}\)\(_{\text{dil}}\)\(_{\text{imp}}\) & -  83.246\\
Ef\(_{\text{OOO}}\)\(_{\text{formation}}\) & -  70.437\\
Ef\(_{\text{OOO}}\)\(_{\text{V}}\)\(_{\text{formation}}\) & -  66.013\\
Ef\(_{\text{OOO}}\)\(_{\text{vac}}\)\(_{\text{sol}}\)\(_{\text{bind}}\) & -   2.076\\
 & \\
Ef\(_{\text{OOOO}}\)\(_{\text{sol}}\) & - 102.470\\
Ef\(_{\text{OOOO}}\)\(_{\text{dil}}\)\(_{\text{imp}}\) & - 109.679\\
Ef\(_{\text{OOOO}}\)\(_{\text{formation}}\) & -  94.070\\
Ef\(_{\text{OOOO}}\)\(_{\text{V}}\)\(_{\text{formation}}\) & -  88.998\\
Ef\(_{\text{OOOO}}\)\(_{\text{vac}}\)\(_{\text{sol}}\)\(_{\text{bind}}\) & -  2.724\\
 & \\
Ef\(_{\text{OOOOO}}\)\(_{\text{sol}}\) & - 128.781\\
Ef\(_{\text{OOOOO}}\)\(_{\text{dil}}\)\(_{\text{imp}}\) & - 135.989\\
Ef\(_{\text{OOOOO}}\)\(_{\text{formation}}\) & - 117.581\\
Ef\(_{\text{OOOOO}}\)\(_{\text{V}}\)\(_{\text{formation}}\) & - 113.649\\
Ef\(_{\text{OOOOO}}\)\(_{\text{vac}}\)\(_{\text{sol}}\)\(_{\text{bind}}\) & - 1.583\\
 & \\
Ef\(_{\text{OOOOOO}}\)\(_{\text{sol}}\) & - 155.148\\
Ef\(_{\text{OOOOOO}}\)\(_{\text{dil}}\)\(_{\text{imp}}\) & - 162.357\\
Ef\(_{\text{OOOOOO}}\)\(_{\text{formation}}\) & - 141.148\\
Ef\(_{\text{OOOOOO}}\)\(_{\text{V}}\)\(_{\text{formation}}\) & - 137.110\\
Ef\(_{\text{OOOOOO}}\)\(_{\text{vac}}\)\(_{\text{sol}}\)\(_{\text{bind}}\) & - 1.690\\
\end{tabular}
\end{center}

\section{Gamma surfaces}
\label{sec:org6bad779}

Energies are accurate to within 2 mJm\(^{\text{-2}}\), comparing the energies of
points in the corners which (the zeros of energy). So surface energies
might be  \(\pm 2\) mJm\(^{\text{-2}}\) off which is reasonable. 

These calculations were done in tight binding with 15 layers for both
basal and prismatic with k-points adjusted accordingly. 
DFT comparisons are usind results of Rodney. 

The Pyramidal surface was obtained using the same 32 atom cell that
Ready used in his paper on the pyramidal gamma surface with DFT
pseudopotentials. 

\newpage
\subsection{Basal}
\label{sec:org1106ab7}

TBE:
\begin{center}
\includegraphics[width=.9\linewidth]{Images/basal_gs_noo_2019-11-08_alat.png}
\end{center}

DFT:
\begin{center}
\includegraphics[width=.9\linewidth]{Images/rodney_basal_ti_gamma_surface.png}
\end{center}

\subsection{Prismatic}
\label{sec:orgc335522}

TBE:
\begin{center}
\includegraphics[width=.9\linewidth]{Images/prismatic_gs_noo_2019-11-08_alat.png}
\end{center}

DFT:
\begin{center}
\includegraphics[width=.9\linewidth]{Images/rodney_prismatic_ti_gamma_surface.png}
\end{center}

\subsection{Pyramidal first order}
\label{sec:org1525fff}

TBE:
\begin{center}
\includegraphics[width=.9\linewidth]{Images/pyramidal_gs_noo_2019-11-08_alat.png}
\end{center}

DFT pseudopot:
\begin{center}
\includegraphics[width=.9\linewidth]{Images/pyramidal_gamma_surface_ready_data_both.png}
\end{center}

\subsection{Data}
\label{sec:org666acc7}
\href{file:///home/tigany/Documents/ti/final\_model\_2019-11-12/results\_2019-11-09\_muc/gamma\_surfaces/basal/basal\_gs\_noo\_alat\_energies.dat}{basal\(_{\text{gs}}\)\(_{\text{data}}\)}
\href{file:///home/tigany/Documents/ti/final\_model\_2019-11-12/results\_2019-11-09\_muc/gamma\_surfaces/prismatic/prismatic\_gs\_noo\_alat\_energies.dat}{prismatic\(_{\text{gs}}\)\(_{\text{data}}\)}
\href{file:///home/tigany/Documents/ti/final\_model\_2019-11-12/gamma\_surfaces/pyramidal\_results\_2019-11-13/pyramidal\_gamma\_surface\_2019-11-13.dat}{pyramidal\(_{\text{gs}}\)\(_{\text{data}}\)}
\section{Dislocation core structures}
\label{sec:orgd06199e}

\subsection{Methodology}
\label{sec:org54616a7}
In the following, we see results of dislocation relaxation. The
partial differential displacement maps are of dislocations in
their initial and final states in different initial positions. The
atoms were relaxed until the root-mean square force acting on each
atom was less than \(1\times 10^-5\) Ryd/Bohr.

These relaxations can be distinguished by the different initial
positions of the dislocation centre (elastic centre) as following
the paper by Tarrat \cite{Tarrat2009}.

The partial burger's vector seen here is the \(1/6 [11\bar{2}0]\)
dislocation, and this is the Burger's vector plotted on each of
hte initial positions. 

\subsection{Discussion}
\label{sec:org7fd4825}
One can see that all of the dislocations have dissociated on the
prismatic plane. But there is a difference between initial
positions as to upon which prismatic plane they dissociate on,
from the original. 

Only initial position 2 actually dissociated on a different
prismatic plane to the others. 

The positions of the partials are also different once each of the
separate initial positions have been relaxed. 


IP2 and IP3, although they are on different planes, have a very
similar core structure to each other. They are both asymmetric
cores. 


IP1 has the upper partial dislocation located within an adjacent
triangle to the left, compared to IP2 and IP3. The lower partial
has been shifted downwards, by one triangle down and to the right,
with respect to IP3. The core structure of IP5 is
indistinguishable from IP1. These cores can be deemed as
metastable, as they have a slightly higher energy than the other
cores.


The upper partial of IP4 has been displaced upwards by one Peierls
valley with respect to IP3. The lower partial is in the same
triangle as IP3. IP4 is a mirrored core. 


Each of these cores are asymmetric, using the definition by Tarrat
\cite{Tarrat2009}. 

The energies for each of the dislocation cores, when relaxed to
\(1\times 10^{-5}\) Ryd/Bohr is 

\begin{center}
\begin{tabular}{rr}
Initial position & E\(_{\text{total}}\)\\
\hline
1 & -331.54658899\\
2 & -331.54660063\\
3 & -331.54660053\\
4 & -331.54660061\\
5 & -331.54658717\\
\end{tabular}
\end{center}

The 



The dissociation distance is consistent between the different
initial positions of the elastic centres. The distance is \(\approx4c =
    35.4\) Bohr \(= 18.7 \AA\), this is double the distance seen in
Ghazisaedi and Trinkle \cite{Ghazisaeidi2012} and double the
distance that is found in the DFT Zr results by Clouet
\cite{Clouet2012}.



\subsection{{\bfseries\sffamily TODO} Dissociation Distance Analysis}
\label{sec:org761ce4a}
Following \cite{Clouet2012}, one can dislocation elasticity theory to
compute the dissociation distance of a dislocation in both the
basal and prism planes.  The energy variation caused by a
dissociation length \(d\) is

\[ \Delta E_{\text{diss}}(d) = - b_i^{(1)}K_{ij}b_j^{(2)}\ln \big( \frac{d}{r_c}
   \big) + \gamma d,  \]

where \(\mathbf{b}^{(i)}\) are the burger's vectors of the dissociated
dislocations.  \(\gamma\) is the corresponding gamma surface energy and
\(K\) is the Stroh matrix. Controlling the dislocation core radius
and the dislocation elastic energy, one can find the equilibrium
dissociation distance as 

\[
   d^{\text{eq}} = \frac{ b_i^{(1)}K_{ij}b_j^{(2) }}{\gamma}
   \]


With the orientation of the simulation cell as, \(U_1 = na \frac{1}{2} [10\bar{1}0]\), \(U_2 = mc [0001]\), 
 \(U_3 =  a \frac{1}{3} [1\bar{2}10]\), one finds the components of
 the Stroh matrix as:

\begin{align}
&K_{11} =& &\frac{1}{2\pi} \big( \bar{C}_{11} + C_{13} \big)
      \sqrt{ \frac{ C_{44} \big( \bar{C}_{11} - C_{13} \big)  }{
	      C_{33} \big( \bar{C}_{11} + C_{13} + 2C_{44} \big)  } 
	   }
\\    
&K_{22 }=& &\sqrt{ \frac{ C_{33} }{ C_{11} }  } K_{11}
\\
&K_{33} =& &\frac{1}{2\pi} \sqrt{ \frac{1}{2} C_{44} \big( C_{11} - C_{12} \big)  }_{}
\end{align}

here, \(\bar{C}_{11} = \sqrt{ C_{11}C_{33} }\).


From the gamma surface, for the basal plane one expects a
dissociation of \(1/3[1\bar{2}10] = 1/3[1\bar{1}00] +
    1/3[0\bar{1}10]\). Then dissociation length in the basal plane is
given by 

\[
    d_{\text{b}}^{\text{eq}} = \frac{ ( 3K_{33} - K_{11} ) a^2 }{ 12 \gamma_{\text{b}} } 
    \]

For the prism plane the \(1/3[1\bar{2}10]\) dislocation can
dissociate into \(1/6[1\bar{2}10] \pm \alpha(c/a)[0001]\) where the
parameter \(\alpha\) controls the position of the stacking fault minimum
along the [0001] direction. Only in interatomic potentials like
the EAM, do we find that \(\alpha = 0.14\). 

The dissociation length is 

\[
    d_{\text{p}}^{\text{eq}} = \frac{ ( K_{33}a^2 - 4 \alpha^2 K_{22} c^2 ) }{ 4 \gamma_{p} }
    \]



\subsubsection{Analysis with Final Ti model.}
\label{sec:orgff20b2f}

 Using 
\[
    d_{\text{p}}^{\text{eq}} = \frac{ ( K_{33}a^2 - 4 \alpha^2 K_{22} c^2 ) }{ 4 \gamma_{p} }
    \]

with \(K_{33} = 6.79853\) GPa \(= 6.79853 / 160.21766208 = 0.042433087\)
eV/\AA{}\(^{\text{3}}\) , \(\alpha = 0\) and \(\gamma_{\text{p}} = 98.95340236\) mJm\(^{\text{-2}}\) \(=
    1.6021766208*10^(-19) * 10^{-3} * 10^20 * 98.95340236 = 1.58540827809\)
eV/\AA{}\(^{\text{3}}\), \(a = 2.955577 \AA\) we have the equilibrium dissociation
distance in the prismatic plane as \(d_{\text{p}}^{\text{eq}} = 0.05845
    \AA\), which seems a little close looking at the results\ldots{} 

\subsection{{\bfseries\sffamily TODO} Disregistry Analysis}
\label{sec:org9fc894c}
Look into the theory of dissociation distance in Clouet paper
\cite{Clouet2012}


Disregistry given by the Peierls-Nabarro model. Analytic
expression given in Hirth and Lothe \cite{anderson2017theory}.

Disregistry \(D(x)\) is defined as the displacement difference
between the atoms in the plane just above and those just below the
dislocation glide plane. The derivative of this function \(\rho(x) = \partial
    D / \partial x\) corresponds to the dislocation density.


\[
    D_{\text{dislo}} = \frac{b}{2\pi} 
    \Bigg\{ \arctan \bigg[  \frac{x - x_0 - d/2}{ \zeta } \bigg] +
           \arctan \bigg[  \frac{x - x_0 + d/2}{ \zeta } + \frac{\pi}{2} \bigg]
	   \Bigg\}
    \]

Given \(x_0\) is the dislocation position, \(d\) is dissociation
length and \(\zeta\) is the spreading of each partial dislocation. 

\begin{align*}
  D_{L} &= &\sum_{n = -\infty}^{\infty}  &D_{\text{dislo}} (x - nL) \\
     &= &\frac{ b }{ 2\pi } 
        \Bigg \{ 
         &\arctan \bigg[ 
            \frac{ 
                  \tan \big( \frac{\pi}{L} [x - x_0 - d/2] \big)
                 }{ 
                 \tanh \big( \frac{\pi\zeta}{L} \big)
                  } \bigg]
       + \pi\bigg\lfloor 
       	 \frac{x - x_0 - d/2}{ \zeta } + \frac{1}{2}
       \bigg\rfloor \\
   & &+
         &\arctan \bigg[ 
            \frac{ 
                  \tan \big( \frac{\pi}{L} [x - x_0 + d/2] \big)
                 }{ 
                 \tanh \big( \frac{\pi\zeta}{L} \big)
                  } \bigg]
       + \pi \bigg\lfloor 
       	 \frac{x - x_0 + d/2}{ \zeta } + \frac{1}{2}
       \bigg\rfloor    \Bigg\},
\end{align*}

where \(\lfloor \cdot \rfloor\) is the floor function. 

For an array of dislocations in the S arrangement, \(D(x) = D_L(x)\),
with \(L = mc\), where \(m\) is the number of repeated unit cells in
the \(U_2\) direction. 

Here, \(U_1 = na \frac{1}{2} [10\bar{1}0]\), \(U_2 = mc [0001]\), 
\(U_3 =  a \frac{1}{3} [1\bar{2}10]\).

Therefore, using this, one can fit the three fitting parameters:
the dislocation position \(x_0\), the dissociation length \(d\), and the
spreading \(\zeta\). This procedure therefore allows us to determine the
location of the dislocation center.

For all interaction models, we find that this center lies in
between two (0001) atomic planes. One can see in Fig. 6 of
\cite{Clouet2012} that this position corresponds to a local symmetry
axis of the differential displacement map. This is different from
the result obtained by Ghazisaeidi and Trinkle
\cite{Ghazisaeidi2012} in Ti where the center of the screw
dislocation was found to lie exactly in one (0001) atomic plane.

\newpage


\subsection{IP1}
\label{sec:orga14152e}
\begin{center}
\includegraphics[width=0.7\textwidth]{Images/final_model_IP1_partial_dd_initial.png}
\end{center}
\begin{center}
\includegraphics[width=0.7\textwidth]{Images/final_model_IP1_partial_dd_final.png}
\end{center} 

\subsection{IP2}
\label{sec:org0f7f248}
\begin{center}
\includegraphics[width=0.7\textwidth]{Images/final_model_IP2_partial_dd_initial..png}
\end{center}
\begin{center}
\includegraphics[width=0.7\textwidth]{Images/final_model_IP2_partial_dd_final.png}
\end{center}
\subsection{IP3}
\label{sec:org9c9ff53}
\begin{center}
\includegraphics[width=0.7\textwidth]{Images/final_model_IP3_partial_dd_initial.png}
\end{center}
\begin{center}
\includegraphics[width=0.7\textwidth]{Images/final_model_IP3_partial_dd_final.png}
\end{center}
\subsection{IP4}
\label{sec:org50d82f9}
\begin{center}
\includegraphics[width=0.7\textwidth]{Images/final_model_IP4_partial_dd_initial.png}
\end{center}
\begin{center}
\includegraphics[width=0.7\textwidth]{Images/final_model_IP4_partial_dd_final.png}
\end{center}
\subsection{IP5}
\label{sec:org41eb5e8}
\begin{center}
\includegraphics[width=0.7\textwidth]{Images/final_model_IP5_partial_dd_initial.png}
\end{center}
\begin{center}
\includegraphics[width=0.7\textwidth]{Images/final_model_IP5_partial_dd_final.png}
\end{center}

\subsection{Ghazisaeidi Results for comparison}
\label{sec:orga56642d}

\begin{center}
\includegraphics[width=0.7\textwidth]{Images/ghazisaiedi-trinkle-scew-dislocation-core-prism-symm-asymm.png}
\end{center}

\subsection{Peierls Stress}
\label{sec:org27a4211}

By straining lattice and incrementally increasing the strain, one
can find the minimum stress necessary to move a dislocation from one
Peierls valley to the next. 

\subsubsection{Applying strain}
\label{sec:org517489b}

Applying strain as in \cite{Chen2013}. 

Here we are incrementing the strain by \(0.001C^{*}\), where \(C^{*}\) is
the transformed elastic constant necessary for transforming a
strain into a stress from the relation \(\sigma_{ij} = C_{ijkl}\varepsilon_{kl}\).

The original elastic constant matrix in its untransformed state
is:

\begin{equation*}
 C =	
  \begin{bmatrix}
   171.6093 &  94.6594 &  61.2262 &   0.     &   0.      &  0.      \\
    94.6594 & 171.6093 &  61.2262 &   0.     &   0.      &  0.      \\
    61.2262 &  61.2262 & 198.9006 &   0.     &   0.      &  0.      \\
     0.     &   0.     &   0.     &  47.4255 &   0.      &  0.      \\
     0.     &   0.     &   0.     &   0.     &  47.4255  &  0.      \\
     0.     &   0.     &   0.     &   0.     &   0.      & 38.4749  
  \end{bmatrix}
\end{equation*}

Transforming it into the dislocation coordinate system, by the
rotation

\begin{equation*}
 C =	
  \begin{bmatrix}
    1.0 & 0.0 & 0.0 \\
    0.0 & 0.0 & -1.0 \\
    0.0 & 1.0 & 0.0 \\
  \end{bmatrix}
\end{equation*}


\begin{equation*}
 C^{}^{\text{rot} }=	
  \begin{bmatrix}
   171.6093 &  61.2262 &  94.6594 &   0.     &   0.      &  0.      \\
    61.2262 & 198.9006 &  61.2262 &   0.     &   0.      &  0.      \\
    94.6594 &  61.2262 & 171.6093 &   0.     &   0.      &  0.      \\
     0.     &   0.     &   0.     &  47.4255 &   0.      &  0.      \\
     0.     &   0.     &   0.     &   0.     &  38.4749  &  0.      \\
     0.     &   0.     &   0.     &   0.     &   0.      & 47.4255  
  \end{bmatrix}
\end{equation*}



For finding the Peierls stress to move partials away from each
other on the prismatic plane plane one finds that the stress if
given by \(\sigma_{xy} = 2C_{xy}_{}^{*}\varepsilon_{xy}\), where \(C_{xy}^{*} =
    47.4255\) GPa.

To move the whole dislocation on the prismatic plane, one needs a
stress applied which is \(\sigma_xz = 2C_{xz}_{}^{*}\varepsilon_{xz}\), \(C_{xz}^{*} =
    38.4749\) GPa.

To move the dislocation onto the basal plane one needs to apply as
stress given by \(\sigma_yz = 2C_{yz}_{}^{*}\varepsilon_{xz}\), \(C_{yz}^{*} =
    47.4255\) GPa.



\subsubsection{xz Strain}
\label{sec:org23b2c84}

Applying an xz strain to the lattice causes the dislocation to
move along the prismatic plane. 

Using an increment in the strain of \(0.0001C^{*}\), where \(C^{*}\) is
the transformed elastic constant, with a value of \(C_{44}^{*}=38.4749\)
GPa, we find that actually the dislocation moves from one Peierls
valley along the prismatic plane at \(0.0012C_{44}^{*}\), giving a Peierls
stress of \(\sigma_xz = 2C_{44}\varepsilon_{xz} = 0.0923\) GPa


\begin{center}
\includegraphics[width=0.7\textwidth]{Images/final_model_peierls_xz_initial.png}
\end{center}
\begin{center}
\includegraphics[width=0.7\textwidth]{Images/final_model_peierls_xz_final_0.0012.png}
\end{center}




\subsubsection{xy strain}
\label{sec:org5f22e25}

An xy strain can move the partials of the prismatic plane apart. 

One can find the Peierls stress for these single partials to move
in opposite directions.

Here the \(\alpha\) parameter is 0.03. 

This means that the stress necessary to move the partial
dislocations apart is 

\begin{align*}
\sigma_{12} &= C_{1212}\varepsilon_{12} \\
    &= 2C^{\text{Voigt}}_{66 }\varepsilon_6^{\text{Voigt}} \\
    &= ( C_{11}- C_{12}) \varepsilon_6^{\text{Voigt}} \\
    &= 47.4255 \times 0.03 \\ 
    &= 1.42 GPa\ 
\end{align*}

The strain is applied to the whole cell, as the dislocation cell
is periodic, then the stress upon each partial is the same. 

\begin{center}
\includegraphics[width=0.7\textwidth]{Images/final_model_peierls_xy_0.03_initial_partials.png}
\end{center}
\begin{center}
\includegraphics[width=0.7\textwidth]{Images/final_model_peierls_xy_0.03_final_partials.png}
\end{center}


\subsubsection{Pyramidal Strain}
\label{sec:org30fddb4}

For a strain to transform the dislocation into the metastable,
pyramidal state, one can apply a strain which applies shear to the
dislocation whereby the maximum resolved shear stress is on the
first-order pyramidal plane. 

In the coordinate system of the dislocation, one finds that the
proportions of \(\sigma_{xz}\) and \(\sigma_{yz}\) should be \(c/a : \sqrt{3}/2 \approx
    1.83 : 1 \approx 1 : 0.54683\).


\subsection{Data}
\label{sec:orgac3ae2e}
\href{file:///home/tigany/Documents/ti/final\_model\_2019-11-12/results\_2019-11-09\_muc/IP1-oo\_19-11-09--04-46-00.log}{IP1}
\href{file:///home/tigany/Documents/ti/final\_model\_2019-11-12/results\_2019-11-09\_muc/IP2-oo\_19-11-09--04-46-00.log}{IP2}
\href{file:///home/tigany/Documents/ti/final\_model\_2019-11-12/results\_2019-11-09\_muc/IP3-oo\_19-11-09--04-46-00.log}{IP3}
\href{file:///home/tigany/Documents/ti/final\_model\_2019-11-12/results\_2019-11-09\_muc/IP4-oo\_19-11-09--04-46-00.log}{IP4}
\href{file:///home/tigany/Documents/ti/final\_model\_2019-11-12/results\_2019-11-09\_muc/IP5-oo\_19-11-09--04-46-00.log}{IP5}

\subsection{Directory of the results}
\label{sec:orge389787}
\url{file:///home/tigany/Documents/ti/2019-09-11\_final\_model/tbe/dislocations/2019-11-08\_no\_omega\_ordering\_ec\_latpar/}
\url{file:///home/tigany/Documents/ti/final\_model\_2019-11}


\section{Binding energies to dislocations}
\label{sec:orgdb8c9ef}

A strategy to find the binding energies of different sites are to
firstly just determine. 

\begin{enumerate}
\item Octahedral sites near the dislocation core
\begin{itemize}
\item Shall one find a radius within which one can find binding
sites?
\item Shall one build the perfect lattice and then move the site
into the relaxed octahedral one.
\item Find non-equivalent sites near the core
\item Find the average dislplacement going from the perfect site to
the relaxed cell with dislocation
\item Displace octahedral site by the average of the displacement of
the octahedral sites.
\end{itemize}

\item Relax the relaxed dislocation and the binding sites such that one
can find the solution energy.

\item Make perfect lattice, then find displacement from relaxed. Find
all octahedral sites near a particular dislocation core and then
displace cite by amount
\end{enumerate}


\section{BOP}
\label{sec:orgeac4dc4}

\subsection{4 recursion levels}
\label{sec:orgeeca7c8}

kbT = 0.1

>> Lattice parameters:

> hcp
\begin{center}
\begin{tabular}{ll}
a & 2.901660  \AA{}\\
c & 4.747485  \AA{}\\
etot & -18.342162  eV\\
\end{tabular}
\end{center}

> omega
\begin{center}
\begin{tabular}{ll}
a & 7.917318  \AA{}\\
c & 2.749892 \AA{}\\
etot & -17.458700 eV\\
\end{tabular}
\end{center}

Omega is still not as stable as hcp as expected from model. 


>> Elastic Constants

\begin{center}
\begin{tabular}{lrr}
Quantity & calc. (10\(^{\text{11}}\) Pa) & exp. (10\(^{\text{11}}\) GPa)\\
\hline
C11 & 1.781 & 1.761\\
C12 & 0.738 & 0.868\\
C13 & 0.611 & 0.682\\
C33 & 1.969 & 1.905\\
C44 & 0.285 & 0.508\\
C66 & 0.522 & 0.450\\
K & 1.050 & 1.101\\
R & 0.669 & 0.618\\
H & 0.558 & 0.489\\
\end{tabular}
\end{center}

\section{Bibliography}
\label{sec:orge31af3c}
\label{orgc315246}

\bibliographystyle{unsrt}
\bibliography{bibliography/org-refs}
\end{document}