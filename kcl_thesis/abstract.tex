\begin{abstract}
  Martensitic bearing steels have been shown to undergo subsurface microstructural decay,
  forming Dark Etching Regions (DERs), promoting failure through rolling contact fatigue
  (RCF). Dislocation-assisted carbon migration is thought to be the underlying mechanism,
  yet empirical studies have been inconclusive as to how dislocations move carbon and
  where excess carbon from the martensitic matrix migrates to upon transformation to
  ferrite---a phase of significantly lower carbon solubility. In this report, we detail
  the first stages of a multi-scale modelling approach to elucidate carbon transport by
  dislocations. Tight-binding simulations of carbon interactions with the $1/2\langle 111
  \rangle$ screw dislocation found solute distribution to vary significantly within
  $\sim2$b of the easy and hard cores; the highest binding energy being found in the
  centre of the hard screw core---which is the ground state carbon-dislocation
  configuration---in agreement with Density Functional Theory (DFT). Determination of
  equilibrium carbon concentration along dislocation lines, at various dislocation
  densities and nominal carbon concentrations, found most sites around the hard core were
  saturated, with all easy cores reconstructing to hard due to saturation of adjacent
  octahedral sites. In the typical temperature range of bearing operation, we expect all
  dislocations to be of hard core type, pinned by carbon in a prismatic site within the
  dislocation core. We anticipate large drag forces acting on dislocations in the initial
  stages of glide, due to carbon-dislocation binding.  Line-tension modelling of kink-pair
  formation shows a small, but consistent, reduction in the kink-pair formation enthalpy
  with carbon ahead of the dislocation, for all stresses, resulting in a modest increase
  in average dislocation velocity. In conditions allowing for carbon to equlibriate
  between trap sites, the average kink-pair formation enthalpy is reduced, increasing
  dislocation velocity, however a self-consistent method is necessary for accurate
  results, which is left for future work. For a solute-drag mechanism of
  dislocation-assisted carbon migration, the reduction in migration energy between carbon
  sites in the vicinity of a screw dislocation is crucial for its potential validity, such that
  carbon can keep up with dislcoations upon movement. Measurement of these migration barriers using a quantum-mechanical method will provide us with more accurate predicted stress and temperature regimes in which solute drag by screw dislcoations is a valid mechanism. These results
  provide data for the last stage in this multi-scale approach: self-consistent kinetic
  Monte Carlo (SCkMC) simulations incorporating solute diffusion, to ascertain how carbon
  moves with dislocations in different stress, temperature and concentration regimes.
\end{abstract}
