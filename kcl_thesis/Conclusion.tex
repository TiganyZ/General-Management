\chapter{Conclusion}
\label{ch:Conclusion}

\epigraph{{\textit{``They tell you I'm insane''}}}{Taylor Swift}

The work presented in this thesis addresses several aspects of the computational simulation of hydrophobic gating within nanopores and ion channels. Initially, the chemical nature of amino acids was varied and modelled in a number of biomimetic nanopore models to show any influence on water flow through (chapter 3). Building such motifs resulted in the energetic understanding of how hydrophobic residues in the models constructed are interacting with water, ions and hydrophobic molecules (chapter 4). It was found that the breaking of such barriers could be effected by external forces such as voltage (chapter 5). The underlying hydrophobic theme was then used to explore the functional state of the crystal structure of an ion channel (chapter 6).

\section{Computational Design of Nanopores}

Firstly, a computational approach to building and simulating $\beta$-barrel nanopores has been investigated in which a number of porin and lytic proteins were systematically explored as mimics for the size of the barrels. Modeller derived barrels and equilibrium simulations of 100 ns in DPPC bilayers suggest that stable barrels can be formed in this manner for further investigations. Within the pores, the shape and chemical nature can also be modified; be it central or terminal construction, and polar or apolar residues, this can result in a different conductive state of the pore based on the size of the models. With the 14 strand $\beta$-pores, I have been able to computationally transplant a hydrophobic barrier to water and ions based on the dimensions of hydrophobic gates within channel proteins. 

Based on this method, transplantation of other motifs that can influence gating and selectivity of their native proteins is seen to be the next reasonable direction. This could be a charged constriction as seen in \textit{e.g.} OmpF and OmpC porins \cite{Kumar2010}, a titrable residue observed in the protonation state of His37 in influenza M2 channel \cite{Cross2012}, or replication of the phenylalanine, tryptophan and leucine selectivity filter in the \textit{Helicobacter pylori} urea transporter, which has been characterised computationally \cite{McNulty2013}. In the case of the latter two (and hydrophobic gating), protein conformation also contributes to the residue behaviour, and to replicate this change in dimensions, barrel size can also be modulated in terms of the number of strands to imitate a change in conductive radii. 

\section{Hydrophobic Gates}

Secondly, a hydrophobic barrier has been characterised within a computational model nanopore. Free energy calculations have been used to estimate ion species and water molecules going through hydrophobic regions of this nature, extending previously studies on hydrophobic nanopores and the energetics of transport through them. Another outcome would be the use of water as a proxy for the ionic conductance of a pore. Shown in the work of this chapter is that high barriers to water permeation do actually correspond to regions which are functionally closed to ions. In turn, this allowed for the prediction of conductivity from shorter simulations without sufficient ion conduction events. The nature of hydrophobic transport through such regions has also been investigated on longer time scale simulations, however still are not sufficient for an accurate estimate of such energies. The free energy simulation method used may also be transferable to other proteins or motifs in which hydrophobic barriers could be investigated, as a relatively quick way of establishing if such barriers are present. I have worked on the assumption that short, small spaced windows can be used to explore the free energy that will in theory not allow much protein movement, shown in previous free energy calculations \cite{Yoluk2013}.

Umbrella sampling has been used in this chapter to investigate the free energy, however there are other methods available \cite{Domene2009}. Examples of this, thermodynamic integration or free energy perturbation, which can be used to calculate the maxima within the hydrophobic bands or regions. Metadynamics can also be used for permeation simulations, such as translocation of ions through potassium selectivity filters \cite{Domene2008}, in which a single simulation can be used instead of multiple, allowing for a more efficient method for free energy calculations on many molecule types.

\section{Simulation of Voltage}

Within chapter 5, it has been demonstrated that electrowetting of a hydrophobic gate within a $\beta$-barrel nanopore is feasible in a simulation using both constant electric field methods and a ``computational electrophysiology'' type protocol to implement a voltage. The conducting water within the hydrophobic region is similar to other proteins with hydrophobicity \cite{Spronk2006}, CNTs \cite{Vaitheeswaran2004f}, and theoretical nanopores \cite{Dzubiella2003x,Dzubiella2005} under a simulation voltage and electroporation under higher voltages.

Higher voltages were applied to try and break the hydrophobic barrier within the larger model nanopores, resulting in the breakdown of the bilayer around the protein. To allow higher voltages, simulations could be performed using a lipid with a higher voltage threshold to induce these breaks. These could include \textit{e.g.} the branched chain archaeal lipid DPhPC and related species which are known in simulation to break at higher thresholds \cite{Polak2013}. There is also (as mentioned throughout this thesis) scope for the use of various water models which could induce a differing dipole, and polarisability under a potential difference.

\section{High-throughput Selection of Protein Gates}

Finally, with the protocol for hydrophobic biomimetic pores devised, this was then utilised on a protein pore which possibly contained a hydrophobic gate. It has been shown that the \HT\ channel is in a closed conformation to cations and anions, based on free energy calculations, and is hydrophobically gated. Short umbrella sample time scales, 1 \AA\ windows, and simulation of only the M2 helices can give an accurate value, corresponding to other calculated values of closed state channels.  This creates a possible protocol for looking at other ion channels, their conductivity and state. However, the method is limited to structures that are present in an anticipated stable form. 

Use of harmonic restraints, in the form of an elastic or Gaussian network have been shown to modulate and stabilise closed and open forms of pLGICs \cite{Nury2010,Zhu2010}, and has been used in combination with free energy methods (ABF) \cite{Cheng2012} to predict accurate barriers. Use of only the M2 helix within this thesis, led to the observation of abnormal motion at the N and C termini and within the helices themselves. Incorporation of such networks into the M2 helices under umbrella sampling would reduce atypical protein movement, thus providing a more accurate measure of the free energy landscape. 


\subsubsection*{Limitations of Molecular Dynamics}

When using MD to simulate a system, it is important to remember the limitations of the method. Problems can arise from the fact that force fields are imperfect, an example being the free energy of solvation of amino acids which often have an error of \til 1 kJ mol\textsuperscript{-1}. As such, the error of the calculations within this thesis can be carried along. When considering polarisation within these force fields, water molecules can reorientate as shown here, however their partial charge is fixed \citep{Apol2013}. Therefore, as repeated within this simulation, it would be advantageous to use a more detailed water model \cite{Chaplin2001}, however this would lead to a decrease in computational efficiency to a high degree.  Other limitations include the simulation of pH and the inability of proton transfer in simulation, and also the general limitation in simulation timescale and size that can lead to low sampling problems. 

