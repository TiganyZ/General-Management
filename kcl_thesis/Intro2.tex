\chapter{Introduction}
\label{ch:Intro}

\epigraph{{\textit{''Are we out of the woods''}}}{Taylor Swift}

\section{Cellular Domains}

All life as we know it is cellular based, with the compartmentalisation of chemical reactions defining the cell, enabling a range of diverse functions. There are three domains with this cellular base, eukaryotes, prokaryotes, and archaea. This last domain are a minor group in our consideration and are defined mainly from prokaryotes by ether linkages in their cellular lipids. They will not be considered in the rest of this thesis. 

Eukaryotes differ to prokaryotes fundamentally due to their enclosed nucleus within the cell and deoxyribonucleic acid (DNA) organisation. Eukaryotes have multiple internal compartmentalised organelles such as mitochondria, the nucleus, plastids (if plant based) but also many others (such as the golgi apparatus, chloroplasts, endoplasmic reticulum). The remainder of the cell volume is occupied by the cytosol. Also differing are the cytoskeletal elements; microtubules, composed mainly of tubulin and microfilaments which are actin based. These exist within the cell to support shape and functions. A major differing factor between eukaryotes and prokaryotes is their size, with generally eukaryotes being much larger (with average cell size of 10 -100 $\mu$m compared to a standard prokaryote. \textit{i.e.} a bacteria 1-5 $\mu$m). A diagram of both, of a standard cell of each are in figure \ref{fig:cells}\footnote{source of images, lassconnection.s3.amazonaws.com/327/flashcards/1031327/jpg/cell\newline -structure1334774434699.jpg \& www.fullfrontalanatomy.com/Bio3/Standard_docs/Resources/Campbell/\newline Chapter_04/B_Jpeg_Images/04_Unlabeled_Images/04_04aProkaryoticArt-U.jpg}.

\begin{figure}[H]
\begin{center}
\includegraphics[width=\linewidth]{./pictures/Intro/cells}
\caption[Eukaryotic and prokaryotic cells] {Eukaryotic and prokaryotic cells. Cell sizes range from 10-100 $\mu$m for the eukaryotic cell (A.) and 1-5 $\mu$m for the bacterial cell, B.  Images are not to scale. Copyright of  Pearson Education.}}
\label{fig:cells}
\end{center}
\end{figure}

\newpage

\section{Biological Membranes}

Cells are surrounded by a semi permeable biological membrane. Examples are the membrane which defines the cytoplasms to the extracellular matrix, but are also used intracellularly to define the organelles. Our description of the membrane is based on the fluid mosaic model first proposed in 1972 where the membrane is described as a two-dimensional liquid in which the lipid, such as phospholipids and proteins can diffuse \cite{Singer1972}.  This is regulated in permeability, with control of what can go through the membrane. Both eukaryotes and prokaryotes are based on the same bilayer model, however, with differing lipid compositions. A lipid is characterised by their hydrocarbon tail, in number and length, the saturation of mentioned tail, and functional head group \cite{Luckey2008}. In a typical arrangement, a membrane is formed from two leaflets of the phospholipids and sphingolipids, with the head groups facing away from another and the hydrocarbon tails in contact. This creates a hydrophobic core surrounded by polar head groups \cite{Wiener1991}; an example of the cellular membrane arrangement is shown in figure \ref{fig:euk_mem}. The presence of such a structure permits a high energy barrier of \til 50 \kj\ to ions \cite{Parsegian1969}.

\begin{figure}[H]
\begin{center}
\includegraphics[width=\linewidth]{./pictures/Intro/euk_mem}
\caption[Schematic of eukaryote cell plasma membrane.] {Schematic of eukaryote cell plasma membrane. Lipids are indicated as dark green, with schematic proteins (green cylinders, pink and yellow adapted circles).  It is estimated 50 \% of the bilayer is protein, which is not represented accurately in this figure. Not shown are lipid rafts, membrane curvature or cytoskeletal elements.}
\label{fig:euk_mem}
\end{center}
\end{figure}

The three major classes of lipid within the eukaryotic bilayer are phospholipids, sphingolipids and sterols (in the form of cholesterol); all of which are amphipathic in chemical nature. Prokaryotes (bacterial bilayers for this thesis) contain a more varied class of lipids with there being two defined classes, Gram-negative and Gram-positive. This is based on their different colours under gram stain \cite{Gram1884}, indicating the presence of peptidoglycan at the surface of the bacterium. A basic representation of the Gram-negative bacterial bilayer structure (both inner and outer leaflets) is shown in figure \ref{fig:bacterial_OM}.

\begin{figure}[H]
\begin{center}
\includegraphics[width=\linewidth]{./pictures/Intro/bacterial_OM}
\caption[Schematic of the cell envelope of a Gram-negative bacteria.] {Schematic of the cell envelope of a Gram-negative bacteria. Shown are both the inner and outer membranes with dark green lipids indicating phospholipids, orange lipid A and the red circles being the remainder of the LPS. Also shown are schematic proteins within both membranes (green). Image is not to scale and adapted from \cite{Parkin2015}.}
\label{fig:bacterial_OM}
\end{center}
\end{figure}
\pagebreak

The most common lipid in biological membranes are phospholipids whose backbone is a \textit{L} isomer \textit{sn-}glycerol-3-phosphate. The acyl chains vary in length from 10 to 24 hydrocarbon tails (and always an even number) and are attached to an alcohol group via a glycerol linkage, resulting in many possibe phospholipids. The main ones to consider are phosphatidylethalonamine (PE), phosphatidylcholine (PC), phosphatidylserine (PS), phosphatidylglycerol (PG), and phosphatidylinositol (PI) lipids. An example of a phospholipid structure is shown in figure \ref{fig:POPC}.

\begin{figure}[H]
\begin{center}
\includegraphics[width=12cm]{./pictures/Intro/POPC}
\caption[POPC lipid structure.] {POPC lipid structure. Above, schematic of a cell membrane, in which each sphere and two tails represents a phospholipid. Below, chemically drawn structure of POPC.}
\label{fig:POPC}
\end{center}
\end{figure}

Simulated in this thesis are 1,2-dipalmitoylphosphatidylcholine (DPPC) and 1-palmitoyl-2-oleoylphosphatidylcholine (POPC) lipids, who have the same choline head group but different lipid tails. DPPC has two 16 carbon saturated chains, however POPC has one saturated 16 carbon, and one 18 carbon chain with one double bond. The chain saturation and its length are important in the shape and thickness of the membrane. Also found in bilayers are glycolipids and sterol based lipids which also contribute to the properties of the bilayer, with interplay between protein and lipid determining ultimate membrane curvature \cite{McMahon2005,McMahon2015}. 

Within the Gram-negative species, the unusual lipid, lipid A is present. It is a moiety of outer lipopolysaccharide (LPS) layer, with the lipid having \til 6 acyl chains between 10 and 16 carbons in length \cite{Sohlenkamp2015} and ester linked to two glucosamine sugars which bond to the other components of the LPS. Also present are phospholipids and some species have hapnoids which act in the same way as sterols to modulate membrane ordering. The outer membrane (OM) is asymmetric, with ~\til 70-80\% PE, 20-30\% PG, and cardiolipin present on the inner leaflet and lipid-A containing LPS on the outer. As well as a lipid A molecule,  the OM contains an inner core, outer core of oligosaccharide components such as the acidic and negatively charged heptose, galactose, mannose (to name a few), and an outer most O antigen of a repeating glycan polymer with variation species to species \cite{Sperandeo2009}. Calcium ions are known to associate with the OM and stabilise the structure of the membrane \cite{Ferrero2007}. The OM provides the region in which passive transport occurs, with substrates and solutes coming from external sources, such as sugars, amino acids, and metal co-factors via the $\beta$-barrel porins. Most metabolic functions are carried out by the inner membrane, using $\alpha$-helical proteins \cite{Tamm2001}. The inner membrane is thought to be phospholipid containing, however composition is species dependent (table 1, \cite{Saenz2012}).


\section{Membrane Proteins \& Function}

Approximately 20 to 30 \% of eukaryotic genes expressed encode for transmembrane proteins \cite{Krogh2001} with some playing vital roles through the bilayer such as transport, cell signalling via a membrane receptor, cellular adhesion in an immune response and as membrane enzymes to regulate the bilayer.  

Membrane proteins can be classed into two groups, peripheral proteins (which attached to the surface of the bilayer) such as a PH (pleckstrin homology) domain and integral proteins be they monotopic, who span part way through the membrane or fully transmembrane (TM) with basic examples shown in figure \ref{fig:euk_mem}. Within this thesis, the focus will be in the TM integral proteins which come in two main structural types, $\alpha$-helices \& $\beta$-barrels. 

\subsubsection*{$\alpha$ helices \& $\beta$ barrels}

Most transmembrane proteins are $\alpha$-helical \cite{Popot2002}, with $\beta$-barrels being found on the outer membrane of Gram-negative bacteria, \textit{e.g.} porins, some bacterial secreted toxins and on the mitochondrial outer membrane (the 19 $\beta$ stranded voltage dependent anion channel-VDAC) (figure \ref{fig:membrane_proteins}). $\alpha$-helices are found on the inner membrane of Gram-negative bacteria and in the membranes of eukaryotes. Diversity of $\alpha$-helical protein is apparent, with variation on the helical assembly within the bilayer, the number of helices, their oligomeric state, and also presence and structure of extracellular and intracellular domains. The TM regions are considered stable domains that do not unfold completely in denaturation due to the hydrophobicity of the membrane. They are translocated and inserted into the inner membrane of bacteria and the endoplasmic reticulum membrane of eukaryotes by the Sec61$\alpha$ $\beta$ $\gamma$ \cite{DuPlessis2011} and SecYEG (bacteria) translocase.

In this thesis, simulations are conducted on the $\alpha$-helical cys loop receptor proteins which will be discussed in more detail further on in this chapter. 

\begin{figure}[H]
\begin{center}
\includegraphics[width=\linewidth]{./pictures/Intro/membrane_proteins}
\caption[Structures of membrane proteins.] {Structures of membrane proteins. Shown (from left to right) pentameric ligand gated ion channel (GLIC), a porin (OmpG) and a lytic, toxin protein $\alpha$-Hemolysin). Grey line indicates bilayer regions.}
\label{fig:membrane_proteins}
\end{center}
\end{figure}

Beta-barrels are membrane proteins consisting of a large $\beta$-sheet of intra-hydrogen bonds  within, to form a barrel like structure known as a porin \cite{Lasters1988,Murzin1994}, and are involved in the permeability of the OM \cite{Hancock1987}. Within these barrels, the hydrogen bonding pattern is usually anti-parallel,  and are encoded by \til3 \% of the genes within Gram-negative species \cite{Wimley2003}. The structure of the barrels are defined by a  shear number (\textit{S}) and by the number of strands. Shear number is the number of hydrogen bonds which have been ``jumped'' from the first strand to the last, and effectively introduce torque to the barrel which can influence the height of the strands and the diameter of the barrel. Thus a barrel can have the same number, and length of strands but have different dimensions in height and width due to the shear number. Porins are structured in this way and also have long flexible loops connecting the strands, with typically the longer loops facing the extracellular region for recognition and selectivity. Within the larger porins (\til 16 strands or more), the loops are known to fold back within to the barrel lumen to provide moderate substrate recognition regions \cite{Tamm2001,Schulz2002,Fairman2011}.  Strand number can range from 8-22 and is always even (apart from the eukaryotic porin VDAC \cite{Bayrhuber2008,Ujwal2008}). They can be general \cite{Rosenbusch1974a,Schenkman1984a,Schirmer1998} or substrate specific in their conduction \cite{Schulz1996,Pages2008}, with some of the larger porins also having a plug-type region for coupled transport events, \textit{e.g.} BtuB which transports cobalamin within \textit{E. coli}. Gating within these pores are varied with voltage gating \cite{Lakey1987}, based on the charge distributions at their construction (OmpC, \cite{Liu2000} and the hydrogen bonding network \cite{Liu1998a}), or the position of key loops such as L3 (RcP porin, \cite{Soares1995}) or L6 loops (OmpG, \cite{Chen2008}) affecting the pore conduction directly. 


Structurally, porins can exists as monomers, or higher order states such as trimers (examples shown in figure \ref{fig:beta_barrels}) and have shown to be highly stable to unfolding over a wide range of temperatures and perturbing conditions \textit{in vitro} \cite{Moon2013}. There are also non-constitutive $\beta$-barrel membrane proteins produced by bacteria as toxins, which have a higher order oligomeric state (such as $\alpha$-hemolysin which is a heptamer, and $\gamma$-hemolysin which is octameric).

\begin{figure}[H]
\begin{center}
\includegraphics[width=\linewidth]{./pictures/Intro/beta_barrels}
\caption[Structures of uuter membrane proteins.] {Structures of outer membrane proteins found in Gram-negative bacteria. Trimeric pores shown of the cation selective OmpC pore, and the maltose specific LamB. Monomeric pores of NanC, which is specific for acidic saccharides and OmpG, for non-specific large oligosaccharides. Face down images are of LamB and OmpG. All pores are shown in the same orientation to which the top reflects the extracellular regions, and bottom the periplasm. Represented with the extracellular region being the top side for all 4 proteins. }
\label{fig:beta_barrels}
\end{center}
\end{figure}


To this month (September 2015), there are a total of 551 unique membrane protein structures \cite{White2015}, out of a total of 36,141 number of unique structures within the protein data bank (\til 1.5 \%). This low number in comparison is due to the difficulties of structure determination of such types of protein especially with over-expression, purification and crystallisation of such proteins \cite{Werten2002} as in most cases, these proteins are expressed at low levels within the native cell \cite{Tate2001}. Many methods are utilised to improve the crystallisation of membrane proteins such as detergents, antibody fragments and recently, lipidic mesophases for gaining x-ray crystal structures \cite{Caffrey2009}, in which the proteins are highly concentrated within a continuous cubic lipid phase.  In recent years structure determination methods have advanced significantly, especially with advances in cryo-electron microscopy (cryo-em). This was used to determine the structure of complex I of the respiratory chain \cite{Vinothkumar2014}, which has 44 subunits with a combined mass of 1 MDa and most recently, different conformational states of the glycine receptor (GlyR) have been defined via this method \cite{Du2015}. Improved computational techniques and algorithms have allowed this method of structure determination to be used for larger proteins, of which some are membrane proteins. Porins are known to be flexible (in their loop regions), thus improvement in structure determination which also has the ability to give dynamic information is also a consideration, such as nuclear magnetic resonance (NMR). The use of solid state magic-angle spinning NMR has allowed information on flexibility, consequently the dynamics, on what is usually considered a static protein by other methods \cite{Hiller2005} such as the structure by solid state NMR methods of the eukaryotic VDAC barrel \cite{Hiller2008}. The reconstitution of an integral membrane protein from its hydrophobic bilayer into a solvent is thought to change the structure, thus to prevent this and mimic a membrane environment within NMR, nanodiscs can also be used. This method has been used to study OM proteins, such as OmpX \cite{Hagn2013}.

To note, there are many different types of TM proteins with examples of $\alpha$-helical and $\beta$-barrels: 
\begin{itemize}
\item \textit{Membrane enzymes}. Prostoglandin H\textsubscript{2} synthase and OmpLA
\item \textit{Receptors}.  GPCR or neurotranmitters receptors such as GluA2 and GluCl
\item \textit{Transporters}. Secondary LacY, GlpT, FucP, LeuT. Also the ABC class of BtuCD-F and TonB and the drug effluxers of ErmE, ArcB, TolC
\item \textit{Channels and pores}. Aqp and GlpP, Potassium, CLC, Mechanosensitve, Porins
\item {Membrane proteins of the electron transport chain}. Complexes I-V, bc\textsubscript{i}
\end{itemize}

of which, only a subset will be discussed and considered for the remainder of this thesis. 

\newpage

\subsubsection*{Channels \& Pores}

Three major classes of TM protein exists for substrate movement through a membrane. Channels, which enable passive substrate flow down the concentration gradient at near diffusional rates. Pumps, which use energy in the form of ATP and its subsequent hydrolysis to transport against the substrates concentration gradient, and transporters, which can couple substrate movement against its concentration gradient to that of another, \textit{e.g.} protons, such as LacY, which transports lactose against its concentration gradient by using a favourable proton gradient. 

It is known that ion channels can be regulated by a variety of mechanisms, such as voltage, current rectification, light, mechanosensivitity, and ligand binding. This allows passive ion conduction events via the binding of a chemical, such as a neurotransmitter or drug. A subgroup of these ligand gated ion channels (LGIC's) are the cys-loop receptors \cite{Connolly2004,Lester2004a}. The structure of \na\ was first determined by low resolution EM (electron microscopy) \cite{Unwin1995a}, followed by a higher resolution (4 \angstrom)  8 years later \cite{Miyazawa2003}, and further refined structures the same year \cite{Unwin2005}. Based on the \na\ structure being the only pentameric ligand gated ion channel (pLGIC) structure available for a long period of time, other structures were initially based on homology models of it. Initially thought to be in an open conformation, the \na\ structure has shown to be functionally closed \cite{Beckstein2003}, showing that observation of a crystal structure and its dimensions cannot present a definitive answer into the conformational state of a channel. The first non-nAChR structure (23 years after the initial \na\ structure) was the bacterial GLIC channel in an open state (3.1 \angstrom\, followed by a higher resolution structure \cite{Bocquet2009a} and closed at neutral pH \cite{Hilf2009}) and the ELIC channel (closed) \cite{Hilf2008}. Another pLGIC, the glutamate gated chloride channel (GluCl) was crystallised in an open state and with agonist ivermectin \cite{Hibbs2011} and lipid (POPC) bound GluCl, \cite{Althoff2014} and most recently, the GABA\textsubscript{A} receptor \cite{Miller2014b}.
A new structure, the 5-HT\textsubscript{3A} receptor will be investigated in this thesis (chapter 6).

All ion transport events have to be regulated so cellular events can happen in the correct time and place, \textit{e.g.} the action potential. This can be done by three main ion current gating mechanisms shown in figure \ref{fig:gating}. 

\begin{figure}[H]
\begin{center}
\includegraphics[width=\linewidth]{./pictures/Intro/gating}
\caption[Ionic current gating mechanisms.] {Ionic current gating mechanisms. Shown are three representations , with the protein indicated in the center in grey. In the middle figure, charge is shown in red, and in the right hand image, hydrophobic bands are shown in black on the protein.  Bilayer is indicated with dark green circles and lines, blue sphere indicates a figurative ion. Ion radius (\textit{r\textsubscript{i}}) and pore radius (\textit{r\textsubscript{i}}) are indicated.}
\label{fig:gating}
\end{center}
\end{figure}


Via a conformational change, the structure of a membrane protein can change to accommodate the needed conductive state. The passage of substrate will not occur due to the physical presence of the channel protein within the conductive pathway. This is indicated in figure \ref{fig:gating} as a physical blockage, in which a change such as a ``helical'' movement changes the conductive pathway creating a region smaller than that of the translocating substrate. An example of this is the mechanism of KcsA with a gating state of the cation being dependent on the helix bundle crossing on the intracellular side therefore being regulated by a physical blockage to all solute species \cite{Uysal2009}.

Protein gating  and selectivity of ions via surface charge is done so by surface charge on the protein being able to exclude ions of the same charge, while ions (or molecules) of the opposite charge or neutral can conduct through, with the role of surface charges also contributing to general sensing \cite{Madeja2000}.  An example of this is also in KcsA's selectivity filter \cite{Zhou2001} and a crystal structure of the closed form \cite{Uysal2009}, in which, carbonyl groups of the p2 loop form the selectivity filter for potassium ions, thus begin cation selective from the charge association.

Ionic current gating via hydrophobicity is based on the unusual behaviour of water in confined hydrophobic regions and is a well studied gating mechanism. 

\subsubsection*{Hydrophobic Gating}

Within a protein pore with a hydrophobic chemical nature, be it a channel or theoretical model, there is a change in the density of water within the pore, dependent of the radius of the hydrophobic region. Therefore, with a change in radius as seen in conformational change of channels there is a drying or wetting transition within the hydrophobic protein region (within TM the protein pore). The radius and hydrophobicity of the channel is able to regulate the water, hence ion or substrate flow through. This was first observed in molecular dynamics (MD) simulations of model nanopores and also within hydrophobic carbon nanotubes of a certain radius \cite{Hummer2001,Beckstein2001}, with initial models showing bulk water transitions below a radius of 4.5 to 5.5 \AA\  \cite{Beckstein2003,Beckstein2004c} and general bulk water transitions modelled in hydrophobic pores \cite{Allen2003a,Allen2003g} (and sometimes conceptualised as the formation of nanoscopic bubbles \cite{Roth2008}).

The diameter of a water molecule is \til 3.5 \angstrom, thus assumed in a crystal structure if the region was at least this in diameter, a water molecule would in theory be present. However, in hydrophobic regions with radius less than \til 7 \angstrom, the bulk behaviour of water changes to varying degrees of liquid-vapour transitions. The most prominent transitions occur between a radius of 4 to 7 \angstrom, with only a 1 \AA\ change in radius changing the water state of a pore (observed in the TWIK potassium channel \cite{Aryal2014g}), thus a small change in radius form a conformational change can result in a different conductive states \cite{Beckstein2003}. A graphical representation is shown in figure \ref{fig:hydro}. 

\begin{figure}[H]
\begin{center}
\includegraphics[width=11cm]{./pictures/Intro/hydro}
\caption[Probability of water based on hydrophobicity] {Probability of water based on hydrophobicity. A. Representation of a pore (grey) in a bilayer (yellow and purple), radius and chemical nature of the pore can result in a dry-vapour region indicated. B Probability of water within the pore for a polar nanopore model and that of a hydrophobic one. The fully polar, hydrophilic pore remains wet. Once radius increases from 4 \angstrom, a change in water within the pore occurs with saturation at 7 \AA\ radius. dotted line represents the radius of a water molecule. Images adapted from \cite{Aryal2014i,Beckstein2004c}.}
\label{fig:hydro}
\end{center}
\end{figure}

Simulation tells that these are true gates in the form of free energy calculations \cite{Beckstein2004a}, not just to water but also to various ion species. In simulation, these barriers have been overcome with the application of a constant electric field \cite{Dzubiella2004d}. Experimentally, hydrophobic nanopores have been investigated in solid state nanopores, in which an applied voltage can cause wetting in an otherwise dry pore \cite{Powell2011}. 

Within this thesis, hydrophobicity will be investigated via the building (chapter 3), free energy calculations (chapter 4) and voltage breaking (chapter 5) of such gating motifs.

With the diversity of ion channels, many have pore dimensions which correlate to the radius of water transition events and also have the hydrophobic residues to do so. 

This has been proposed for a number of ion channel gates (review \cite{Aryal2014i,Trick2015}) with experimental investigations and simulations on the nicotinic acetylcholine receptor (\na) \cite{Miyazawa2003,Beckstein2006b,Corry2006}, also other pentameric ligand gated ion channels (pLGIC), \cite{Cheng2009,Zhu2012,Zhu2012g} and MscL \cite{Blount1999a,Beckstein2004c,Yoshimura1999,Anishkin2010,Birkner2012} with all using hydrophobic leucine residues. MD studies and free energy calculations into the \na\ receptor \cite{Beckstein2006b} demonstrated that the hydrophobic region of the pore, and the liquid-vapour state is indeed a true energetic barrier. 

Hydrophobic gating was also noted in voltage gated potassium channel \cite{Jensen2010,Jensen2012} and the TWIK potassium channel experimentally and via simulation \cite{Aryal2014g},  and with many other channels such as GLIC, and MscS (summarised in \cite{Aryal2014i}). Also recently in the magnesium channel CorA \cite{Neale2015}, which has a key leucine residue (L294) within the dry region of the pore


\section{Nanopores}

Protein engineering has been an expanding field in recent years with two strategies; rational design and directed evolution to design such proteins. With the improvement of protein structures and the use of computational tools, rational design of proteins can be used for nanoprotein assembly \cite{Ardejani2011}, enzymatic improvement for industry \cite{Singh2013}, membrane protein design \cite{Perez-Aguilar2012} particularly within the nanopore field. 

Nanopores have many potential uses within single molecule chemistry \citep{Bayley2008x} such as stochastic biosensors \cite{Braha1997a,Bayley2001}, electrical devices \cite{Lee2010a,Hilder2012}, water filtration \cite{Holt2006,Majumder2011a,Corry2008,Kosa2012}, separation of gases, ions, and biomolecules \cite{Lee2013e,Park2006,Diao2012}, and also in DNA sequencing \cite{Cherf2012,Manrao2012,Pennisi2012} which is the basis of the sequencing by Oxford Nanopores MiniION. This has been successfully used to identify bacterial antibiotic resistance islands \cite{Ashton2015} as part of the movement into next-generation sequencing with continual improvements \cite{Jain2015}. 

They can be constructed from different biological and non biological materials, with an overview of the most familiar nanopores in figure \ref{fig:other_nanopores}, which shall be discussed further and will be a focal point of this thesis. 

\begin{figure}[H]
\begin{center}
\includegraphics[width=\linewidth]{./pictures/Intro/other_nanopores}
\caption[Various types of nanopores] {Various types of nanopores. A. Biologically based monomeric porin OmpG. B. Cyclic peptide nanopore of  D,L-$\alpha$-peptide, with a pore formed via stacking (image from \cite{Lopez2001}. C. Carbon Nanotube. D. Solid state nanopore  and a typical scaffold and dimensions (image from \cite{Janssen2012}).}
\label{fig:other_nanopores}
\end{center}
\end{figure}

\subsection{Types of Nanopore}

To date, many reviews are available on the various types of nanopore \cite{Howorka2009,Hou2011,Kowalczyk2011,Stoloff2013}, with the basis that there are three main types of nanopores: protein, artificial and solid state. 

\subsubsection{Protein}


Protein nanopores were based initially on the lytic, pore forming protein, $\alpha$-Hemolysin ($\alpha$-HL) expressed by the \textit{Streptococcus} bacterial genus, which as a pathogen, punctures red blood cells. As a protein, it contains a barrel domain and a cap domain (figure \ref{fig:nanotech_pores} on page \pageref{fig:nanotech_pores}) and after association of the heptameric units, it spontaneously inserts itself into a lipid bilayer.

With the structure determined for \ahl\ \cite{Song1996}, a zinc binding site (using 4 histidine residues) was integrated into the pore which in turn turned the channel into a Zn\textsuperscript{2+} sensor \cite{Kasianowicz1999,Braha1997a} with the basis it could be used for various other detection  processes \cite{Bayley2001} such as protein \cite{Mov2000}, organic molecules \cite{Gu1999}, TNT \cite{Guan2005} and also detection of different chemical enantiomers through the pore \cite{Kang2006}. 

A recent development of the nanopore of \ahl\ is its ability to sequence DNA \cite{Branton2008} and RNA \cite{Akeson1999}. The pore itself ($\alpha$-HL) has a constriction of 1.4 nm (present at the barrier cap region at residue M113), which is larger than the diameter of a single stranded DNA (ssDNA), but smaller than that of the double helix, thus the single strand DNA can be threaded through the pore. Mutagenesis of \ahl\ \cite{Bayley2004} has allowed more manipulation, with enhanced translocation and sequencing of DNA \cite{Clarke2009} through via manipulation of the internal charge of the pore \cite{Maglia2008} to allow faster translocation and also the opposite, slowing down of DNA translocation through such a pore with the site directed mutagenesis of positive charges within the constriction \cite{Rincon2011}. Additionally adaptors \cite{Gu1999} such as $\beta$-cyclodextrin (and others of interest \cite{Sanchez2000}) have been used to improve stand recognition and gating events into the pore. 

Another nanopore of interest is the MspA pore \cite{Butler2008} (figure \ref{fig:nanotech_pores} on page \pageref{fig:nanotech_pores}) which, like \ahl\ has been shown to detect and translocate ssDNA with recognition of all 4 bases \cite{Derrington2010}. Unlike $\alpha$-HL, it is a member of the porin family produced by the \textit{Mycobacteria} genus, allowing hydrophilic substrates into the bacterium.  In comparison to $\alpha$-HL, it has a smaller constriction (of 1.2 nm) which is present at the base of the protein, near the periplasm entrance.

Other DNA sequencing pores for consideration are the mechanosensitve channel, MscL \cite{Farimani2015} and also the phi29 protein. This protein is a DNA-packaging motor from bacteriophage, which allows double stranded DNA (dsDNA) to enter the virus \cite{Wendell2009,Geng2013b}. It has also been combined with MspA to improve DNA sequencing \cite{Manrao2012}and is being used as the next-generation DNA sequencing device \cite{Schneider2012,Feng2015} and the basis of many biotech companies for such purpose.

\begin{figure}[H]
\begin{center}
\includegraphics[width=\linewidth]{./pictures/Intro/nanotech_pores}
\caption[Protein pores for DNA sequencing.] {Protein pores for DNA sequencing. \ahl\ \& MspA. \ahl\ consists of 7 subunits with a transmembrane $\beta$-barrel domain and a cap domain which is present as shown, between the cytoplasm and extracellular region in an eukaryotic membrane. MspA is an octameric porin with each subunit containing a $\beta$-sandwich and a beta ribbon. This is present physiologically on the OM of the bacterium.}
\label{fig:nanotech_pores}
\end{center}
\end{figure}

The sensing of analyte and DNA above is based on stochastic sensing (figure \ref{fig:stochastic}). With the addition of an applied potential, a normalised current is present due to the flow of ions on either side of the bilayer through the pore. A native or engineered binding site for the analyte is present in the pore (or the presence of a DNA strand). Each time an analyte binds, or a specific base of DNA is present within the constriction/pore region, the current is modulated - and its unique to the analyte and pore, from which concentration of the analyte can also be calculated based on the lifetimes of bound/unbound state.

\begin{figure}[H]
\begin{center}
\includegraphics[width=12cm]{./pictures/Intro/stochastic}
\caption[Stochastic biosensing of DNA.] {Stochastic biosensing of DNA. A time-current trace is shown from a theoretical binding event, in this figure it is the presence of a polymer (DNA) strand through the pore. The open pore has a higher current than that of a modulated, DNA-present pore, with DNA sequencing each base (A,T,C, or G) has its own current level. The change in current is dependent of the base present, and from the values, the sequence can be determined. Image adapted from \cite{Branton2008}.}
\label{fig:stochastic}
\end{center}
\end{figure}

Other biological pores mimicry and biosensing are also the similar to \ahl\ such as the anthrax pore (has a much longer barrel domain) \cite{Halverson2005}. But also the porins such as OmpF \cite{Hadi2014}, FhuA \cite{Mohammad2011,Niedzwiecki2012,Killmann1996} \& OmpG \cite{Chen2008,Zhuang2014} which have all been mutagenised to improve function as DNA sequencing pores or biosensors. It is also possible to modify the general activity of nanopores, such as MscL via various chemical biology modifications to become light activated \cite{Kocer2005}. 

\subsubsection{Artificial \& Peptide (including Graphene)}

\subsubsection*{Artificial} 
These nanopores are much small in size than biological pores and come in a variety of constructs and chemical motifs \cite{Ghadiri1994,Clark1999,Sakai2013}. Originally, these pores were made out of a central crown ethers, which are three amphipathic macrocyles on each side of the membrane, making the wall of the nanopore, which in turn produces a hydrophilic channel \cite{Carmichael1989} and the basis of many synthetic channels available in a wide range of channel motifs \cite{Echegoyen1994}. They can be formed out of metal-organic chemicals, from $\pi$-$\pi$ stacking, based on these macrocycles (with variable inward and outward pointing functional groups) and are shown to be ion selective and also water permeable \cite{Zhou2012}.

An example of a recent artificial channel is that constructed from hydrazine-appended pillar-[5]-arene derivatives which have been shown via an x-ray crystal structure to adopt tubular, pore structure which have been shown to insert into the membrane of lipid vesicles and conductive to water \cite{Hu2012}. The structure design for this synthetic pore is mimicked from the single pore features of aquaporins \cite{Verkman2011}. 

A number of artificial systems for transmembrane transport exist with transport of ions and neutral organic solutes \cite{Haynes2011,Cho2011} and water transport \cite{Liu2010} through many artificial channels \cite{Barboiu2013}.


\subsubsection*{Peptide}

Peptide nanopores can be formed from artificial $\beta$-barrels with varying structural motifs based on the strand \cite{Sakai2008}. Synthetic $\alpha$-helices are also functional, constructed from leucine and serine residues in a heptad repeat to form a helix. The arrangement places all leucines membrane facing and all hydrophilic serine residues lumen facing to create a conductive channel, and self assemble into a helix bundle \cite{Lear1997}. Based on this structure, it is possible to build such a pore in which selectivity and functionality can be represented through lumen pointing amino acids.

Cyclic peptide nanotubes \cite{Montenegro2013,Ghadiri1994,Ghadiri1993a} composed of $\alpha$ amino acids (example in figure \ref{fig:other_nanopores}, B. on page \pageref{fig:other_nanopores}) are also functional pores, with alternative stereochemistry can be used to form pores creating a structure known as $\beta$-helices. 

Most recently, nanopores have been constructed out of DNA \cite{Burns2013} and also DNA origami constructs \cite{Bell2012}. 

\subsubsection*{Graphene}

This well known material made from an allotrope of carbon which can exist in the 2D, as graphene sheets or within the 3D as buckminsterfullerene \cite{Kroto1985} or carbon nanotubes (CNT's) \cite{Iijima1991} (figure \ref{fig:nanotech_pores}, C. on page \pageref{fig:other_nanopores}). In the 2D form, it can be used as a nanopore, with functional conductive pores from 3 \AA\ in diameter \cite{Garaj2010}. The 2D graphene pores have also been shown to translocate DNA \cite{Merchant2010,Schneider2010,Min2011,Liang2013,Liang2014a} with investigations into its hydrophobicity for such purpose \cite{Schneider2013c}. Computationally, the pores have been modelled with a  biomimetic ammonia switch within a theorised synthetic channel \cite{Titov2010} and as bioinspired pores with selectivity for \Na\ and K\textsuperscript{+} \cite{He2013}.

CNTs can form functional pores and has been simulated as water conductive \cite{Hummer2001,Zhu2003}, with the addition of charges used to modulate selectivity through a model CNT pore \cite{Garcia-Fandino2012b}. Recently, this stochastic transport through these CNT in lipid bilayers and live cell membranes has been experimentally noted\cite{Geng2014}. However, these pores are difficult to insert into bilayers unless externally modified \cite{Pogodin2010} due to the hydrophobic nature of the nanotube and their interaction with the hydrophobic lipid tails. 

Both carbon forms can act as general pores (with CNT being radius dependent), however specificity is difficult introduce in the water-conductive lumen of both via specific chemical modification due to the size and the ubiquitous nature of the material. External modification of both is well characterised (in CNT at the end of the tubes) and could have useful medical implications \cite{Kuila2012,Vardharajula2012}.


\subsubsection{Solid State}

First fabricated in \til 1991 as a polymer nanopore (overview in \cite{Dekker2007,Keyser2011a}), they were designed initially as a scaffold for biological applications. Initially pores were created with ion beam sculpting on Si\textsubscript{3}N\textsubscript{4} membranes which created pores, with diameters of \til 60 nm. With the use of ion-track etching, pores with much smaller diameters of up to 2 nm were constructed on silicon nitride (SiN) membranes \cite{Siwy2002} with the development of such method for thinner membranes \cite{Kuan2012}, with recent electron beam methods being used to create pores which can be visually inspected \cite{Storm2003}. 

Solid state nanopores have been used in the detection of many biological species, be they DNA \cite{Li2001,Fologea2005,Storm2005,Kowalczyk2010}, RNA \cite{Skinner2009}, proteins \cite{Han2006,Talaga2009,Martin2007}, CNTs \cite{Hall2011}, and also the recognition of DNA structures \cite{Kowalczyk2010,Wanunu2009}.  They have also been modified for the sensing of small molecules, such as potassium in the form of a DNA switching event (within the lining of the pore) with the binding of K\textsuperscript{+}, to these DNA strands and changing the diamater of the pore and modulating pore current \cite{Hou2009}. Also implementation of zinc finger motifs which have been immobilised on the channel walls, changing configuration with Zn\textsuperscript{2+} binding \cite{Tian2010}, With the same principle being applied to F\textsuperscript{-} binding and recognition \cite{Liu2014}.

Variations to the standard solid state pore have been applied with chemical modification of pores allowing for regulation via pH \& voltage \cite{Wanunu2007} and development into voltage gated pores \cite{Siwy2006}. Based on immobilisation with (trimethylsily)diazomethane \cite{Powell2011} which incorporates hydrophobicity into the newer polyethylene terephthalate (PET) nanopores. This type of modulation brings about much interest into the use of such mechanisms for nano-electronics and voltage gated nanopores \cite{Smirnov2011}. Such pores have also been lipid coated \cite{Yusko2011a}, based on insect olfactory systems, which trap odourants within lipid pores on their skin. This nanopore system has been shown to act in the same manner, with concentration of such molecules within the lipid-lined solid state pore. 

More variations have been applied in the replication of the nuclear pore complex within a solid state system \cite{Jovano2014,Kowalczyk2011} with recent example in which a solid state pore has been functionalised with an FG strand mimicking polyisopropylacrylamide (pNIPAM),  where faster single stranded DNA translocation was noted through the functionalised pore \cite{Jovano2014}. Pores have also been demonstrated for their DNA sequencing ability \cite{Schneider2010} with comparisons between both solid state and biological sequencing pores being made \cite{Haque2013}.

The solid state scaffold can also be hybridised with protein nanopores such as gramicidin A \cite{Balme2010} and \ahl\ \cite{Hall2010}. 

\section{Utility of Simulation}

Simulations can be performed using many or few levels of detail, from the computationally demanding quantum mechanics to the continuum based models of simulation (figure \ref{fig:comp_methods}, A.). 

Generally, the more detail in a simulation the longer it will need to be simulated, therefore one needs to consider a balance between level of detail and efficiency. Within this thesis, only atomistic (AT) and coarse grained (CG) simulations will be used, with the main detailed focus of this thesis on the AT simulations with an approximation of accessible time scales shown (figure \ref{fig:comp_methods}, B.). 

\begin{figure}[H]
\begin{center}
\includegraphics[width=10cm]{./pictures/Intro/comp_methods}
\caption[Compromise between computational details and demands and computational methods time scales.] {Compromise between computational details and demands and computational methods time scales. A. Comparison of details and demands. QM methods require a higher computational demand with comparably the least reduction in detail whereas property based methods are the opposite. Methods used in this thesis are indicated in bold. B. The circles indicate an estimate of the boundaries of the possible time and length scales feasible. Figure adopted from Dr. Jochen Klingelhoefer. }
\label{fig:comp_methods}
\end{center}
\end{figure}

\section{Work Presented in this Thesis}

Based on the increasing number of nanopores, numerous biomimicry available, and increasing number of protein structures, this thesis will be based on the design and implementation of biological motifs into a computational nanopore model. 

Within chapter 3, the question of whether a hydrophobic gate can be computationally transplanted into a simple nanopore model will be addressed, based on the known structure and simulation of other $\beta$-barrel proteins \cite{Ziervogel2013c,Aksimentiev2005a,Wells2007,Luo2010}. The functionally of such a motif will be added in the form of hydrophobicity and polarity of amino acids, and analysed by water flow through the pore and compared with many previous simulations of water within hydrophobic nanopores \cite{Hummer2001,Beckstein2003a,Allen2003g,Dzubiella2005,Peter2005,Smirnov2010,
Chiavazzo2014b} to investigate the hydrophobic gate, and validate the theoretical model's stability and shape.

Based on previous free energy calculations \cite{Pongprayoon2009,Rui2011,Modi2012d,Modi2014}, ion conduction through hydrophobic pores \cite{Song2009,Thomas2014} and in combination with nanopore design in pores \cite{Pongprayoon2012,Garcia-Fandino2012b,Corry2011c}, the validity of the gate from the previous chapter will be assessed via free energy calculations. This will be conducted on some of the more interesting model pores (ability to change water conductance through) with the aim to address the way in which the hydrophobic region regulates ions and water through the pore. 

Simulation under voltage has also been extensively simulated, be it membrane only \cite{Tarek2005,Bockmann2008,Delemotte2012,Polak2013} or with protein incorporated \cite{Suenaga1998,Siu2007,Wells2007,Roux2008}. Therefore, models will be simulated under voltage, as it is known that it can indeed break a hydrophobic barrier within a solid state pore \cite{Powell2011}. Therefore, can such a transplanted barrier within the model nanopores be broken in the same manner? The simulation will be analysed based on previous simulations of hydrophobic pores and their water behaviour \cite{Ho2013,Tokman2013} within nanopores \cite{Vaitheeswaran2004f,Dzubiella2004d,Dzubiella2005,Bratko2007,Vanzo2015x} as an assessment of the affect of voltage upon these models and the hydrophobic barriers. 

The function of the biomimetic hydrophobic gates shall be compared to a new serotonin receptor (5-HT\textsubscript{3}) which is presumed to be in a closed state and gated in a similar manner. It will also be investigated via free energy calculations based on previously known gating mechanisms of pLGIC's \cite{Calimet2013} with combined theory and simulation into their ion conduction events \cite{Cheng2010,Zhu2012a}. Previous simulation of the drying transition in the hydrophobic gate of GLIC \cite{Zhu2012}, \na\ \cite{Beckstein2006b} channels, as well as equilibrium MD \cite{Murail2011a,Murail2012} and conformational states simulations \cite{Yoluk2013,Yoluk2015} on other ligand gated ion channels allow a full overview and comparison to the new channels conductive state.

Chapter 7 will present an overview of the findings of the previous chapters highlighting their significance with future direction discussion. 
