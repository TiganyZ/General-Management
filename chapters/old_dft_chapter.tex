% Created 2019-07-09 Tue 15:30
% Intended LaTeX compiler: pdflatex
\documentclass[11pt]{article}
\usepackage[utf8]{inputenc}
\usepackage[T1]{fontenc}
\usepackage{graphicx}
\usepackage{grffile}
\usepackage{longtable}
\usepackage{wrapfig}
\usepackage{rotating}
\usepackage[normalem]{ulem}
\usepackage{amsmath}
\usepackage{textcomp}
\usepackage{amssymb}
\usepackage{capt-of}
\usepackage{hyperref}
\usepackage{minted}
\usepackage[hyperref,x11names]{xcolor}
\usepackage{physics}
\usepackage{cases}
\graphicspath{ {./} }
\usepackage{tikz}
\usetikzlibrary{arrows,plotmarks,calc,positioning,fit}
\usetikzlibrary{shapes.geometric, decorations.pathmorphing, patterns, backgrounds}
\newcommand{\tikzremember}[1]{{  \tikz[remember picture,overlay]{\node (#1) at (0,11pt) { };}}}
\tikzset{snake it/.style={decorate, decoration=snake}}
\usepackage[nottoc]{tocbibind}
\date{\today}
\title{}
\hypersetup{
 pdfauthor={},
 pdftitle={},
 pdfkeywords={},
 pdfsubject={},
 pdfcreator={Emacs 25.3.1 (Org mode 9.2.4)}, 
 pdflang={English}}
\begin{document}

\tableofcontents

PLAN 

\begin{itemize}
\item Start with the theory of DFT with the HK theorem and KS equations to solve
for the density.
\item Talk a bit about the approximations that are made.
\item Alloy theory and the structure of Ti3Al
\item Look at literature to do with all of this. Go through why solutes are
incredibly important and why it is necessary do to this research.
\item How can solutes lead to failure and why is oxygen in particular bad for the alloy.
\item Is it wavy or planar slip
\item Go into the calculations and what I have done for them
\item Cite the research about the Fe and C vacancy concentration and what the
implications are for alloy research
\item What can I add to the research.
\end{itemize}

\section{Introduction}
\label{sec:org07bfc2c}

Since the discovery that quanta may accurately describe the phenomena of black body radiation,
it has been known that atoms are governed by a different physical paradigm to that of matter on
the macroscopic scale: quantum mechanics. To model the behaviour of atoms in a solid, we can
use quantum mechanics.

The first postulate of of quantum mechanics states that a quantum mechanical system can be
described by a wavefunction, a function of the positions of the quantum mechanical entities,
and time, which satisfies the Schr$\backslash$:odinger equation. Bonding in materials is heavily dependent
on how electrons arrange themselves upon the assembly of atoms to make solid. By the first postulate we
would expect upon assembly, the state of a the system can be fully described by a \emph{many body
wavefunction}: a wavefunction which is a function of the positions of each of the electrons and
nuclei in the system.

Assuming that we have non-relativistic electrons, the Schr$\backslash$:odinger equation one must solve is then 


\[( \sum_{i} - \frac{1}{2} \nabla_{i}^{2} + \frac{1}{2} \sum_i\sum_j    \frac{1}{| \mathbf{r}_i - \mathbf{r}_j |}+ \sum_i\sum_I \frac{-Z_I}{| \mathbf{r}_i - \mathbf{R}_I |} ) \mathbf{ \Psi } = E \mathbf{ \Psi },\]


where we have used Hartree units ( \(e = m = \hslash = 4\pi\epsilon_0 = 1\) ). 

Quite often, the state we would most like to find is the lowest energy state (\emph{ground state})
of the system, as this is the fundamental state from which other mechanisms return to or start
from. This necessarily occurs at zero kelvin by the third law of thermodynamics. To find this
state for the full system, one must minimise the many-body wavefunction state that satisfies
the Schr$\backslash$:odinger equation with respect to energy. This quickly leads to a problem of
computation. Each electron and nucleus each have three spatial coordinates, so even given a
collection of a few atoms, we would struggle in being able to accurately determine the ground
state. So we can make a few approximations. 

\subsection{Born-Oppenheimer Approximation}
\label{sec:orge10daf6}

The first approximation that we make is the Born-Oppenheimer approximation, which allows us to
separate nuclear and electronic motion. One must first make the assumption that the
wavefunction which describes the system is a product state between the nuclear and electronic
portions of the system.

\[ \psi(\mathbf{R}, \mathbf{r}) = \Phi(\mathbf{R}) \cdot \psi_{\mathbf{R}}(\mathbf{r}), \]
where \(\psi_{\mathbf{R}}(\mathbf{r})\) is the electronic wavefunction and \(\Phi(\mathbf{R})\) is
the nuclear portion of the wavefunction.

 Due to the large disparity in the masses of the
electron compared to the nucleus (\(\sim 2000 m_e\)) we can neglect the contribution to the
Hamiltonian that comes from the nuclear kinetic energy operator acting on the electronic
wavefunction, as the resulting term is far smaller than the electronic kinetic energy operator acting on the
electronic wavefunction.

\[T_{\text{Nuclear}}(\mathbf{R}) \psi_{\mathbf{R}}(\mathbf{r}) = \sum_{I =
1}^{M} - \frac{1}{2M_{I}} \nabla^{2}_{\mathbf{R}_{I}} \psi_{\mathbf{R}}(\mathbf{r})\]


\[T_{\text{Electronic}}(\mathbf{r}) \psi_{\mathbf{R}}(\mathbf{r}) = \sum_{i =
1}^{M} - \frac{1}{2m_{i}} \nabla^{2}_{\mathbf{r}_{i}}
\psi_{\mathbf{R}}(\mathbf{r}) \]

Due to this, we can say that the motion of the electrons is instantaneous with regard to the
motion of the ions in the system and that the electrons relax into their ground state with
respect to any configuration of the ions. This means that the total energy is a function of
only the nuclear coordinates \cite{Finnis1997}. 

To obtain the total energy, we can solve for the electronic part of the wavefunction and then
solve for the nuclear wavefunction,

\[ \Big( T_{\text{Nuclear}} + E_{\mathbf{R}} \Big) \Phi(\mathbf{R}) = E_{\text{Total}} \Phi(\mathbf{R}),\]

where \(E_{\mathbf{R}}\) contains all of the information of the configuration of the electrons in
the system. 

\subsection{Self-Consistent Mean-Field Theory}
\label{sec:org0c54aa8}

\subsection{Density Functional Theory}
\label{sec:orgf3dccb3}

\subsubsection{Theory}
\label{sec:org380e90e}

Now that the electronic motion has been decoupled from that of the nucleus, we can now try to
find the ground-state solution for a given configuration of ions. A computationally tractable
way of doing this for a many-body solid is via the \emph{Density Functional Theory}. 

The main tenet of Density Functional Theory is: given a system that consists of \(N\) electrons,
there exists an energy functional \(E[\rho(\mathbf{r})]\), which is a functional of a local
electron density \(\rho(\mathbf{r})\); upon minimising the energy with respect the changes in the
electron density, subject to the constraint that the number of electrons is conserved, \[
\int_{\text{all space}} \rho(\mathbf{r}) = N, \] the resulting density is unique and
corresponds to the ground-state energy.

A functional is an entity that maps a function to a value, similar as how a function maps a
variable to a value. The Hohenburg-Kohn theorem states that there exists a functional of the
electron density which has the correct ground state energy upon minimising with respect to
energy \cite{hohenburg64_inhomog_electron_gas}. 

To actually find this density, one can use the Kohn-Sham equations to find a self-consistent
solution for the electron density \cite{kohn65_self_cons_eq}. 

The process by which this happens are as follows: one solves for the Hartree potential first with a
given input density (the solution of Poisson's equation), then one finds the total effective potential
for the system which is the sum of the Hartree potential, the potential from the nuclei
(\(V_{\text{ext}}\)) and the exchange-correlation potential \(V_{\text{xc}}\). The Schr$\backslash$:odinger
equation is subsequently solved, and a new electron density is found. This density can be





\subsubsection{Practical steps towards accurate calculations}
\label{sec:orgf91dd60}



\section{Defects in Materials}
\label{sec:orgce39476}

\subsection{Vacancies and Solutes}
\label{sec:org2eb0d84}

\subsection{Ti3Al Solutes and their effects.}
\label{sec:orgac0ce46}

\subsection{Current research: Vacancy-Solute Complexes.}
\label{sec:orgdfded82}


\section{Bibliography}
\label{sec:org44b2d18}
\label{org9d4da5b}

\bibliographystyle{unsrt}
\bibliography{../bibliography/org-refs}
\end{document}
