% Created 2019-07-22 Mon 09:37
% Intended LaTeX compiler: pdflatex
\documentclass[11pt]{article}
\usepackage[utf8]{inputenc}
\usepackage[T1]{fontenc}
\usepackage{graphicx}
\usepackage{grffile}
\usepackage{longtable}
\usepackage{wrapfig}
\usepackage{rotating}
\usepackage[normalem]{ulem}
\usepackage{amsmath}
\usepackage{textcomp}
\usepackage{amssymb}
\usepackage{capt-of}
\usepackage{hyperref}
\usepackage{minted}
\usepackage[hyperref,x11names]{xcolor}
\usepackage{physics}
\usepackage{cases}
\graphicspath{ {./} }
\usepackage{tikz}
\usetikzlibrary{arrows,plotmarks,calc,positioning,fit}
\usetikzlibrary{shapes.geometric, decorations.pathmorphing, patterns, backgrounds}
\newcommand{\tikzremember}[1]{{  \tikz[remember picture,overlay]{\node (#1) at (0,11pt) { };}}}
\tikzset{snake it/.style={decorate, decoration=snake}}
\usepackage[nottoc]{tocbibind}
\author{Tigany Zarrouk}
\date{\today}
\title{DFT Theory and Application to Defect Cluster Concentration}
\hypersetup{
 pdfauthor={Tigany Zarrouk},
 pdftitle={DFT Theory and Application to Defect Cluster Concentration},
 pdfkeywords={},
 pdfsubject={},
 pdfcreator={Emacs 25.3.1 (Org mode 9.2.4)}, 
 pdflang={English}}
\begin{document}

\maketitle
\tableofcontents

PLAN 

\begin{itemize}
\item Start with the theory of DFT with the HK theorem and KS equations to solve
for the density.
\item Talk a bit about the approximations that are made.
\item Alloy theory and the structure of Ti3Al
\item Look at literature to do with all of this. Go through why solutes are
incredibly important and why it is necessary do to this research.
\item How can solutes lead to failure and why is oxygen in particular bad for the alloy.
\item Is it wavy or planar slip
\item Go into the calculations and what I have done for them
\item Cite the research about the Fe and C vacancy concentration and what the
implications are for alloy research
\item What can I add to the research.
\end{itemize}

\section{Introduction}
\label{sec:org2541f29}

Since the discovery that quanta may accurately describe the phenomena of black body radiation,
it has been known that atoms are governed by a different physical paradigm to that of matter on
the macroscopic scale: quantum mechanics. To model the behaviour of atoms in a solid, we can
use quantum mechanics.

The first postulate of of quantum mechanics states that a quantum mechanical system can be
described by a wavefunction, a function of the positions of the quantum mechanical entities,
and time, which satisfies the Schr$\backslash$:odinger equation. Bonding in materials is heavily dependent
on how electrons arrange themselves upon the assembly of atoms to make a solid. By the first postulate, we
would expect upon assembly, that the state of the system can be fully described by a \emph{many body
wavefunction}: a wavefunction which is a function of the positions of each of the electrons and
nuclei in the system.

Assuming that we have non-relativistic electrons, the Schr$\backslash$:odinger equation one must solve is then 


\[\Big( \sum_{i} - \frac{1}{2} \nabla_{i}^{2} + \frac{1}{2} \sum_i\sum_j    \frac{1}{|
\mathbf{r}_i - \mathbf{r}_j |}+ \sum_i\sum_I \frac{-Z_I}{| \mathbf{r}_i - \mathbf{R}_I |} \Big)
\mathbf{ \Psi }(\{\mathbf{r}\}; t) = E \mathbf{ \Psi }(\{\mathbf{r}\}; t),\]


where \(\mathbf{ \Psi }(\{\mathbf{r}\}; t)\) is an eigenstate and Hartree units have been used( \(e = m = \hslash = 4\pi\epsilon_0 = 1\) ).

A time-independent observable is the expectation value of a given operator, 

\[ \expval{\hat{O}} = \frac{\expval{\hat{O}}{\mathbf{ \Psi }} }{ \bra{\mathbf{ \Psi}}\ket{\mathbf{ \Psi }} }, \]
this is an integral over all of the coordinates. 

The electronic density operator is defined as 

\[ \hat{\rho}(\mathbf{r}) = \sum_{i=1}^{N} \delta ( \mathbf{r} - \mathbf{r}_{i} ) \]

\[ \rho(\mathbf{r}) = \frac{\expval{\hat{n}(\mathbf{r})}{\mathbf{ \Psi }} }{ \bra{\mathbf{\Psi}}\ket{\mathbf{ \Psi }} }. \]

The total energy is the expectation value of the Hamiltonian:

\begin{align}
 E &= \frac{\expval{\hat{H}}{\mathbf{ \Psi }} }{ \bra{\mathbf{\Psi}}\ket{\mathbf{ \Psi }}}\\
   &= \expval{\hat{T}} + \expval{\hat{V}_{\text{int}}} + \int \text{d}^3 V_{\text{ext}} \rho(\mathbf{r}) + E_{II}, 
\label{eq:totalenergy}
\end{align}

where 

\[ V_\text{ext}(\mathbf{r}) = \sum_{I} -\frac{eZ_{I}}{|\mathbf{r} - \mathbf{R}_I|} \label{eq:externaleZpotential}$\] 

is the external potential due to the coulomb interaction acting on the electrons
from the nuclei---the expectation value has been explicitly written as an integral over the
local electron density, \(E_{II}\) is the classical electrostatic nucleus-nucleus interaction
energy, \(V_{\text{int}}\) is the electron-electron interaction energy and \(\expval{\hat{T}}\) is
the expectation value of the electronic kinetic energy. 

Stationary points in the total energy correspond to eigenstates of the many-body Hamiltonian. One can
vary the ratio, or the numerator, in \eqref{eq:totalenergy}. The latter must be subjected
to the constraint of orthonormality (\(\bra{\mathbf{\Psi}}\ket{\mathbf{ \Psi }} = 1\)), which is
possible with the use of Lagrange multipliers. One finds that upon variation of the bra
\(\bra{\mathbf{\Psi}}\) that the ket must satisfy the time-independent Schr$\backslash$:odinger equation:

\[ \hat{H}\ket{\mathbf{\Psi}_m} = E \ket{\mathbf{\Psi}}, \]
where \(\ket{\mathbf{\Psi}_m}\) is an eigenstate.

Quite often, the state we would most like to find is the lowest energy state (\emph{ground state})
of the system, as this is the fundamental state from which other mechanisms return to or start
from. This necessarily occurs at zero kelvin by the third law of thermodynamics. To find this
state for the full system, one must minimise the energy with respect to the parameters of the
many-body wavefunction that satisfies the Schr$\backslash$:odinger equation and appropriate symmetry
constraints (e.g. for electrons, \(\mathbf{\Psi}\) must be antisymmetric). This quickly leads to
a problem of computation: each electron has four degrees of freedom (the quantum numbers \(n, m,
l\) and \(s\)), and each nucleus will have \(n^d_I\) degrees of freedom; given a collection of
just \(M\) atoms, we would have \(4M + Mn^d_I\) variables to minimise with respect to energy. To
make the problem tractable, we must make a few approximations.

\subsection{Born-Oppenheimer Approximation}
\label{sec:org8f6526e}

The first approximation that we make allows us to separate nuclear and electronic motion: this is the
Born-Oppenheimer approximation. One must first make the assumption that the
wavefunction which describes the system is a product state between the nuclear and electronic
portions of the system.

\[ \psi(\mathbf{R}, \mathbf{r}) = \Phi(\mathbf{R}) \cdot \psi_{\mathbf{R}}(\mathbf{r}), \]

where \(\psi_{\mathbf{R}}(\mathbf{r})\) is the electronic wavefunction and \(\Phi(\mathbf{R})\) is
the nuclear portion of the wavefunction.

Due to the large disparity in mass between an electron and the nucleus of an atom (\(M_\text{Nuc} \sim
2000 m_e\)) we can neglect the contribution to the Hamiltonian that comes from the nuclear
kinetic energy operator acting on the electronic wavefunction, as the resulting term is far
smaller than the electronic kinetic energy operator acting on the electronic wavefunction.

\[T_{\text{Nuclear}}(\mathbf{R}) \psi_{\mathbf{R}}(\mathbf{r}) = \sum_{I =
1}^{M} - \frac{1}{2M_{I}} \nabla^{2}_{\mathbf{R}_{I}} \psi_{\mathbf{R}}(\mathbf{r})\]


\[T_{\text{Electronic}}(\mathbf{r}) \psi_{\mathbf{R}}(\mathbf{r}) = \sum_{i =
1}^{M} - \frac{1}{2m_{i}} \nabla^{2}_{\mathbf{r}_{i}}
\psi_{\mathbf{R}}(\mathbf{r}) \]

This results in the motion of the electrons being instantaneous with regard to the
motion of the ions in the system and that the electrons relax into their ground state with
respect to any configuration of the ions. This results in the total energy being a function of
only the nuclear coordinates \cite{Finnis1997}. 

To obtain the total energy, we can solve for the electronic part of the wavefunction and then
solve for the nuclear wavefunction,

\[ \Big( T_{\text{Nuclear}} + E_{\mathbf{R}} \Big) \Phi(\mathbf{R}) = E_{\text{Total}} \Phi(\mathbf{R}),\]

where \(E_{\mathbf{R}}\) contains all of the information of the configuration of the electrons in
the system. 

\subsection{Self-Consistent Mean-Field Theory}
\label{sec:orga817c17}

\subsection{Density Functional Theory}
\label{sec:orgad6ec01}

\subsubsection{Theory}
\label{sec:org3cb8a06}

Now that the electronic motion has been decoupled from that of the nucleus, we can now try to
find the ground-state solution for a given configuration of ions. A computationally tractable
way of doing this for a many-body solid describing all of the electrons is via the \emph{Density
Functional Theory}. 

Hohenburg and Kohn's seminal paper in 1964 proved that there exists an energy functional of the
electron density which can provide the exact ground-state energy and density upon
minimisation. This reduces the number of variables from \(3N\) to that of only 3 for any number
of electrons \(N\) in the system of interest.

In 1965, Kohn and Sham then developed a formalism to practically calculate this ground-state density via
a set of self-consistent equations: the Kohn-Sham equations. 

The main tenet of Density Functional Theory is: given a system that consists of \(N\) electrons,
there exists an energy functional of the local electron density \(E[\rho(\mathbf{r})]\), which
corresponds to an antisymmetric wavefunction \(\ket{\Psi}\); upon minimisiation by
variation of the electron density, subject to the constraint that the number of electrons is
conserved, \[ \int_{\text{all space}} \rho(\mathbf{r}) = N, \] the resulting density is unique
and the value of the functional is the ground-state energy. To obtain this result we can go
through the following theory.

Assuming \(\rho(\mathbf{r})\) is the exact ground state density, and \(\ket{\Psi}\)
is the ground-state, by the variational principle we can write, 

\[ \bra{\Psi} \mathcal{\hat{T}} + \frac{1}{2} \sum_i\sum_j \frac{1}{| \mathbf{r}_i -
\mathbf{r}_j |}  \ket{\Psi} + \int \rho(\mathbf{r}) V_{\text{ext}}(\mathbf{r})
\text{d}\mathbf{r} \geq E_0. 
\label{eq:dftvarprinciple} \]

The state \(\bra{\Psi}\) may not be unique. To make it so, one can subject equation \ref{eq:dftvarprinciple}
to the constraints that the number of electrons is conserved and that \(\rho\) is constant. 

This defines the functional: 

\[ F[\rho] = \underset{\Psi \rightarrow \rho}{\text{min}} \bra{\Psi} \mathcal{\hat{T}} +
\frac{1}{2} \sum_i\sum_j \frac{1}{| \mathbf{r}_i - \mathbf{r}_j |}  \ket{\Psi} = T[\rho] +
E_{ee}[\rho],   \]

where the notation \(\Psi \rightarrow \rho\) is to show that the minimisation is with respect to
all \(\Psi\) that can make the density \(\rho\). 

One can see the functional with the minimum value as the ground-state energy is then 

\[ E[\rho] = F[\rho] + E_{\text{ext}}[\rho], \]

where \[ E_{\text{ext}}[\rho] = E_{\text{ext}}[\rho(\mathbf{r})] =  \int \rho(\mathbf{r})
V_{\text{\text{ext}}}(\mathbf{r}), \]
of which its functional derivative is \(V_{\text{ext}}\).

By the Euler-Lagrange equations, one finds that the ground state must satisfy 

\[ \frac{\delta F[\rho]}{\delta \rho(\mathbf{r}) } + V_{\text{ext}}= \mu. \]

A functional is an entity that maps a function to a value, similar as how a function maps a
variable to a value. The Hohenburg-Kohn theorem states that there exists a functional of the
electron density which has the correct ground state energy upon minimising the energy with respect to
the electron density \cite{hohenburg64_inhomog_electron_gas}. Thus all the information of the
system is contained in the electron density, which reduces the minimisation
problem fron \(4M\) variables to one of just 3 for any number of electrons in the system. 

To actually find this density, one can use the Kohn-Sham equations to find a self-consistent
solution for the electron density \cite{kohn65_self_cons_eq}. 




To actually find the eigenvalues, one can replace the problem of solving a fully-interacting
electronic system with a given electronic density with an auxiliary non-interacting electronic
system which has the same electronic density. The resulting eigenvalues can be used to find the
expectation value of the kinetic energy functional, \(T_s[\rho(\mathbf{r})]\).

The Hohenburg-Kohn-Sham functional can be defined as

\[
E^{\text{HKS}}[\rho] = T_{\text{s}}[\rho] + E_{\text{H}}[\rho] + E_{\text{xc}}[\rho] + E_{\text{ext}}[\rho] + E_{\text{ZZ}},
\]

where \(T_{\text{s}}[\rho]\) is the kinetic energy of the fictitious non-interacting auxiliary system
acting in the same effective potential \(V_{\text{eff}}[\rho]\). The assumption made here is that the
ground state density of the non-interacting, auxiliary system is equal to that of the system with
full electronic interactions. 

This definition of the functional redefines the exchange-correlation functional: the difference
between the true kinetic energy and that of the non-interacting system is added to it. Such that the
true exchange-correlation functional has the form of

\[
E_{\text{xc}}[\rho] = ( \expval{\hat{T}} - T_{\text{s}}[\rho] ) + ( \expval{\hat{V}_{\text{int}}} - E_{\text{H}}[\rho])
\]

where we can interpret the first term as being the increase in kinetic energy from electronic
correlation in a fully interacting system, compared to a non-interacting one---correlations cause
electrons to move to more energetically favourable areas of the potential---and the second term is
the change in the potential of a fully interacting system, with exchange and correlation included,
and an electron density acting through the

The difference between the true kinetic energy \(\Delta T = T - T_{\text{s}}\) is now approximated by
the exchange-correlation functional \(E_{\text{xc}}[\rho]\). This is a reasonable
approximation. Separating the kinetic energy from the long-range coulomb interactions means that the
exchange-correlation potential can be approximated by an approximately local functional. Is this due
the the fact that correlations basically lead to a screening of the coulomb potential?


The process by which this happens are as follows: one solves for the Hartree potential
first with a given input density (the solution of Poisson's equation), then one finds the
total effective potential for the system which is the sum of the Hartree potential, the
potential from the nuclei (\(V_{\text{ext}}\)) and the exchange-correlation potential
\(V_{\text{xc}}\). The Schr$\backslash$:odinger equation is subsequently solved, and a new electron
density is found. This density can be put back into the Poisson's equation to find the
hartree potential and start the cycle again. These equations must be solved
self-consistently as the electron density that one puts into Poisson's equation is the
quantity that one solves for. Once the input and output densities are within some
tolerance of each other, then one can say that the \(\rho^{\text{out}}(\mathbf{r}) = \rho^{\text{exact}}(\mathbf{r})\),
and the resulting Kohn-Sham eigenvalues are the ground state energies. 

The Kohn-Sham eigenvalues are not strictly correct. 


\subsubsection{Kohn-Sham Equations and Self-Consistency}
\label{sec:orgca19915}


\subsubsection{Practical steps towards accurate calculations}
\label{sec:org197c46a}



\section{Defects in Materials}
\label{sec:org8207af3}

\subsection{Vacancies and Solutes}
\label{sec:org77dbb0f}

\subsection{Ti3Al Solutes and their effects.}
\label{sec:orgf10c21b}

\subsection{Current research: Vacancy-Solute Complexes.}
\label{sec:org7f7b8a7}


\section{Bibliography}
\label{sec:org8f96ad0}
\label{org5fc5d83}

\bibliographystyle{unsrt}
\bibliography{../bibliography/org-refs}
\end{document}
