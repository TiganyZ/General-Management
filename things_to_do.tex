% Created 2019-02-19 Tue 16:49
\documentclass[11pt]{article}
\usepackage[utf8]{inputenc}
\usepackage[T1]{fontenc}
\usepackage{fixltx2e}
\usepackage{graphicx}
\usepackage{longtable}
\usepackage{float}
\usepackage{wrapfig}
\usepackage{rotating}
\usepackage[normalem]{ulem}
\usepackage{amsmath}
\usepackage{textcomp}
\usepackage{marvosym}
\usepackage{wasysym}
\usepackage{amssymb}
\usepackage{hyperref}
\tolerance=1000
\usepackage{minted}
\usepackage[hyperref,x11names]{xcolor}
\usepackage{physics}
\usepackage{cases}
\graphicspath{ {./} }
\usepackage{tikz}
\usetikzlibrary{arrows,plotmarks,calc,positioning,fit}
\usetikzlibrary{shapes.geometric, decorations.pathmorphing, patterns, backgrounds}
\newcommand{\tikzremember}[1]{{  \tikz[remember picture,overlay]{\node (#1) at (0,11pt) { };}}}
\tikzset{snake it/.style={decorate, decoration=snake}}
\usepackage[nottoc]{tocbibind}
\author{Tigany Zarrouk}
\date{\today}
\title{things-to-do}
\hypersetup{
  pdfkeywords={},
  pdfsubject={},
  pdfcreator={Emacs 25.3.1 (Org mode 8.2.10)}}
\begin{document}

\maketitle
\tableofcontents

\section{Electronic Structure}
\label{sec-1}
\subsection{Tight Binding}
\label{sec-1-1}
\subsubsection{Tight Binding Band Model}
\label{sec-1-1-1}

As with all tight-binding models, they are based on a linear combination of atomic orbitals basis. 
We compose our hamiltonian as a matrix that is in this basis, where each column/row corresponds to a particular
atom's orbital.
The \emph{on-site} terms are on the diagonal as these simply correspond to the energy of a particular orbital
e.g.$\epsilon_s$, $\epsilon_p$, $\epsilon_d$, which are also parameters along with the assumed orbital occupancies 
of the free atoms $N_{\mathbf{R}\ell}$.
The other terms are off diagonal terms which either represent the bonding between different atoms, if it is a matrix element
between two different atomic sites, or if the terms are on the the same atomic site, this represents an interaction
between orbitals on the same atom. These are crystal field terms and relate to the distortion of sperical symmetry arising from
interaction of orbitals on the same site, which causes a splitting of energy levels. 
In the tight binding model these come from the self-consistent polarisable tight-binding 
model. If the cutoffs between bond integrals is short, then the Hamiltonian matrix becomes sparse, 
which means that diagonalisation is quick. 

For the band model we have the density matrix composed in the normal way. 
$$ \hat{\rho} = \sum_{n} f_{n} \ket{n} \bra{n}$$ and it is straight forward to see that this density matrix satisfies 
$$ \hat{\rho} = N $$ where $N$ is the number of electrons in this system. 

The band energy is then given by $$E_{\text{band}} = \text{Tr}\hat{\rho}\hat{H} = \sum_{n} f_{n}\epsilon_{n} $$, where $\epsilon_{n}$
is the energy of an eigenstate. This band energy is essentially the sum of the electron kinetic energies and electron-ion interaction energies. 

A pair potential is added to this to account for the electron-electron and ion-ion contributions to the total energy 
as given in density functional theory.
So the total energy is 

$$ E_{\text{tot}} =   E_{\text{band}} + E_{\text{rep}} $$


The band model is \emph{not} consistent with the force theorem, as the bond model is.
The force theorem states that there is no contribution to the force from self-consistent redistribution of charge 
as a result of the virtual displacement of an atom. 

There is no self-consistent charge redistribution in the band model.

If we move an atom in the band model, then there should be a change in the band structure energy from electron-electron interaction. 
However, all electron-electron interactions are controlled by the pair-potential, as per its definition so there is \emph{not} exact 
cancellation. 

This is exactly cancelled in DFT by the double counting term. 


\subsubsection{Tight Binding Bond Model}
\label{sec-1-1-2}

\begin{enumerate}
\item Paxton: Implementation by Diagonalisation.
\label{sec-1-1-2-1}
This follows from \cite{Paxton:153084}

In the TBBM we use the \emph{covalent bond energy} as the fundamental quantity of interest that we want to calculate.
To obtain this energy, one simply ignores the diagonal terms when taking the trace over the product of the density matrix and
the Hamiltonian (when we calculate the band energy):)

\begin{export}
\[ 
E_{\text{bond}} = \frac{1}{2} \sum_{\mathbf{R}L\mathbf{R}'L'//\mathbf{R}/neq\mathbf{R}'}
                             2\rho_{\mathbf{R}L\mathbf{R}'L'} H^{0}_{\mathbf{R}'L'\mathbf{R}L}.
\]
\end{export}


The terms we have excluded from the band energy are now grouped with the corresponding quantities in the free atom. 
This gives rise to the \emph{promotion energy}.

\begin{export}
\begin{align}
E_{\test{prom}} &= \sum_{\mathbf{R}L} \Big(\rho_{\mathbf{R}L\mathbf{R}'L'}H^{0\mathbf{R}'L'\mathbf{R}L} - N_{\mathbf{R}\ell}\epsilon_{\mathbf{R}\ell} \Big)\\
              &= \sum_{\mathbf{R}L} \Big(\rho_{\mathbf{R}L\mathbf{R}'L'} - N_{\mathbf{R}\ell} \Big) \epsilon_{\mathbf{R}L}\\
              &= \sum_{\mathbf{R}L} \Delta q_{\mathbf{R}L} \epsilon_{\mathbf{R}\ell}
\end{align}
\end{export}

\begin{export}
\begin{align}
E_{\test{prom}}^{\text{TBBM}} &= \sum_{\mathbf{R}L} \Big(\rho_{\mathbf{R}L\mathbf{R}'L'} - N_{\mathbf{R}\ell} \Big) H_{\mathbf{R}L\mathbf{R}L}\\
              &= \sum_{\mathbf{R}L} \Delta q_{\mathbf{R}L}( \epsilon_{\mathbf{R}\ell} + \Delta \epsilon_{\mathbf{R}\ell} )
\end{align}
\end{export}

Where the promotion energy represents the preparation of an atom going from the free state to that of one about to undergo bonding. 
E.g. the energy levels of particular atoms being prepared to undergo hybridisation.


The band model \emph{is} consistent with the force theorem as there is  self-consistent redistribution of charges. 
In the \emph{band} model there is no contribution to the forces from the promotion energy as 
\$ $\epsilon$$_{\mathbf{R}\text{L}}$$\delta$ q = $\delta$$\epsilon$\{\mathbf{R}L\} q = 0 \$ as the eigenvalues do not change: they are constants. 

However the Mulliken charge transfers are not necessarily zero, and the force theorem necessitates contributions due to  
electrostatic charge transfer to be included in the calculation of the interatomic force. This means that the band model is 
inconsistent due to its neglect of these terms. 

The self-consistency that is imposed in the TBBM bond model is that of \emph{local charge neutrality}. This means that the total charge 
on a particular site should equal that of the reference atom. This means that the \emph{total} Mulliken charge difference between that and
free atoms is \emph{zero}. 

This self consistency is achieved by adjusting the on-site orbital energies. This only affects the diagonal of the Hamiltonian 
so only $E_{\text{prom}}$ changes. 

This means there is an additional contribtion to the force on atom $\mathbf{R}$ which comes from $\sum_{L} \Delta q_{L} \Delta \epsilon_{\ell}$. 

If all orbital energies are shifted by the same amount at each site, during self-consistency, then this term vanishes as 
$\epsilon_{\ell}$ is independent of $L$---as $L$ is the index which corresponds to angular momentum $nlm$. 
So the term moves to the front of the summation, and as the sum of all changes in charges are zero, by local charge neutrality,
 then the term vanishes. 

\item Sutton 1988: The Tight-Binding Bond Model
\label{sec-1-1-2-2}
\cite{Sutton1988}

\begin{enumerate}
\item General Formulation
\label{sec-1-1-2-2-1}
In the original paper of the tight-binding bond model, Sutton \emph{et al} obtain
the tight-binding bond model from the Harris-Foulkes functional.

If we say that the single particle potential of the Kohn-Sham equations is 
\[
\widetilde{V}(\mathbf{r}) = v(\mathbf{r}) + V^{\text{f}}(\mathbf{r}) + \mu_{\text{xc}}(\mathbf{r})
\]

where 
\begin{itemize}
\item $v(\mathbf{r})$ is the total ionic potential
\item $V^{f}(\mathbf{r})$ is the Hartree potential (self-energy of the electron density.)
\item $\mu_{xc}(\mathbf{r})$ is the exchange-correlation potential.
\end{itemize}

Then we can solve for the Hamiltonian once, and an output charge density $\rho^{\text{out}}$ is
constructed from its eigenstates, without any self-consistency iterations.  

\[
\widetilde{H} = -\frac{1}{2} \nabla^{2} + \widetilde{V}(\mathbf{r})
\]



The Harris-Foulkes functional, which exploits the variational principle, gives 
the leading corrections to the total energy are second order in the difference 
between $\rho^{\text{f}}$, the input charge density and the exact charge density $\rho^{\text{sc}}$. 

\begin{align}
E = \sum_{n} f_{n} \tilde{\varepsilon_{n}} &- int \text{d}r \rho^{\text{f}}(\mathbf{r})
               \big( \frac{1}{2} V^{\text{f}}(\mathbf{r}) + \mu_{xc}^{\text{f}}(\mathbf{r})  \big)
               + E_{\text{xc}}[\rho^{\text{f}}] + E_{\text{ii}}\\
               &+ \mathcal{O}(\rho^{\text{sc}} - \rho^{\text{f}})^2 + \mathcal{O}(\rho^{\text{sc}} - \rho^{\text{out}})^2
\end{align}

where,
\begin{itemize}
\item $\rho^{\text{f}}$ is an input charge density which is formed
from a superposition of isolated atomic charge densities. This if formed
from condensing atoms infinitely far apart without allowing the atomic
charge densities to change.

\item $\rho^{\text{out}}$ is the charge
density formed from the \emph{eigenstates} of the Hamiltonian.

\item $\rho^{\text{sc}}$ is the exact ground state charge density

\item $f_{n}$ is the occupation number

\item $\tilde{\varepsilon}_{n}$ is an eigenvalue of the single-particle
Hamiltonian.

\item $E_{\text{ii}}$ is the ion-ion interaction energy.

\item $\Delta E_{xc}[\rho^{\text{f}}]$  which is the change in the exchange-correlation energy in forming the charge
density $\rho^{\text{f}}$.
\end{itemize}

One obtains the binding energy as 
\[
E_{B} = \text{Tr}( \rho^{\text{out}} - \rho^{\text{f}} )\widetilde{H} + \Delta E_{es}[\rho^{\text{f}}] + \Delta E_{xc}[\rho^{\text{f}}]
\]

Where 
\begin{itemize}
\item $\rho^{\text{f}}$ is an input charge density which is formed
from a superposition of isolated atomic charge densities. This if formed
from condensing atoms infinitely far apart without allowing the atomic
charge densities to change.

\item $\rho^{\text{out}}$ is the charge
density formed from the \emph{eigenstates} of the Hamiltonian.

\item $\Delta E_{es}[\rho^{\text{f}}]$ is the change in electrostatic energy
of all valence electrons and ion cores when the atoms are condensed from
infinity to make the solid---which included electron-electron, electron-ion
and ion-ion electrostatic interactions

\item $\Delta E_{xc}[\rho^{\text{f}}]$  which is the change in the exchange-correlation energy in forming the charge
density $\rho^{\text{f}}$.
\end{itemize}

They express $\text{Tr}(\rho^{\text{out}} - \rho^{\text{f}})\widetilde{H}$ in the atomic
orbital representation as a sum over on-site terms and a sum over inter-site
terms. 

$\backslash$[
\text{Tr}($\rho$$^{\text{out}}$ - $\rho$$^{\text{f}}$)\widetilde{H} =
   \underset\{\text{Promotion Energy}\}\{ 
      $\sum$$_{\text{i\\ }\alpha\beta}$ 
         \big[ ($\rho$$^{\text{out}}$)$^{\text{i}\alpha\ \text{i}\beta}$ -
               ($\rho$$^{\text{f}}$)$^{\text{i}\alpha\ \text{i}\beta}$ $\delta$$^{\text{i}\alpha}_{\text{i}\beta}$ \big]
         \widetilde{H}$_{\text{i}\beta\ \text{i}\alpha}$ 
            \}
\begin{itemize}
\item \underset{Covalent Bond Energy}\{
\underset{i \neq j}\{ $\sum$$_{\text{i }\alpha}$$\sum$$_{\text{j }\beta}$ \}
   ($\rho$$^{\text{out}}$)$^{\text{i}\alpha\ \text{i}\beta}$ \widetilde{H}$_{\text{j}\beta\ \text{i}\alpha}$
          \}
\end{itemize}
$\backslash$]

In ther first term, in a perfect cubic crystal, 
$\widetilde{H}_{i\alpha i\beta} = \widetilde{H}_{i\alpha i\beta}
\delta^{i\beta}_{i\alpha}$.
The diagonal elements $\widetilde{H}_{i\alpha i\alpha}$ are the free atomic
term values corrected by the crystal-field terms in the solid. So this first
term is the promotion energy: the enrgy associated with the change of
occupancy of the atomic orbitals on forming the solid from free atoms. 

if the atomic environment around each site in the solid is distorted then
$\widetilde{H}_{i\alpha i\beta}$ are non-zero because of crystal-field terms.

It is possible to disgonalise the part of the Hamiltonian $\widetilde{H}$
associated only with site $i$ and thys express the terms involving site $i$ in
this equation, but the new diagonal elements of $\widetilde{H}$ will vary from site
to site.

The second term is the covalent bond energy of the solid and it si equal to
the sum of the covalent energies of individual bonds between orbitals on
different atoms. This bond energy is part of the band energy of the solid

\begin{align}
E_{\text{cov}} &= \frac{1}{2}  \underset{i \neq j}{ \sum_{i \alpha}\sum_{j \beta} }
         (2 \rho^{\text{out}})^{i\alpha i\beta} \widetilde{H}_{j\beta i\alpha}\\
               &= E_{\text{band}} - 
              \sum_{i\\ \alpha\beta} 
                   (\rho^{\text{out}})^{i\alpha i\beta}
                    \widetilde{H}_{j\beta i\alpha}
\end{align}


They further argued that the terms $\Delta E_{xc}[\rho^{\text{f}}]$ 
$\Delta E_{es}[\rho^{\text{f}}]$ can be approximated by a repulsive pair potential
centred at atomic sites.

So then the binding energy can then be expressed as a sum of bond energies and
promotion energies, where each bond energy is a sum of the covalent energy of
the bond and the pair potential interaction. 

\item Interatomic forces
\label{sec-1-1-2-2-2}

The forces on each atom can be found by differentiating the binding energy
with respect to $\mathbf{r}$, using the relation between the density matrix
and the Green's function,
\[
(\rho^{\text{out}})^{i\alpha i\beta} = -\frac{2}{\pi} \Im \int^{E_{\text{F}}} 
   G^{i\alpha j\beta} (E^{+}) \text{d}E
\]
and 
\[
 -\frac{1}{\pi} \Im \sum_{i \alpha} H_{p\mu i\alpha} G^{i\alpha l\gamma}
     =  -\frac{1}{\pi} \Im E G_{p\mu}^{l\gamma}
\]
with the Hellmann-Feynman theorem, which has also been derived by
\cite{Foulkes1989}. 

So the derivative of the binding energy with respect to $x_k$ is 

$\backslash$[
\frac\{$\partial$ E$_{\text{B}}$\}\{$\partial$ x$_{\text{k}}$\} = 
   \frac{1}{2}$\sum$$_{\text{i}\alpha}$$\sum$$_{\text{j}\beta}$ 2($\rho$$^{\text{out}}$)$^{\text{i}\alpha}_{\text{j}\beta}$
       \frac\{$\partial$ H$^{\text{j}\beta}_{\text{i}\alpha}$\}\{$\partial$ x$_{\text{k}}$\}
\begin{itemize}
\item $\sum$$_{\text{i}\alpha}$ ($\rho$$^{\text{f}}$)$^{\text{i}\alpha\ \text{i}\alpha}$ 
                  \frac\{$\partial$ H$_{\text{i}\alpha\ \text{i}\alpha}$\}\{$\partial$ x$_{\text{k}}$\}
\item \frac{\partial}\{$\partial$ x$_{\text{k}}$\} \big($\Delta$ E$_{\text{xc}}$[$\rho$$^{\text{f}}$] + $\Delta$ E$_{\text{es}}$[$\rho$$^{\text{f}}$] \big )
\end{itemize}
$\backslash$]


This can be simplified with the following approximations:

\begin{enumerate}
\item The non-orthogonality of the atomic-orbital basis set may often be
neglected becasue the leading correction terms to the energy are \emph{second
order} in the overlap matrix.
\item Three-centre terms may be neglected because the leading three-centre
corrections to the energy are also of second order.
\item Assume that $\widetilde{H}_{i\alpha i\beta} = \widetilde{H}_{i\alpha i\alpha}
   \delta^{i\beta}_{i\alpha}$---i.e. the Hamiltoniam matrix elements between
\emph{different orbitals} on the same atom may be neglected.
\item Each atom may be assumed to remain charge neutral by varying the on-site
Hamiltonian matrix elements in such a way that the energy splitting
between different orbitals on the same atom are preserved.
\begin{itemize}
\item This approximation ensures that conributions to the force in the above
equation from on-site terms in ther first sum cancel those of the second sum
\item This leads to consistency with the force theorem.
\end{itemize}
\end{enumerate}

These approximations give the derivative of the binding energy to be:

$\backslash$[
\frac\{$\partial$ E$_{\text{B}}$\}\{$\partial$ x$_{\text{k}}$\} = 
   $\sum$$_{\text{j }\neq\ \text{k}}$$\sum$$_{\alpha\beta}$ 2($\rho$$^{\text{out}}$)$_{\text{k}\alpha\ \text{j}\beta}$
       \frac\{$\partial$ H$_{\text{j}\beta\ \text{k}\alpha}$\}\{$\partial$ x$_{\text{k}}$\}
\begin{itemize}
\item \frac{\partial}\{$\partial$ x$_{\text{k}}$\} \big($\Delta$ E$_{\text{xc}}$[$\rho$$^{\text{f}}$] + $\Delta$ E$_{\text{es}}$[$\rho$$^{\text{f}}$] \big )
\end{itemize}
$\backslash$]

\item Local Charge Neutrality
\label{sec-1-1-2-2-3}

The approximation to local charge neutrality is motivated by 
\begin{enumerate}
\item In a metal where there is perfect screening any excess charge associated
with an atom will be neutralised---similarly in a semiconductor where the
band gap is much smaller than the widths of the valence and conduction
bands the screening length is approximately the same as an inter-atomic
separation and hence local charge neutrality will be a good
approximation.
\item To obtain the correct bulk modulus of the solid when it is calculated by
the method of long waves or by homogeneous dilatation, the charge density
must be treated within a self-consistent scheme, and LCN is the simplest
assumption to make.
\item There would be long range Coulomb terms if local charge neutrality were
not required, in binary systems. Is is always possible to find a basis
where each atom is neutral.
\item Charge neutrality within the TBBM leads to an internally consistent
picture of the heats of formation of transition metal alloys.
\end{enumerate}

This local charge neutrality is achieved by varying the on-site terms of the
Hamiltonian matrix elements. 

\begin{enumerate}
\item Priester Ge-GaAs Justification
\label{sec-1-1-2-2-3-1}
From \cite{Priester1986} local charge neutrality was motivated in a Ge-GaAs
interface. 

All inter-atomic matrix are specified and the intra-atomic (on-site) terms are
adjusted to within an additive constant. These on-site terms become very
important when it comes to dealing with interfaces. 

\[
\Delta = \bar{ E }( \text{ GaAs } ) - \bar{ E }( \text{ Ge } )
\]

where $\bar{ E }( \text{ GaAs } )$, $\bar{ E }( \text{ Ge } )$ are average $sp^3$ energy 
(in the compound $\bar{E}$ is the average between the Ga and As $sp^3$
energies). These are the on-site terms. 

The charge disturbance is near the vicinity of the interface and it has been
shown that the screening length roughly $\sim 2\AA$, on the order of the
interplanar spacing. So one finds bulk values away from the interface. 

One wants to find this $\Delta$ so the valence-band maxima in each material
are positioned with respect to each other to obtain a band offset. 

There is only electron transfer between the two
interface planes. As there are two atoms per two-dimensional unit cell, the
electron transfer can be split into two components $\delta N_1$ from Ga to Ge
and $\delta N_2$ from As to Ge, where the difference is from the population of
the state compared to that of the bulk, which is the change in electron
concentration due to the formation of the interface. 

If all on-site terms of an atom $i$ are shifted by the same
amount $U_{I}$, from the reference state, proportional to the excess
population on the other atom then
\[
U_{I} = \sum_{j} \gamma_{ij} \delta N_{j}
\] 
where $\gamma_{ij}$ are the corresponding intra- and interatomic Coulomb
terms. 

The important point is that in the reference situation, where all intra-atomic (on-site)
terms take their bulk values, $\Delta = 0$, the quantities $\delta N_1$,
$\delta N_2$ do not exactly compensate. 

This means that there is a dipole layer with an average excess population of
$( \delta N_1 + \delta N_1)/2$ per atom in the Ge plane. The effect of this
dipole layer is to shife the average $sp^3$ level on one site with respect to
the other, i.e. to make the quantity $\Delta \neq = 0$. 

So local charge neutrality tells us that 

\[ 
\delta N_1(\Delta) + \delta N_2(\Delta) = 0
\]
\end{enumerate}
\end{enumerate}
\end{enumerate}
% Emacs 25.3.1 (Org mode 8.2.10)
\end{document}
