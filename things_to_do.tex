% Created 2019-03-18 Mon 15:28
% Intended LaTeX compiler: pdflatex
\documentclass[11pt]{article}
\usepackage[utf8]{inputenc}
\usepackage[T1]{fontenc}
\usepackage{graphicx}
\usepackage{grffile}
\usepackage{longtable}
\usepackage{wrapfig}
\usepackage{rotating}
\usepackage[normalem]{ulem}
\usepackage{amsmath}
\usepackage{textcomp}
\usepackage{amssymb}
\usepackage{capt-of}
\usepackage{hyperref}
\usepackage{minted}
\usepackage[hyperref,x11names]{xcolor}
\usepackage{physics}
\usepackage{cases}
\graphicspath{ {./} }
\usepackage{tikz}
\usetikzlibrary{arrows,plotmarks,calc,positioning,fit}
\usetikzlibrary{shapes.geometric, decorations.pathmorphing, patterns, backgrounds}
\newcommand{\tikzremember}[1]{{  \tikz[remember picture,overlay]{\node (#1) at (0,11pt) { };}}}
\tikzset{snake it/.style={decorate, decoration=snake}}
\usepackage[nottoc]{tocbibind}
\author{Tigany Zarrouk}
\date{\today}
\title{things-to-do}
\hypersetup{
 pdfauthor={Tigany Zarrouk},
 pdftitle={things-to-do},
 pdfkeywords={},
 pdfsubject={},
 pdfcreator={Emacs 25.3.1 (Org mode 9.2.2)}, 
 pdflang={English}}
\begin{document}

\maketitle
\tableofcontents

\section{Finite Electron Temperature}
\label{sec:org0bc5cb8}
\begin{enumerate}
\item Gillian 1986
\label{sec:org472883d}
\cite{Gillan1989}

Finite electron temperature for estimation of the band energy at zero kelvin. 

The finite-temperature scheme is merely a device, whose main purpose is to
smooth discontinuities at the Fermi level.

Really want to get the ground state energy \(E_0\), for small \(T\) the free
energy deviates from \(E_0\) by a term quadratic in \(T\): \(A = E_0 -
\frac{1}{2}\gamma T^2\) and that the deviation of the energy \(E\) ie equal and
opposite \(E = E_0 + \frac{1}{2}\gamma T^2\). Therefor the best estimate for the
ground-state energy will be \(\frac{1}{2}(E + A)\) of which deviation from \(E_0\)
will only be \(\mathcal{O}(T^3)\).

In the ground state the occupation numbers \(f_{\mathbf{q}i}\) are equal to 1
for \(\varepsilon_{\mathbf{q} i} < \mu\) and zero if \(\varepsilon_{\mathbf{q} i}
> \mu\), where \(\varepsilon_{\mathbf{q} i}\) are Kohn-Sham eigenvalues at
wavevector \(\mathbf{q}\) and \(\mu\) is the chemical potential (Fermi Energy
\(E_F\) ).

In the case of a metal, this discontinuity of \(f_{\mathbf{q}i}\) as a function
of energy is troublesome. Because of occupation numbers, the response
functions, (ASIDE: susceptibilities, linear response: think of polarisation
\(\mathbf{P} = \epsilon_0 \chi_{e} \mathbf{E}\), the measure of how
polarisable a material is, or how well it \emph{responds} to an electric field is
\(\chi\)) 

\[
\chi_2^0 \approx \frac{2}{\Omega} \sum_{\mathbf{q}} w_{\mathbf{q}}\sum_{G'} 
     \frac{ f_{\mathbf{q}}^0 + G'  - f_{\mathbf{q} + G + G'}^0
     }{\varepsilon_{\mathbf{q}}^0 +  G'  - \varepsilon_{\mathbf{q} + G + G'}^0 }
\]

where \(\varepsilon_{\mathbf{q} + G + G'}^0\) are non-interactng single-particle
energies and \(f_{\mathbf{k}}^0\) are the associated occupation numbers. 

The point is that Brillouin zone sampling is effective with a small number of
\(\mathbf{q}\) -vectors only if the funciton bein sampled is smooth in
\emph{reciprocal} space. But because of the occupation numbers, the response
function actually becomes far from smooth and is in fact discontinuous at zero
temperature. 

There is another related reason why we get trouble. Since we cannot know the
self-consistent eigenvalues in advance, we do not know how many occupied states
there will be at each \(\mathbf{q}\). As the iteration progresses to self-consistency, eigenvalues at
different \(\mathbf{q}\) will generally cross each other and the Fermi energy, and this would require
a discontinuous change of occupation numbers. Such discontinuities would presumably
play havoc with the minimisation scheme. 

The solution we have adopted to these problems is to allow the \(f_{\mathbf{q}i}\) to vary
continuously in the range (0, 1). This has the effect both of smoothing the sampled
function and of eliminating discontinuities due to level crossing. A convenient way of
formulating this idea is to consider the calculation formally at finite temperature, and
this is the reason for introducing the finite-temperature generalisation of
density functional theory. 

\item Horsfield 1996
\label{sec:orge3d424d}
\cite{Horsfield1996}

Used Gillian's technique \cite{Gillan1989} with finite electron temperature for estimation of
the band energy at zero kelvin. 


Hamiltonian scales badly with system size \(\mathcal{O}(N^3)\) for
diagonalisation. Linear scaling works well for covalent materials as the
density matrix is short ranged, but metals have had limited success other than
with some moment's methods.

Short ranged density matrices can be made short ranged with a finite electron
temperature, but leads to a weakening of the bonds, as electrons are promoted
to higher energy states, making dynamic unrealistic. 

For eletronic degrees of freedom, one can work sith the density matrix
\(\hat{\rho}\). The electron internal energy, \(U(T)\), electron entropy \(S(T)\)
and the electron free energy \(A(T)\) are 

\begin{LaTeX}
\begin{align}
U(T) &= 2 \text{Tr}\{ \hat{\rho} \hat{H} \} \\
S(T) &= 2 k_{\text{B}} \text{Tr}\{ s( \hat{\rho} ) \} \\
A(T) &= 2 \text{Tr}\{ \hat{\rho} \hat{H} - k_{\text{B}} T s( \hat{\rho} ) \},
\end{align}
\end{LaTeX}

where \(\hat{H}\) is the Hamiltonian and \(s(\hat{\rho})\) is the entropy density.

\begin{LaTeX}
\begin{align}
\hat{\rho} &= f(\hat{x}) = \big[ 1 + \text{exp}(\hat{x}) \big]^{-1} \\
s(f) &= - \big[ f\text{ln}f + (1-f)\text{ln}(1-f) \big] 
\end{align}
\end{LaTeX}

where \(\hat{x} = ( \hat{H} - \mu )/k_{\text{B}} T\).

We can have an approximation to the free energy calculated exactly from the
density matrix.

\[
A(0) = U_{0} \approx B(T) = A(T) + \frac{1}{2} T S(T)
\]

Defining a more general functional we have 
\[
\Phi_{\alpha}(T) = A(T) + \alpha T S(T)
\]

Taking the derivative of this with respect to position \(\mathbf{r}_i\), for an
atom \(i\), we get the contribution to the force, at a constant number of
electrons as 

\begin{LaTeX}
\begin{align}
\Phi_{\alpha}(T) &= 2 \text{Tr}\{ f(\hat{x}) \hat{H} - (1 - \alpha) k_{\text{B}} T s( f( \hat{x} ) )  \}\\
\mathbf{F}^{(\alpha)}_{i} &= -2 \text{Tr}\Big\{ \rho_{\text{eff}}^{(\alpha)} \frac{\partial \hat{H} }{ \partial \mathbf{r}_{i} } \Big\} 
\end{align}
\end{LaTeX}

where 

\[
\langle \hat{x} \rangle = \frac{ \text{Tr}\{ \hat{x}f'(\hat{x}) \}}{
      \text{Tr}\{ f'(\hat{x}) \} }
\]
and
\[
\rho_{\text{eff}}^{(\alpha)} = [ f(\hat{x}) + \alpha ( \hat{x} - \langle
           \hat{x} \rangle ) f'(\hat{x})]
\]
 is an effective density matrix. 

\(\rho_{\text{eff}}^{(0.5)}\) appears to be an optimall approximation to the
ground-state density matrix. 

This works well if the electron temperature is less than \(10\%\) of the
bandwidth.
\end{enumerate}
\end{document}
